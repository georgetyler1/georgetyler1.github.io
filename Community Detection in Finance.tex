---


---

<p>\title{A study of Bethe Hessian Spectral Clustering for Detecting Block Structure with an Application to Financial Markets}</p>
<p>% Use letters for affiliations, numbers to show equal authorship (if applicable) and to indicate the corresponding author</p>
<p>% Please give the surname of the lead author for the running footer</p>
<p>% Please add a significance statement to explain the relevance of your work<br>
\significancestatement{Block structure in networks is a generalisation of the classical definition of community. We explore a powerful spectral method for detecting block structure and demonstrate an application of how it might be used to better understand financial markets.}</p>
<p>% At least three keywords are required at submission. Please provide three to five keywords, separated by the pipe symbol.<br>
\keywords{Bethe Hessian <span class="katex--inline"><span class="katex"><span class="katex-mathml"><math xmlns="http://www.w3.org/1998/Math/MathML"><semantics><mrow><mi mathvariant="normal">∣</mi></mrow><annotation encoding="application/x-tex">|</annotation></semantics></math></span><span class="katex-html" aria-hidden="true"><span class="base"><span class="strut" style="height: 1em; vertical-align: -0.25em;"></span><span class="mord">∣</span></span></span></span></span> Stochastic Block Models <span class="katex--inline"><span class="katex"><span class="katex-mathml"><math xmlns="http://www.w3.org/1998/Math/MathML"><semantics><mrow><mi mathvariant="normal">∣</mi></mrow><annotation encoding="application/x-tex">|</annotation></semantics></math></span><span class="katex-html" aria-hidden="true"><span class="base"><span class="strut" style="height: 1em; vertical-align: -0.25em;"></span><span class="mord">∣</span></span></span></span></span> Spectral Clustering }</p>
<p>\begin{abstract}<br>
Community detection in networks has been the subject of much research. As a result, a wide variety of detection approaches are available, such as modularity maximization and spectral clustering. However, in more recent years, the definition of a community in a network has been revised and has come to take on a more general definition.<br>
In this article we explore newer methods for detecting assortative and disassortative communities within complex networks. We focus upon spectral methods for detecting this sort of block structure involving the Non-Backtracking matrix and in particular the Bethe Hessian matrix. We provide numerical results illustrating the power of using the Bethe Hessian clustering algorithm described in \cite{SaadeBethe} including an interesting application to investment portfolios and stock market data.<br>
\end{abstract}<br>
\begin{document}</p>
<p>\maketitle<br>
\thispagestyle{firststyle}<br>
\ifthenelse{\boolean{shortarticle}}{\ifthenelse{\boolean{singlecolumn}}{\abscontentformatted}{\abscontent}}{}</p>
<p>% If your first paragraph (i.e. with the \dropcap) contains a list environment (quote, quotation, theorem, definition, enumerate, itemize…), the line after the list may have some extra indentation. If this is the case, add \parshape=0 to the end of the list environment.</p>
<p>\section{Introduction}</p>
<p>Community detection has been a huge part of recent research in networks. Typically, a community is defined simply as a subset of nodes in a network which have more connections between them than with nodes outside the community \cite{NetworkNotes}. However, as loose as this definition may sound, it is still rather restrictive. For instance, envision a directed network of buyers, sellers and farmers of bananas, with links pointed in the direction of one agent selling some bananas to another. Sellers may be selling to each other and buyers, producers would be selling to sellers. These three groups make the most sensible community groupings yet by some standard community detection algorithms they would go undetected. Indeed, the producers and buyers of bananas may not have any links between each other at all!<br>
\newline<br>
\subsection{Preliminary Definitions}<br>
Let <span class="katex--inline"><span class="katex"><span class="katex-mathml"><math xmlns="http://www.w3.org/1998/Math/MathML"><semantics><mrow><mi mathvariant="script">G</mi><mo stretchy="false">(</mo><mi>V</mi><mo separator="true">,</mo><mi>E</mi><mo stretchy="false">)</mo></mrow><annotation encoding="application/x-tex">\mathcal{G}(V,E)</annotation></semantics></math></span><span class="katex-html" aria-hidden="true"><span class="base"><span class="strut" style="height: 1em; vertical-align: -0.25em;"></span><span class="mord mathcal" style="margin-right: 0.0593em;">G</span><span class="mopen">(</span><span class="mord mathnormal" style="margin-right: 0.22222em;">V</span><span class="mpunct">,</span><span class="mspace" style="margin-right: 0.166667em;"></span><span class="mord mathnormal" style="margin-right: 0.05764em;">E</span><span class="mclose">)</span></span></span></span></span> be a graph with nodes <span class="katex--inline"><span class="katex"><span class="katex-mathml"><math xmlns="http://www.w3.org/1998/Math/MathML"><semantics><mrow><mi>V</mi></mrow><annotation encoding="application/x-tex">V</annotation></semantics></math></span><span class="katex-html" aria-hidden="true"><span class="base"><span class="strut" style="height: 0.68333em; vertical-align: 0em;"></span><span class="mord mathnormal" style="margin-right: 0.22222em;">V</span></span></span></span></span> and links <span class="katex--inline"><span class="katex"><span class="katex-mathml"><math xmlns="http://www.w3.org/1998/Math/MathML"><semantics><mrow><mi>E</mi><mo>⊆</mo><mi>V</mi><mo>×</mo><mi>V</mi></mrow><annotation encoding="application/x-tex">E \subseteq V \times V</annotation></semantics></math></span><span class="katex-html" aria-hidden="true"><span class="base"><span class="strut" style="height: 0.8193em; vertical-align: -0.13597em;"></span><span class="mord mathnormal" style="margin-right: 0.05764em;">E</span><span class="mspace" style="margin-right: 0.277778em;"></span><span class="mrel">⊆</span><span class="mspace" style="margin-right: 0.277778em;"></span></span><span class="base"><span class="strut" style="height: 0.76666em; vertical-align: -0.08333em;"></span><span class="mord mathnormal" style="margin-right: 0.22222em;">V</span><span class="mspace" style="margin-right: 0.222222em;"></span><span class="mbin">×</span><span class="mspace" style="margin-right: 0.222222em;"></span></span><span class="base"><span class="strut" style="height: 0.68333em; vertical-align: 0em;"></span><span class="mord mathnormal" style="margin-right: 0.22222em;">V</span></span></span></span></span>. Given nodes <span class="katex--inline"><span class="katex"><span class="katex-mathml"><math xmlns="http://www.w3.org/1998/Math/MathML"><semantics><mrow><mi>u</mi><mo separator="true">,</mo><mi>v</mi><mo separator="true">,</mo><mi>w</mi><mo>∈</mo><mi>V</mi></mrow><annotation encoding="application/x-tex">u,v, w \in V</annotation></semantics></math></span><span class="katex-html" aria-hidden="true"><span class="base"><span class="strut" style="height: 0.73354em; vertical-align: -0.19444em;"></span><span class="mord mathnormal">u</span><span class="mpunct">,</span><span class="mspace" style="margin-right: 0.166667em;"></span><span class="mord mathnormal" style="margin-right: 0.03588em;">v</span><span class="mpunct">,</span><span class="mspace" style="margin-right: 0.166667em;"></span><span class="mord mathnormal" style="margin-right: 0.02691em;">w</span><span class="mspace" style="margin-right: 0.277778em;"></span><span class="mrel">∈</span><span class="mspace" style="margin-right: 0.277778em;"></span></span><span class="base"><span class="strut" style="height: 0.68333em; vertical-align: 0em;"></span><span class="mord mathnormal" style="margin-right: 0.22222em;">V</span></span></span></span></span>, an equivalence relation <span class="katex--inline"><span class="katex"><span class="katex-mathml"><math xmlns="http://www.w3.org/1998/Math/MathML"><semantics><mrow><mo>∼</mo></mrow><annotation encoding="application/x-tex">\sim</annotation></semantics></math></span><span class="katex-html" aria-hidden="true"><span class="base"><span class="strut" style="height: 0.36687em; vertical-align: 0em;"></span><span class="mrel">∼</span></span></span></span></span> is a relation between nodes which is reflexive: <span class="katex--inline"><span class="katex"><span class="katex-mathml"><math xmlns="http://www.w3.org/1998/Math/MathML"><semantics><mrow><mi>u</mi><mo>∼</mo><mi>u</mi></mrow><annotation encoding="application/x-tex">u  \sim u</annotation></semantics></math></span><span class="katex-html" aria-hidden="true"><span class="base"><span class="strut" style="height: 0.43056em; vertical-align: 0em;"></span><span class="mord mathnormal">u</span><span class="mspace" style="margin-right: 0.277778em;"></span><span class="mrel">∼</span><span class="mspace" style="margin-right: 0.277778em;"></span></span><span class="base"><span class="strut" style="height: 0.43056em; vertical-align: 0em;"></span><span class="mord mathnormal">u</span></span></span></span></span>, symmetric: <span class="katex--inline"><span class="katex"><span class="katex-mathml"><math xmlns="http://www.w3.org/1998/Math/MathML"><semantics><mrow><mi>u</mi><mo>∼</mo><mi>v</mi>  <mo>⟺</mo>  <mi>v</mi><mo>∼</mo><mi>u</mi></mrow><annotation encoding="application/x-tex">u \sim v \iff v \sim u</annotation></semantics></math></span><span class="katex-html" aria-hidden="true"><span class="base"><span class="strut" style="height: 0.43056em; vertical-align: 0em;"></span><span class="mord mathnormal">u</span><span class="mspace" style="margin-right: 0.277778em;"></span><span class="mrel">∼</span><span class="mspace" style="margin-right: 0.277778em;"></span></span><span class="base"><span class="strut" style="height: 0.549em; vertical-align: -0.024em;"></span><span class="mord mathnormal" style="margin-right: 0.03588em;">v</span><span class="mspace" style="margin-right: 0.277778em;"></span><span class="mspace" style="margin-right: 0.277778em;"></span><span class="mrel">⟺</span><span class="mspace" style="margin-right: 0.277778em;"></span><span class="mspace" style="margin-right: 0.277778em;"></span></span><span class="base"><span class="strut" style="height: 0.43056em; vertical-align: 0em;"></span><span class="mord mathnormal" style="margin-right: 0.03588em;">v</span><span class="mspace" style="margin-right: 0.277778em;"></span><span class="mrel">∼</span><span class="mspace" style="margin-right: 0.277778em;"></span></span><span class="base"><span class="strut" style="height: 0.43056em; vertical-align: 0em;"></span><span class="mord mathnormal">u</span></span></span></span></span> and transitive: <span class="katex--inline"><span class="katex"><span class="katex-mathml"><math xmlns="http://www.w3.org/1998/Math/MathML"><semantics><mrow><mi>u</mi><mo>∼</mo><mi>v</mi></mrow><annotation encoding="application/x-tex">u \sim v</annotation></semantics></math></span><span class="katex-html" aria-hidden="true"><span class="base"><span class="strut" style="height: 0.43056em; vertical-align: 0em;"></span><span class="mord mathnormal">u</span><span class="mspace" style="margin-right: 0.277778em;"></span><span class="mrel">∼</span><span class="mspace" style="margin-right: 0.277778em;"></span></span><span class="base"><span class="strut" style="height: 0.43056em; vertical-align: 0em;"></span><span class="mord mathnormal" style="margin-right: 0.03588em;">v</span></span></span></span></span> and <span class="katex--inline"><span class="katex"><span class="katex-mathml"><math xmlns="http://www.w3.org/1998/Math/MathML"><semantics><mrow><mi>v</mi><mo>∼</mo><mi>w</mi></mrow><annotation encoding="application/x-tex">v \sim w</annotation></semantics></math></span><span class="katex-html" aria-hidden="true"><span class="base"><span class="strut" style="height: 0.43056em; vertical-align: 0em;"></span><span class="mord mathnormal" style="margin-right: 0.03588em;">v</span><span class="mspace" style="margin-right: 0.277778em;"></span><span class="mrel">∼</span><span class="mspace" style="margin-right: 0.277778em;"></span></span><span class="base"><span class="strut" style="height: 0.43056em; vertical-align: 0em;"></span><span class="mord mathnormal" style="margin-right: 0.02691em;">w</span></span></span></span></span> <span class="katex--inline"><span class="katex"><span class="katex-mathml"><math xmlns="http://www.w3.org/1998/Math/MathML"><semantics><mrow>  <mo>⟺</mo>  <mi>u</mi><mo>∼</mo><mi>w</mi></mrow><annotation encoding="application/x-tex">\iff u \sim w</annotation></semantics></math></span><span class="katex-html" aria-hidden="true"><span class="base"><span class="strut" style="height: 0.549em; vertical-align: -0.024em;"></span><span class="mspace" style="margin-right: 0.277778em;"></span><span class="mrel">⟺</span><span class="mspace" style="margin-right: 0.277778em;"></span><span class="mspace" style="margin-right: 0.277778em;"></span></span><span class="base"><span class="strut" style="height: 0.43056em; vertical-align: 0em;"></span><span class="mord mathnormal">u</span><span class="mspace" style="margin-right: 0.277778em;"></span><span class="mrel">∼</span><span class="mspace" style="margin-right: 0.277778em;"></span></span><span class="base"><span class="strut" style="height: 0.43056em; vertical-align: 0em;"></span><span class="mord mathnormal" style="margin-right: 0.02691em;">w</span></span></span></span></span>.<br>
The term block comes from the appearance of the adjacency matrix of a network with block structure - under a certain similarity transform, it will have a sparsity pattern of large blocks of concentrated nonzero entries. The detection of communities by the classical definition is equivalent to determining the similarity transform of the adjacency matrix that reduces it as close to being block diagonal as possible. These communities are described as ‘assortative’ and can be detected via a number of means. One approach involves searching for a partition of the network that maximizes the Newman Modularity \cite{newman_2006}. Communities that correspond to the off-diagonal blocks of the adjacency matrix  have a higher concentration of links out of their community to another than they do within their community. These are termed disassortative.</p>
<p>\subsection{Stochastic Block Models (SBM)}<br>
A Stochastic Block Model is a suggested model for generating random graphs with block structure. The idea is simple: given a set of <span class="katex--inline"><span class="katex"><span class="katex-mathml"><math xmlns="http://www.w3.org/1998/Math/MathML"><semantics><mrow><mi>n</mi></mrow><annotation encoding="application/x-tex">n</annotation></semantics></math></span><span class="katex-html" aria-hidden="true"><span class="base"><span class="strut" style="height: 0.43056em; vertical-align: 0em;"></span><span class="mord mathnormal">n</span></span></span></span></span> nodes, partition these nodes into <span class="katex--inline"><span class="katex"><span class="katex-mathml"><math xmlns="http://www.w3.org/1998/Math/MathML"><semantics><mrow><mi>k</mi></mrow><annotation encoding="application/x-tex">k</annotation></semantics></math></span><span class="katex-html" aria-hidden="true"><span class="base"><span class="strut" style="height: 0.69444em; vertical-align: 0em;"></span><span class="mord mathnormal" style="margin-right: 0.03148em;">k</span></span></span></span></span> disjoint communities<br>
<span class="katex--inline">KaTeX parse error: Undefined control sequence: \hdots at position 8: \{C_1, \̲h̲d̲o̲t̲s̲, C_k\}</span> and define a symmetric matrix <span class="katex--inline"><span class="katex"><span class="katex-mathml"><math xmlns="http://www.w3.org/1998/Math/MathML"><semantics><mrow><mi>P</mi><mo>∈</mo><msup><mi mathvariant="double-struck">R</mi><mrow><mi>k</mi><mo>×</mo><mi>k</mi></mrow></msup></mrow><annotation encoding="application/x-tex">P \in \mathbb{R}^{k \times k}</annotation></semantics></math></span><span class="katex-html" aria-hidden="true"><span class="base"><span class="strut" style="height: 0.72243em; vertical-align: -0.0391em;"></span><span class="mord mathnormal" style="margin-right: 0.13889em;">P</span><span class="mspace" style="margin-right: 0.277778em;"></span><span class="mrel">∈</span><span class="mspace" style="margin-right: 0.277778em;"></span></span><span class="base"><span class="strut" style="height: 0.849108em; vertical-align: 0em;"></span><span class="mord"><span class="mord mathbb">R</span><span class="msupsub"><span class="vlist-t"><span class="vlist-r"><span class="vlist" style="height: 0.849108em;"><span class="" style="top: -3.063em; margin-right: 0.05em;"><span class="pstrut" style="height: 2.7em;"></span><span class="sizing reset-size6 size3 mtight"><span class="mord mtight"><span class="mord mathnormal mtight" style="margin-right: 0.03148em;">k</span><span class="mbin mtight">×</span><span class="mord mathnormal mtight" style="margin-right: 0.03148em;">k</span></span></span></span></span></span></span></span></span></span></span></span></span> where <span class="katex--inline"><span class="katex"><span class="katex-mathml"><math xmlns="http://www.w3.org/1998/Math/MathML"><semantics><mrow><msub><mi>P</mi><mrow><mi>i</mi><mi>j</mi></mrow></msub></mrow><annotation encoding="application/x-tex">P_{ij}</annotation></semantics></math></span><span class="katex-html" aria-hidden="true"><span class="base"><span class="strut" style="height: 0.969438em; vertical-align: -0.286108em;"></span><span class="mord"><span class="mord mathnormal" style="margin-right: 0.13889em;">P</span><span class="msupsub"><span class="vlist-t vlist-t2"><span class="vlist-r"><span class="vlist" style="height: 0.311664em;"><span class="" style="top: -2.55em; margin-left: -0.13889em; margin-right: 0.05em;"><span class="pstrut" style="height: 2.7em;"></span><span class="sizing reset-size6 size3 mtight"><span class="mord mtight"><span class="mord mathnormal mtight" style="margin-right: 0.05724em;">ij</span></span></span></span></span><span class="vlist-s">​</span></span><span class="vlist-r"><span class="vlist" style="height: 0.286108em;"><span class=""></span></span></span></span></span></span></span></span></span></span> is the probability a node from <span class="katex--inline"><span class="katex"><span class="katex-mathml"><math xmlns="http://www.w3.org/1998/Math/MathML"><semantics><mrow><msub><mi>C</mi><mi>i</mi></msub></mrow><annotation encoding="application/x-tex">C_i</annotation></semantics></math></span><span class="katex-html" aria-hidden="true"><span class="base"><span class="strut" style="height: 0.83333em; vertical-align: -0.15em;"></span><span class="mord"><span class="mord mathnormal" style="margin-right: 0.07153em;">C</span><span class="msupsub"><span class="vlist-t vlist-t2"><span class="vlist-r"><span class="vlist" style="height: 0.311664em;"><span class="" style="top: -2.55em; margin-left: -0.07153em; margin-right: 0.05em;"><span class="pstrut" style="height: 2.7em;"></span><span class="sizing reset-size6 size3 mtight"><span class="mord mathnormal mtight">i</span></span></span></span><span class="vlist-s">​</span></span><span class="vlist-r"><span class="vlist" style="height: 0.15em;"><span class=""></span></span></span></span></span></span></span></span></span></span> and <span class="katex--inline"><span class="katex"><span class="katex-mathml"><math xmlns="http://www.w3.org/1998/Math/MathML"><semantics><mrow><msub><mi>C</mi><mi>j</mi></msub></mrow><annotation encoding="application/x-tex">C_j</annotation></semantics></math></span><span class="katex-html" aria-hidden="true"><span class="base"><span class="strut" style="height: 0.969438em; vertical-align: -0.286108em;"></span><span class="mord"><span class="mord mathnormal" style="margin-right: 0.07153em;">C</span><span class="msupsub"><span class="vlist-t vlist-t2"><span class="vlist-r"><span class="vlist" style="height: 0.311664em;"><span class="" style="top: -2.55em; margin-left: -0.07153em; margin-right: 0.05em;"><span class="pstrut" style="height: 2.7em;"></span><span class="sizing reset-size6 size3 mtight"><span class="mord mathnormal mtight" style="margin-right: 0.05724em;">j</span></span></span></span><span class="vlist-s">​</span></span><span class="vlist-r"><span class="vlist" style="height: 0.286108em;"><span class=""></span></span></span></span></span></span></span></span></span></span> share a link. Alternatively, a vector <span class="katex--inline"><span class="katex"><span class="katex-mathml"><math xmlns="http://www.w3.org/1998/Math/MathML"><semantics><mrow><mi>α</mi><mo>∈</mo><msup><mi mathvariant="double-struck">R</mi><mi>r</mi></msup></mrow><annotation encoding="application/x-tex">\alpha \in \mathbb{R}^r</annotation></semantics></math></span><span class="katex-html" aria-hidden="true"><span class="base"><span class="strut" style="height: 0.5782em; vertical-align: -0.0391em;"></span><span class="mord mathnormal" style="margin-right: 0.0037em;">α</span><span class="mspace" style="margin-right: 0.277778em;"></span><span class="mrel">∈</span><span class="mspace" style="margin-right: 0.277778em;"></span></span><span class="base"><span class="strut" style="height: 0.68889em; vertical-align: 0em;"></span><span class="mord"><span class="mord mathbb">R</span><span class="msupsub"><span class="vlist-t"><span class="vlist-r"><span class="vlist" style="height: 0.664392em;"><span class="" style="top: -3.063em; margin-right: 0.05em;"><span class="pstrut" style="height: 2.7em;"></span><span class="sizing reset-size6 size3 mtight"><span class="mord mathnormal mtight" style="margin-right: 0.02778em;">r</span></span></span></span></span></span></span></span></span></span></span></span> can be given to specify the probability distribution of the nodes into the communities. The number of edges of each pair of nodes <span class="katex--inline"><span class="katex"><span class="katex-mathml"><math xmlns="http://www.w3.org/1998/Math/MathML"><semantics><mrow><mi>i</mi></mrow><annotation encoding="application/x-tex">i</annotation></semantics></math></span><span class="katex-html" aria-hidden="true"><span class="base"><span class="strut" style="height: 0.65952em; vertical-align: 0em;"></span><span class="mord mathnormal">i</span></span></span></span></span> and <span class="katex--inline"><span class="katex"><span class="katex-mathml"><math xmlns="http://www.w3.org/1998/Math/MathML"><semantics><mrow><mi>j</mi></mrow><annotation encoding="application/x-tex">j</annotation></semantics></math></span><span class="katex-html" aria-hidden="true"><span class="base"><span class="strut" style="height: 0.85396em; vertical-align: -0.19444em;"></span><span class="mord mathnormal" style="margin-right: 0.05724em;">j</span></span></span></span></span> is given independently by a Poisson distribution with a mean <span class="katex--inline"><span class="katex"><span class="katex-mathml"><math xmlns="http://www.w3.org/1998/Math/MathML"><semantics><mrow><msub><mi>ω</mi><mrow><mi>r</mi><mi>s</mi></mrow></msub><mo>=</mo><mi mathvariant="double-struck">E</mi><mo stretchy="false">(</mo><msub><mi>A</mi><mrow><mi>i</mi><mi>j</mi></mrow></msub><mo stretchy="false">)</mo></mrow><annotation encoding="application/x-tex">\omega_{rs} = \mathbb{E}(A_{ij})</annotation></semantics></math></span><span class="katex-html" aria-hidden="true"><span class="base"><span class="strut" style="height: 0.58056em; vertical-align: -0.15em;"></span><span class="mord"><span class="mord mathnormal" style="margin-right: 0.03588em;">ω</span><span class="msupsub"><span class="vlist-t vlist-t2"><span class="vlist-r"><span class="vlist" style="height: 0.151392em;"><span class="" style="top: -2.55em; margin-left: -0.03588em; margin-right: 0.05em;"><span class="pstrut" style="height: 2.7em;"></span><span class="sizing reset-size6 size3 mtight"><span class="mord mtight"><span class="mord mathnormal mtight">rs</span></span></span></span></span><span class="vlist-s">​</span></span><span class="vlist-r"><span class="vlist" style="height: 0.15em;"><span class=""></span></span></span></span></span></span><span class="mspace" style="margin-right: 0.277778em;"></span><span class="mrel">=</span><span class="mspace" style="margin-right: 0.277778em;"></span></span><span class="base"><span class="strut" style="height: 1.03611em; vertical-align: -0.286108em;"></span><span class="mord mathbb">E</span><span class="mopen">(</span><span class="mord"><span class="mord mathnormal">A</span><span class="msupsub"><span class="vlist-t vlist-t2"><span class="vlist-r"><span class="vlist" style="height: 0.311664em;"><span class="" style="top: -2.55em; margin-left: 0em; margin-right: 0.05em;"><span class="pstrut" style="height: 2.7em;"></span><span class="sizing reset-size6 size3 mtight"><span class="mord mtight"><span class="mord mathnormal mtight" style="margin-right: 0.05724em;">ij</span></span></span></span></span><span class="vlist-s">​</span></span><span class="vlist-r"><span class="vlist" style="height: 0.286108em;"><span class=""></span></span></span></span></span></span><span class="mclose">)</span></span></span></span></span> where nodes <span class="katex--inline"><span class="katex"><span class="katex-mathml"><math xmlns="http://www.w3.org/1998/Math/MathML"><semantics><mrow><mi>i</mi></mrow><annotation encoding="application/x-tex">i</annotation></semantics></math></span><span class="katex-html" aria-hidden="true"><span class="base"><span class="strut" style="height: 0.65952em; vertical-align: 0em;"></span><span class="mord mathnormal">i</span></span></span></span></span> and <span class="katex--inline"><span class="katex"><span class="katex-mathml"><math xmlns="http://www.w3.org/1998/Math/MathML"><semantics><mrow><mi>j</mi></mrow><annotation encoding="application/x-tex">j</annotation></semantics></math></span><span class="katex-html" aria-hidden="true"><span class="base"><span class="strut" style="height: 0.85396em; vertical-align: -0.19444em;"></span><span class="mord mathnormal" style="margin-right: 0.05724em;">j</span></span></span></span></span> are members of <span class="katex--inline"><span class="katex"><span class="katex-mathml"><math xmlns="http://www.w3.org/1998/Math/MathML"><semantics><mrow><msub><mi>C</mi><mi>r</mi></msub></mrow><annotation encoding="application/x-tex">C_r</annotation></semantics></math></span><span class="katex-html" aria-hidden="true"><span class="base"><span class="strut" style="height: 0.83333em; vertical-align: -0.15em;"></span><span class="mord"><span class="mord mathnormal" style="margin-right: 0.07153em;">C</span><span class="msupsub"><span class="vlist-t vlist-t2"><span class="vlist-r"><span class="vlist" style="height: 0.151392em;"><span class="" style="top: -2.55em; margin-left: -0.07153em; margin-right: 0.05em;"><span class="pstrut" style="height: 2.7em;"></span><span class="sizing reset-size6 size3 mtight"><span class="mord mathnormal mtight" style="margin-right: 0.02778em;">r</span></span></span></span><span class="vlist-s">​</span></span><span class="vlist-r"><span class="vlist" style="height: 0.15em;"><span class=""></span></span></span></span></span></span></span></span></span></span> and <span class="katex--inline"><span class="katex"><span class="katex-mathml"><math xmlns="http://www.w3.org/1998/Math/MathML"><semantics><mrow><msub><mi>C</mi><mi>s</mi></msub></mrow><annotation encoding="application/x-tex">C_s</annotation></semantics></math></span><span class="katex-html" aria-hidden="true"><span class="base"><span class="strut" style="height: 0.83333em; vertical-align: -0.15em;"></span><span class="mord"><span class="mord mathnormal" style="margin-right: 0.07153em;">C</span><span class="msupsub"><span class="vlist-t vlist-t2"><span class="vlist-r"><span class="vlist" style="height: 0.151392em;"><span class="" style="top: -2.55em; margin-left: -0.07153em; margin-right: 0.05em;"><span class="pstrut" style="height: 2.7em;"></span><span class="sizing reset-size6 size3 mtight"><span class="mord mathnormal mtight">s</span></span></span></span><span class="vlist-s">​</span></span><span class="vlist-r"><span class="vlist" style="height: 0.15em;"><span class=""></span></span></span></span></span></span></span></span></span></span> respectively and <span class="katex--inline"><span class="katex"><span class="katex-mathml"><math xmlns="http://www.w3.org/1998/Math/MathML"><semantics><mrow><msub><mi>A</mi><mrow><mi>i</mi><mi>j</mi></mrow></msub></mrow><annotation encoding="application/x-tex">A_{ij}</annotation></semantics></math></span><span class="katex-html" aria-hidden="true"><span class="base"><span class="strut" style="height: 0.969438em; vertical-align: -0.286108em;"></span><span class="mord"><span class="mord mathnormal">A</span><span class="msupsub"><span class="vlist-t vlist-t2"><span class="vlist-r"><span class="vlist" style="height: 0.311664em;"><span class="" style="top: -2.55em; margin-left: 0em; margin-right: 0.05em;"><span class="pstrut" style="height: 2.7em;"></span><span class="sizing reset-size6 size3 mtight"><span class="mord mtight"><span class="mord mathnormal mtight" style="margin-right: 0.05724em;">ij</span></span></span></span></span><span class="vlist-s">​</span></span><span class="vlist-r"><span class="vlist" style="height: 0.286108em;"><span class=""></span></span></span></span></span></span></span></span></span></span> is the number of links between said nodes. We will subsequently denote the community that node <span class="katex--inline"><span class="katex"><span class="katex-mathml"><math xmlns="http://www.w3.org/1998/Math/MathML"><semantics><mrow><mi>i</mi></mrow><annotation encoding="application/x-tex">i</annotation></semantics></math></span><span class="katex-html" aria-hidden="true"><span class="base"><span class="strut" style="height: 0.65952em; vertical-align: 0em;"></span><span class="mord mathnormal">i</span></span></span></span></span> has been assigned as <span class="katex--inline"><span class="katex"><span class="katex-mathml"><math xmlns="http://www.w3.org/1998/Math/MathML"><semantics><mrow><msub><mi>g</mi><mi>i</mi></msub></mrow><annotation encoding="application/x-tex">g_i</annotation></semantics></math></span><span class="katex-html" aria-hidden="true"><span class="base"><span class="strut" style="height: 0.625em; vertical-align: -0.19444em;"></span><span class="mord"><span class="mord mathnormal" style="margin-right: 0.03588em;">g</span><span class="msupsub"><span class="vlist-t vlist-t2"><span class="vlist-r"><span class="vlist" style="height: 0.311664em;"><span class="" style="top: -2.55em; margin-left: -0.03588em; margin-right: 0.05em;"><span class="pstrut" style="height: 2.7em;"></span><span class="sizing reset-size6 size3 mtight"><span class="mord mathnormal mtight">i</span></span></span></span><span class="vlist-s">​</span></span><span class="vlist-r"><span class="vlist" style="height: 0.15em;"><span class=""></span></span></span></span></span></span></span></span></span></span>. See Figure \ref{coreperiphery} for an illustration of 3 different types of SBM.<br>
\newline<br>
Now consider a network G with <span class="katex--inline"><span class="katex"><span class="katex-mathml"><math xmlns="http://www.w3.org/1998/Math/MathML"><semantics><mrow><mi>n</mi></mrow><annotation encoding="application/x-tex">n</annotation></semantics></math></span><span class="katex-html" aria-hidden="true"><span class="base"><span class="strut" style="height: 0.43056em; vertical-align: 0em;"></span><span class="mord mathnormal">n</span></span></span></span></span> nodes and corresponding adjacency matrix <span class="katex--inline"><span class="katex"><span class="katex-mathml"><math xmlns="http://www.w3.org/1998/Math/MathML"><semantics><mrow><mi>A</mi></mrow><annotation encoding="application/x-tex">A</annotation></semantics></math></span><span class="katex-html" aria-hidden="true"><span class="base"><span class="strut" style="height: 0.68333em; vertical-align: 0em;"></span><span class="mord mathnormal">A</span></span></span></span></span>. As demonstrated in \cite{karrer_newman_2011}, the probability that G is generated given a <span class="katex--inline"><span class="katex"><span class="katex-mathml"><math xmlns="http://www.w3.org/1998/Math/MathML"><semantics><mrow><mi>n</mi><mo>×</mo><mi>n</mi></mrow><annotation encoding="application/x-tex">n \times n</annotation></semantics></math></span><span class="katex-html" aria-hidden="true"><span class="base"><span class="strut" style="height: 0.66666em; vertical-align: -0.08333em;"></span><span class="mord mathnormal">n</span><span class="mspace" style="margin-right: 0.222222em;"></span><span class="mbin">×</span><span class="mspace" style="margin-right: 0.222222em;"></span></span><span class="base"><span class="strut" style="height: 0.43056em; vertical-align: 0em;"></span><span class="mord mathnormal">n</span></span></span></span></span> matrix of parameters <span class="katex--inline"><span class="katex"><span class="katex-mathml"><math xmlns="http://www.w3.org/1998/Math/MathML"><semantics><mrow><mo stretchy="false">(</mo><msub><mi>ω</mi><mrow><msub><mi>g</mi><mi>i</mi></msub><msub><mi>g</mi><mi>j</mi></msub></mrow></msub><mo stretchy="false">)</mo></mrow><annotation encoding="application/x-tex">(\omega_{g_i g_j})</annotation></semantics></math></span><span class="katex-html" aria-hidden="true"><span class="base"><span class="strut" style="height: 1.09732em; vertical-align: -0.34732em;"></span><span class="mopen">(</span><span class="mord"><span class="mord mathnormal" style="margin-right: 0.03588em;">ω</span><span class="msupsub"><span class="vlist-t vlist-t2"><span class="vlist-r"><span class="vlist" style="height: 0.151392em;"><span class="" style="top: -2.55em; margin-left: -0.03588em; margin-right: 0.05em;"><span class="pstrut" style="height: 2.7em;"></span><span class="sizing reset-size6 size3 mtight"><span class="mord mtight"><span class="mord mtight"><span class="mord mathnormal mtight" style="margin-right: 0.03588em;">g</span><span class="msupsub"><span class="vlist-t vlist-t2"><span class="vlist-r"><span class="vlist" style="height: 0.328086em;"><span class="" style="top: -2.357em; margin-left: -0.03588em; margin-right: 0.0714286em;"><span class="pstrut" style="height: 2.5em;"></span><span class="sizing reset-size3 size1 mtight"><span class="mord mathnormal mtight">i</span></span></span></span><span class="vlist-s">​</span></span><span class="vlist-r"><span class="vlist" style="height: 0.143em;"><span class=""></span></span></span></span></span></span><span class="mord mtight"><span class="mord mathnormal mtight" style="margin-right: 0.03588em;">g</span><span class="msupsub"><span class="vlist-t vlist-t2"><span class="vlist-r"><span class="vlist" style="height: 0.328086em;"><span class="" style="top: -2.357em; margin-left: -0.03588em; margin-right: 0.0714286em;"><span class="pstrut" style="height: 2.5em;"></span><span class="sizing reset-size3 size1 mtight"><span class="mord mathnormal mtight" style="margin-right: 0.05724em;">j</span></span></span></span><span class="vlist-s">​</span></span><span class="vlist-r"><span class="vlist" style="height: 0.281886em;"><span class=""></span></span></span></span></span></span></span></span></span></span><span class="vlist-s">​</span></span><span class="vlist-r"><span class="vlist" style="height: 0.34732em;"><span class=""></span></span></span></span></span></span><span class="mclose">)</span></span></span></span></span> and vector of community assignments <span class="katex--inline">KaTeX parse error: Undefined control sequence: \hdots at position 12: g \in \{1, \̲h̲d̲o̲t̲s̲ ̲k \}^n</span> is:</p>
<p><span class="katex--display"><span class="katex-display"><span class="katex"><span class="katex-mathml"><math xmlns="http://www.w3.org/1998/Math/MathML" display="block"><semantics><mrow><mi>P</mi><mo stretchy="false">(</mo><mi>G</mi><mi mathvariant="normal">∣</mi><mi>ω</mi><mo separator="true">,</mo><mi>g</mi><mo stretchy="false">)</mo><mo>=</mo><mfrac><mrow><munder><mo>∏</mo><mrow><mi>r</mi><mo separator="true">,</mo><mi>s</mi></mrow></munder><msubsup><mi>ω</mi><mrow><mi>r</mi><mi>s</mi></mrow><mrow><msub><mi>m</mi><mrow><mi>r</mi><mi>s</mi></mrow></msub><mi mathvariant="normal">/</mi><mn>2</mn></mrow></msubsup><mi>e</mi><mi>x</mi><mi>p</mi><mo stretchy="false">(</mo><mfrac><mrow><mo>−</mo><mn>1</mn></mrow><mn>2</mn></mfrac><msub><mi>n</mi><mi>r</mi></msub><msub><mi>n</mi><mi>s</mi></msub><msub><mi>ω</mi><mrow><mi>r</mi><mi>s</mi></mrow></msub><mo stretchy="false">)</mo></mrow><mrow><munder><mo>∏</mo><mrow><mi>i</mi><mo>&lt;</mo><mi>j</mi></mrow></munder><msub><mi>A</mi><mrow><mi>i</mi><mi>j</mi></mrow></msub><mo stretchy="false">!</mo><munder><mo>∏</mo><mi>i</mi></munder><msup><mn>2</mn><mrow><msub><mi>A</mi><mrow><mi>i</mi><mi>i</mi></mrow></msub><mi mathvariant="normal">/</mi><mn>2</mn></mrow></msup><mo stretchy="false">(</mo><msub><mi>A</mi><mrow><mi>i</mi><mi>i</mi></mrow></msub><mi mathvariant="normal">/</mi><mn>2</mn><mo stretchy="false">)</mo><mo stretchy="false">!</mo></mrow></mfrac></mrow><annotation encoding="application/x-tex">P(G| \omega, g) = \frac{\prod_{r,s}\omega_{rs}^{m_{rs}/2}exp(\frac{-1}{2} n_r n_s \omega_{rs}) }{\prod_{i &lt; j}A_{ij}! \prod_i 2^{A_{ii}/2}(A_{ii}/2)!} </annotation></semantics></math></span><span class="katex-html" aria-hidden="true"><span class="base"><span class="strut" style="height: 1em; vertical-align: -0.25em;"></span><span class="mord mathnormal" style="margin-right: 0.13889em;">P</span><span class="mopen">(</span><span class="mord mathnormal">G</span><span class="mord">∣</span><span class="mord mathnormal" style="margin-right: 0.03588em;">ω</span><span class="mpunct">,</span><span class="mspace" style="margin-right: 0.166667em;"></span><span class="mord mathnormal" style="margin-right: 0.03588em;">g</span><span class="mclose">)</span><span class="mspace" style="margin-right: 0.277778em;"></span><span class="mrel">=</span><span class="mspace" style="margin-right: 0.277778em;"></span></span><span class="base"><span class="strut" style="height: 3.01044em; vertical-align: -1.13982em;"></span><span class="mord"><span class="mopen nulldelimiter"></span><span class="mfrac"><span class="vlist-t vlist-t2"><span class="vlist-r"><span class="vlist" style="height: 1.87062em;"><span class="" style="top: -2.3408em;"><span class="pstrut" style="height: 3.0448em;"></span><span class="mord"><span class="mop"><span class="mop op-symbol small-op" style="position: relative; top: -5e-06em;">∏</span><span class="msupsub"><span class="vlist-t vlist-t2"><span class="vlist-r"><span class="vlist" style="height: 0.161954em;"><span class="" style="top: -2.40029em; margin-left: 0em; margin-right: 0.05em;"><span class="pstrut" style="height: 2.7em;"></span><span class="sizing reset-size6 size3 mtight"><span class="mord mtight"><span class="mord mathnormal mtight">i</span><span class="mrel mtight">&lt;</span><span class="mord mathnormal mtight" style="margin-right: 0.05724em;">j</span></span></span></span></span><span class="vlist-s">​</span></span><span class="vlist-r"><span class="vlist" style="height: 0.435818em;"><span class=""></span></span></span></span></span></span><span class="mspace" style="margin-right: 0.166667em;"></span><span class="mord"><span class="mord mathnormal">A</span><span class="msupsub"><span class="vlist-t vlist-t2"><span class="vlist-r"><span class="vlist" style="height: 0.311664em;"><span class="" style="top: -2.55em; margin-left: 0em; margin-right: 0.05em;"><span class="pstrut" style="height: 2.7em;"></span><span class="sizing reset-size6 size3 mtight"><span class="mord mtight"><span class="mord mathnormal mtight" style="margin-right: 0.05724em;">ij</span></span></span></span></span><span class="vlist-s">​</span></span><span class="vlist-r"><span class="vlist" style="height: 0.286108em;"><span class=""></span></span></span></span></span></span><span class="mclose">!</span><span class="mspace" style="margin-right: 0.166667em;"></span><span class="mop"><span class="mop op-symbol small-op" style="position: relative; top: -5e-06em;">∏</span><span class="msupsub"><span class="vlist-t vlist-t2"><span class="vlist-r"><span class="vlist" style="height: 0.161954em;"><span class="" style="top: -2.40029em; margin-left: 0em; margin-right: 0.05em;"><span class="pstrut" style="height: 2.7em;"></span><span class="sizing reset-size6 size3 mtight"><span class="mord mathnormal mtight">i</span></span></span></span><span class="vlist-s">​</span></span><span class="vlist-r"><span class="vlist" style="height: 0.29971em;"><span class=""></span></span></span></span></span></span><span class="mspace" style="margin-right: 0.166667em;"></span><span class="mord"><span class="mord">2</span><span class="msupsub"><span class="vlist-t"><span class="vlist-r"><span class="vlist" style="height: 0.814em;"><span class="" style="top: -2.989em; margin-right: 0.05em;"><span class="pstrut" style="height: 2.7em;"></span><span class="sizing reset-size6 size3 mtight"><span class="mord mtight"><span class="mord mtight"><span class="mord mathnormal mtight">A</span><span class="msupsub"><span class="vlist-t vlist-t2"><span class="vlist-r"><span class="vlist" style="height: 0.328086em;"><span class="" style="top: -2.357em; margin-left: 0em; margin-right: 0.0714286em;"><span class="pstrut" style="height: 2.5em;"></span><span class="sizing reset-size3 size1 mtight"><span class="mord mtight"><span class="mord mathnormal mtight">ii</span></span></span></span></span><span class="vlist-s">​</span></span><span class="vlist-r"><span class="vlist" style="height: 0.143em;"><span class=""></span></span></span></span></span></span><span class="mord mtight">/2</span></span></span></span></span></span></span></span></span><span class="mopen">(</span><span class="mord"><span class="mord mathnormal">A</span><span class="msupsub"><span class="vlist-t vlist-t2"><span class="vlist-r"><span class="vlist" style="height: 0.311664em;"><span class="" style="top: -2.55em; margin-left: 0em; margin-right: 0.05em;"><span class="pstrut" style="height: 2.7em;"></span><span class="sizing reset-size6 size3 mtight"><span class="mord mtight"><span class="mord mathnormal mtight">ii</span></span></span></span></span><span class="vlist-s">​</span></span><span class="vlist-r"><span class="vlist" style="height: 0.15em;"><span class=""></span></span></span></span></span></span><span class="mord">/2</span><span class="mclose">)!</span></span></span><span class="" style="top: -3.2748em;"><span class="pstrut" style="height: 3.0448em;"></span><span class="frac-line" style="border-bottom-width: 0.04em;"></span></span><span class="" style="top: -3.87062em;"><span class="pstrut" style="height: 3.0448em;"></span><span class="mord"><span class="mop"><span class="mop op-symbol small-op" style="position: relative; top: -5e-06em;">∏</span><span class="msupsub"><span class="vlist-t vlist-t2"><span class="vlist-r"><span class="vlist" style="height: 0.001682em;"><span class="" style="top: -2.40029em; margin-left: 0em; margin-right: 0.05em;"><span class="pstrut" style="height: 2.7em;"></span><span class="sizing reset-size6 size3 mtight"><span class="mord mtight"><span class="mord mathnormal mtight" style="margin-right: 0.02778em;">r</span><span class="mpunct mtight">,</span><span class="mord mathnormal mtight">s</span></span></span></span></span><span class="vlist-s">​</span></span><span class="vlist-r"><span class="vlist" style="height: 0.435818em;"><span class=""></span></span></span></span></span></span><span class="mspace" style="margin-right: 0.166667em;"></span><span class="mord"><span class="mord mathnormal" style="margin-right: 0.03588em;">ω</span><span class="msupsub"><span class="vlist-t vlist-t2"><span class="vlist-r"><span class="vlist" style="height: 1.0448em;"><span class="" style="top: -2.58341em; margin-left: -0.03588em; margin-right: 0.05em;"><span class="pstrut" style="height: 2.7em;"></span><span class="sizing reset-size6 size3 mtight"><span class="mord mtight"><span class="mord mathnormal mtight">rs</span></span></span></span><span class="" style="top: -3.2198em; margin-right: 0.05em;"><span class="pstrut" style="height: 2.7em;"></span><span class="sizing reset-size6 size3 mtight"><span class="mord mtight"><span class="mord mtight"><span class="mord mathnormal mtight">m</span><span class="msupsub"><span class="vlist-t vlist-t2"><span class="vlist-r"><span class="vlist" style="height: 0.164543em;"><span class="" style="top: -2.357em; margin-left: 0em; margin-right: 0.0714286em;"><span class="pstrut" style="height: 2.5em;"></span><span class="sizing reset-size3 size1 mtight"><span class="mord mtight"><span class="mord mathnormal mtight">rs</span></span></span></span></span><span class="vlist-s">​</span></span><span class="vlist-r"><span class="vlist" style="height: 0.143em;"><span class=""></span></span></span></span></span></span><span class="mord mtight">/2</span></span></span></span></span><span class="vlist-s">​</span></span><span class="vlist-r"><span class="vlist" style="height: 0.116592em;"><span class=""></span></span></span></span></span></span><span class="mord mathnormal">e</span><span class="mord mathnormal">x</span><span class="mord mathnormal">p</span><span class="mopen">(</span><span class="mord"><span class="mopen nulldelimiter"></span><span class="mfrac"><span class="vlist-t vlist-t2"><span class="vlist-r"><span class="vlist" style="height: 0.845108em;"><span class="" style="top: -2.655em;"><span class="pstrut" style="height: 3em;"></span><span class="sizing reset-size6 size3 mtight"><span class="mord mtight"><span class="mord mtight">2</span></span></span></span><span class="" style="top: -3.23em;"><span class="pstrut" style="height: 3em;"></span><span class="frac-line" style="border-bottom-width: 0.04em;"></span></span><span class="" style="top: -3.394em;"><span class="pstrut" style="height: 3em;"></span><span class="sizing reset-size6 size3 mtight"><span class="mord mtight"><span class="mord mtight">−</span><span class="mord mtight">1</span></span></span></span></span><span class="vlist-s">​</span></span><span class="vlist-r"><span class="vlist" style="height: 0.345em;"><span class=""></span></span></span></span></span><span class="mclose nulldelimiter"></span></span><span class="mord"><span class="mord mathnormal">n</span><span class="msupsub"><span class="vlist-t vlist-t2"><span class="vlist-r"><span class="vlist" style="height: 0.151392em;"><span class="" style="top: -2.55em; margin-left: 0em; margin-right: 0.05em;"><span class="pstrut" style="height: 2.7em;"></span><span class="sizing reset-size6 size3 mtight"><span class="mord mathnormal mtight" style="margin-right: 0.02778em;">r</span></span></span></span><span class="vlist-s">​</span></span><span class="vlist-r"><span class="vlist" style="height: 0.15em;"><span class=""></span></span></span></span></span></span><span class="mord"><span class="mord mathnormal">n</span><span class="msupsub"><span class="vlist-t vlist-t2"><span class="vlist-r"><span class="vlist" style="height: 0.151392em;"><span class="" style="top: -2.55em; margin-left: 0em; margin-right: 0.05em;"><span class="pstrut" style="height: 2.7em;"></span><span class="sizing reset-size6 size3 mtight"><span class="mord mathnormal mtight">s</span></span></span></span><span class="vlist-s">​</span></span><span class="vlist-r"><span class="vlist" style="height: 0.15em;"><span class=""></span></span></span></span></span></span><span class="mord"><span class="mord mathnormal" style="margin-right: 0.03588em;">ω</span><span class="msupsub"><span class="vlist-t vlist-t2"><span class="vlist-r"><span class="vlist" style="height: 0.151392em;"><span class="" style="top: -2.55em; margin-left: -0.03588em; margin-right: 0.05em;"><span class="pstrut" style="height: 2.7em;"></span><span class="sizing reset-size6 size3 mtight"><span class="mord mtight"><span class="mord mathnormal mtight">rs</span></span></span></span></span><span class="vlist-s">​</span></span><span class="vlist-r"><span class="vlist" style="height: 0.15em;"><span class=""></span></span></span></span></span></span><span class="mclose">)</span></span></span></span><span class="vlist-s">​</span></span><span class="vlist-r"><span class="vlist" style="height: 1.13982em;"><span class=""></span></span></span></span></span><span class="mclose nulldelimiter"></span></span></span></span></span></span></span><br>
where <span class="katex--inline"><span class="katex"><span class="katex-mathml"><math xmlns="http://www.w3.org/1998/Math/MathML"><semantics><mrow><msub><mi>n</mi><mi>r</mi></msub></mrow><annotation encoding="application/x-tex">n_r</annotation></semantics></math></span><span class="katex-html" aria-hidden="true"><span class="base"><span class="strut" style="height: 0.58056em; vertical-align: -0.15em;"></span><span class="mord"><span class="mord mathnormal">n</span><span class="msupsub"><span class="vlist-t vlist-t2"><span class="vlist-r"><span class="vlist" style="height: 0.151392em;"><span class="" style="top: -2.55em; margin-left: 0em; margin-right: 0.05em;"><span class="pstrut" style="height: 2.7em;"></span><span class="sizing reset-size6 size3 mtight"><span class="mord mathnormal mtight" style="margin-right: 0.02778em;">r</span></span></span></span><span class="vlist-s">​</span></span><span class="vlist-r"><span class="vlist" style="height: 0.15em;"><span class=""></span></span></span></span></span></span></span></span></span></span> is the number of nodes in community <span class="katex--inline"><span class="katex"><span class="katex-mathml"><math xmlns="http://www.w3.org/1998/Math/MathML"><semantics><mrow><msub><mi>C</mi><mi>r</mi></msub></mrow><annotation encoding="application/x-tex">C_r</annotation></semantics></math></span><span class="katex-html" aria-hidden="true"><span class="base"><span class="strut" style="height: 0.83333em; vertical-align: -0.15em;"></span><span class="mord"><span class="mord mathnormal" style="margin-right: 0.07153em;">C</span><span class="msupsub"><span class="vlist-t vlist-t2"><span class="vlist-r"><span class="vlist" style="height: 0.151392em;"><span class="" style="top: -2.55em; margin-left: -0.07153em; margin-right: 0.05em;"><span class="pstrut" style="height: 2.7em;"></span><span class="sizing reset-size6 size3 mtight"><span class="mord mathnormal mtight" style="margin-right: 0.02778em;">r</span></span></span></span><span class="vlist-s">​</span></span><span class="vlist-r"><span class="vlist" style="height: 0.15em;"><span class=""></span></span></span></span></span></span></span></span></span></span> and <span class="katex--inline"><span class="katex"><span class="katex-mathml"><math xmlns="http://www.w3.org/1998/Math/MathML"><semantics><mrow><msub><mi>m</mi><mrow><mi>r</mi><mi>s</mi></mrow></msub><mo>=</mo><msub><mo>∑</mo><mrow><mi>i</mi><mi>j</mi></mrow></msub><msub><mi>A</mi><mrow><mi>i</mi><mi>j</mi></mrow></msub><msub><mi>δ</mi><mrow><msub><mi>g</mi><mi>i</mi></msub><mo separator="true">,</mo><mi>r</mi></mrow></msub><msub><mi>δ</mi><mrow><msub><mi>g</mi><mi>j</mi></msub><mo separator="true">,</mo><mi>s</mi></mrow></msub></mrow><annotation encoding="application/x-tex">m_{rs} = \sum_{ij}A_{ij}\delta_{g_i,r} \delta_{g_j,s}</annotation></semantics></math></span><span class="katex-html" aria-hidden="true"><span class="base"><span class="strut" style="height: 0.58056em; vertical-align: -0.15em;"></span><span class="mord"><span class="mord mathnormal">m</span><span class="msupsub"><span class="vlist-t vlist-t2"><span class="vlist-r"><span class="vlist" style="height: 0.151392em;"><span class="" style="top: -2.55em; margin-left: 0em; margin-right: 0.05em;"><span class="pstrut" style="height: 2.7em;"></span><span class="sizing reset-size6 size3 mtight"><span class="mord mtight"><span class="mord mathnormal mtight">rs</span></span></span></span></span><span class="vlist-s">​</span></span><span class="vlist-r"><span class="vlist" style="height: 0.15em;"><span class=""></span></span></span></span></span></span><span class="mspace" style="margin-right: 0.277778em;"></span><span class="mrel">=</span><span class="mspace" style="margin-right: 0.277778em;"></span></span><span class="base"><span class="strut" style="height: 1.18582em; vertical-align: -0.435818em;"></span><span class="mop"><span class="mop op-symbol small-op" style="position: relative; top: -5e-06em;">∑</span><span class="msupsub"><span class="vlist-t vlist-t2"><span class="vlist-r"><span class="vlist" style="height: 0.161954em;"><span class="" style="top: -2.40029em; margin-left: 0em; margin-right: 0.05em;"><span class="pstrut" style="height: 2.7em;"></span><span class="sizing reset-size6 size3 mtight"><span class="mord mtight"><span class="mord mathnormal mtight" style="margin-right: 0.05724em;">ij</span></span></span></span></span><span class="vlist-s">​</span></span><span class="vlist-r"><span class="vlist" style="height: 0.435818em;"><span class=""></span></span></span></span></span></span><span class="mspace" style="margin-right: 0.166667em;"></span><span class="mord"><span class="mord mathnormal">A</span><span class="msupsub"><span class="vlist-t vlist-t2"><span class="vlist-r"><span class="vlist" style="height: 0.311664em;"><span class="" style="top: -2.55em; margin-left: 0em; margin-right: 0.05em;"><span class="pstrut" style="height: 2.7em;"></span><span class="sizing reset-size6 size3 mtight"><span class="mord mtight"><span class="mord mathnormal mtight" style="margin-right: 0.05724em;">ij</span></span></span></span></span><span class="vlist-s">​</span></span><span class="vlist-r"><span class="vlist" style="height: 0.286108em;"><span class=""></span></span></span></span></span></span><span class="mord"><span class="mord mathnormal" style="margin-right: 0.03785em;">δ</span><span class="msupsub"><span class="vlist-t vlist-t2"><span class="vlist-r"><span class="vlist" style="height: 0.151392em;"><span class="" style="top: -2.55em; margin-left: -0.03785em; margin-right: 0.05em;"><span class="pstrut" style="height: 2.7em;"></span><span class="sizing reset-size6 size3 mtight"><span class="mord mtight"><span class="mord mtight"><span class="mord mathnormal mtight" style="margin-right: 0.03588em;">g</span><span class="msupsub"><span class="vlist-t vlist-t2"><span class="vlist-r"><span class="vlist" style="height: 0.328086em;"><span class="" style="top: -2.357em; margin-left: -0.03588em; margin-right: 0.0714286em;"><span class="pstrut" style="height: 2.5em;"></span><span class="sizing reset-size3 size1 mtight"><span class="mord mathnormal mtight">i</span></span></span></span><span class="vlist-s">​</span></span><span class="vlist-r"><span class="vlist" style="height: 0.143em;"><span class=""></span></span></span></span></span></span><span class="mpunct mtight">,</span><span class="mord mathnormal mtight" style="margin-right: 0.02778em;">r</span></span></span></span></span><span class="vlist-s">​</span></span><span class="vlist-r"><span class="vlist" style="height: 0.286108em;"><span class=""></span></span></span></span></span></span><span class="mord"><span class="mord mathnormal" style="margin-right: 0.03785em;">δ</span><span class="msupsub"><span class="vlist-t vlist-t2"><span class="vlist-r"><span class="vlist" style="height: 0.151392em;"><span class="" style="top: -2.55em; margin-left: -0.03785em; margin-right: 0.05em;"><span class="pstrut" style="height: 2.7em;"></span><span class="sizing reset-size6 size3 mtight"><span class="mord mtight"><span class="mord mtight"><span class="mord mathnormal mtight" style="margin-right: 0.03588em;">g</span><span class="msupsub"><span class="vlist-t vlist-t2"><span class="vlist-r"><span class="vlist" style="height: 0.328086em;"><span class="" style="top: -2.357em; margin-left: -0.03588em; margin-right: 0.0714286em;"><span class="pstrut" style="height: 2.5em;"></span><span class="sizing reset-size3 size1 mtight"><span class="mord mathnormal mtight" style="margin-right: 0.05724em;">j</span></span></span></span><span class="vlist-s">​</span></span><span class="vlist-r"><span class="vlist" style="height: 0.281886em;"><span class=""></span></span></span></span></span></span><span class="mpunct mtight">,</span><span class="mord mathnormal mtight">s</span></span></span></span></span><span class="vlist-s">​</span></span><span class="vlist-r"><span class="vlist" style="height: 0.34732em;"><span class=""></span></span></span></span></span></span></span></span></span></span>.<br>
Taking the logarithm, differentiating with respect to <span class="katex--inline"><span class="katex"><span class="katex-mathml"><math xmlns="http://www.w3.org/1998/Math/MathML"><semantics><mrow><msub><mi>ω</mi><mrow><mi>r</mi><mi>s</mi></mrow></msub></mrow><annotation encoding="application/x-tex">\omega_{rs}</annotation></semantics></math></span><span class="katex-html" aria-hidden="true"><span class="base"><span class="strut" style="height: 0.58056em; vertical-align: -0.15em;"></span><span class="mord"><span class="mord mathnormal" style="margin-right: 0.03588em;">ω</span><span class="msupsub"><span class="vlist-t vlist-t2"><span class="vlist-r"><span class="vlist" style="height: 0.151392em;"><span class="" style="top: -2.55em; margin-left: -0.03588em; margin-right: 0.05em;"><span class="pstrut" style="height: 2.7em;"></span><span class="sizing reset-size6 size3 mtight"><span class="mord mtight"><span class="mord mathnormal mtight">rs</span></span></span></span></span><span class="vlist-s">​</span></span><span class="vlist-r"><span class="vlist" style="height: 0.15em;"><span class=""></span></span></span></span></span></span></span></span></span></span> reveals that the maximum is <span class="katex--inline">KaTeX parse error: Undefined control sequence: \Hat at position 1: \̲H̲a̲t̲{\omega_{rs}} =…</span>. \cite{karrer_newman_2011}</p>
<p>\begin{figure}[h]%[tbhp]</p>
<pre><code>\includegraphics[width = 0.3\linewidth]{Assortative SBM 3 comms.png}
\label{3assortative}
</code></pre>
<p>\hfill<br>
\includegraphics[width = 0.3\linewidth]{Dissassortative SBM 2 comms.png}<br>
\label{2dissassortative}<br>
\hfill<br>
\includegraphics[width = 0.3\linewidth]{Core periphery structure SBM.png}<br>
\caption{On the left, the adjacency matrix of a SBM of 300 nodes with 3 assortative equal sized communities. In the middle, a SBM of 300 nodes with 2 disassortative communities. On the right, an SBM with core-periphery structure - the core contains 200 nodes and the periphery has 100 nodes. }<br>
\label{coreperiphery}<br>
\end{figure}</p>
<p>\subsection{Degree-Corrected Stochastic Block Models (DCSBM)}<br>
Karrer and Newman  \cite{karrer_newman_2011} presented the valid criticism that the SBM may not  be an accurate depiction of real-world networks since it is built upon the assumption that the distribution of links of a node in one community to the same/another community is given by a fixed Poisson distribution. This is quite coarse as it determines the degree of a node based on which community it belongs to, and which community alone.<br>
To counteract this issue, \cite{karrer_newman_2011} designed the degree-corrected stochastic block model where the expected value of an entry of the adjacency matrix <span class="katex--inline"><span class="katex"><span class="katex-mathml"><math xmlns="http://www.w3.org/1998/Math/MathML"><semantics><mrow><msub><mi>A</mi><mrow><mi>i</mi><mi>j</mi></mrow></msub></mrow><annotation encoding="application/x-tex">A_{ij}</annotation></semantics></math></span><span class="katex-html" aria-hidden="true"><span class="base"><span class="strut" style="height: 0.969438em; vertical-align: -0.286108em;"></span><span class="mord"><span class="mord mathnormal">A</span><span class="msupsub"><span class="vlist-t vlist-t2"><span class="vlist-r"><span class="vlist" style="height: 0.311664em;"><span class="" style="top: -2.55em; margin-left: 0em; margin-right: 0.05em;"><span class="pstrut" style="height: 2.7em;"></span><span class="sizing reset-size6 size3 mtight"><span class="mord mtight"><span class="mord mathnormal mtight" style="margin-right: 0.05724em;">ij</span></span></span></span></span><span class="vlist-s">​</span></span><span class="vlist-r"><span class="vlist" style="height: 0.286108em;"><span class=""></span></span></span></span></span></span></span></span></span></span> is given by:</p>
<p><span class="katex--display"><span class="katex-display"><span class="katex"><span class="katex-mathml"><math xmlns="http://www.w3.org/1998/Math/MathML" display="block"><semantics><mrow><mi mathvariant="double-struck">E</mi><mo stretchy="false">(</mo><msub><mi>A</mi><mrow><mi>i</mi><mi>j</mi></mrow></msub><mo stretchy="false">)</mo><mo>=</mo><msub><mi>θ</mi><mi>i</mi></msub><msub><mi>θ</mi><mi>j</mi></msub><msub><mi>ω</mi><mrow><msub><mi>g</mi><mi>i</mi></msub><mo separator="true">,</mo><msub><mi>g</mi><mi>j</mi></msub></mrow></msub><mo separator="true">,</mo></mrow><annotation encoding="application/x-tex">\mathbb{E}(A_{ij}) = \theta_i \theta_j \omega_{g_i, g_j} ,</annotation></semantics></math></span><span class="katex-html" aria-hidden="true"><span class="base"><span class="strut" style="height: 1.03611em; vertical-align: -0.286108em;"></span><span class="mord mathbb">E</span><span class="mopen">(</span><span class="mord"><span class="mord mathnormal">A</span><span class="msupsub"><span class="vlist-t vlist-t2"><span class="vlist-r"><span class="vlist" style="height: 0.311664em;"><span class="" style="top: -2.55em; margin-left: 0em; margin-right: 0.05em;"><span class="pstrut" style="height: 2.7em;"></span><span class="sizing reset-size6 size3 mtight"><span class="mord mtight"><span class="mord mathnormal mtight" style="margin-right: 0.05724em;">ij</span></span></span></span></span><span class="vlist-s">​</span></span><span class="vlist-r"><span class="vlist" style="height: 0.286108em;"><span class=""></span></span></span></span></span></span><span class="mclose">)</span><span class="mspace" style="margin-right: 0.277778em;"></span><span class="mrel">=</span><span class="mspace" style="margin-right: 0.277778em;"></span></span><span class="base"><span class="strut" style="height: 1.04176em; vertical-align: -0.34732em;"></span><span class="mord"><span class="mord mathnormal" style="margin-right: 0.02778em;">θ</span><span class="msupsub"><span class="vlist-t vlist-t2"><span class="vlist-r"><span class="vlist" style="height: 0.311664em;"><span class="" style="top: -2.55em; margin-left: -0.02778em; margin-right: 0.05em;"><span class="pstrut" style="height: 2.7em;"></span><span class="sizing reset-size6 size3 mtight"><span class="mord mathnormal mtight">i</span></span></span></span><span class="vlist-s">​</span></span><span class="vlist-r"><span class="vlist" style="height: 0.15em;"><span class=""></span></span></span></span></span></span><span class="mord"><span class="mord mathnormal" style="margin-right: 0.02778em;">θ</span><span class="msupsub"><span class="vlist-t vlist-t2"><span class="vlist-r"><span class="vlist" style="height: 0.311664em;"><span class="" style="top: -2.55em; margin-left: -0.02778em; margin-right: 0.05em;"><span class="pstrut" style="height: 2.7em;"></span><span class="sizing reset-size6 size3 mtight"><span class="mord mathnormal mtight" style="margin-right: 0.05724em;">j</span></span></span></span><span class="vlist-s">​</span></span><span class="vlist-r"><span class="vlist" style="height: 0.286108em;"><span class=""></span></span></span></span></span></span><span class="mord"><span class="mord mathnormal" style="margin-right: 0.03588em;">ω</span><span class="msupsub"><span class="vlist-t vlist-t2"><span class="vlist-r"><span class="vlist" style="height: 0.151392em;"><span class="" style="top: -2.55em; margin-left: -0.03588em; margin-right: 0.05em;"><span class="pstrut" style="height: 2.7em;"></span><span class="sizing reset-size6 size3 mtight"><span class="mord mtight"><span class="mord mtight"><span class="mord mathnormal mtight" style="margin-right: 0.03588em;">g</span><span class="msupsub"><span class="vlist-t vlist-t2"><span class="vlist-r"><span class="vlist" style="height: 0.328086em;"><span class="" style="top: -2.357em; margin-left: -0.03588em; margin-right: 0.0714286em;"><span class="pstrut" style="height: 2.5em;"></span><span class="sizing reset-size3 size1 mtight"><span class="mord mathnormal mtight">i</span></span></span></span><span class="vlist-s">​</span></span><span class="vlist-r"><span class="vlist" style="height: 0.143em;"><span class=""></span></span></span></span></span></span><span class="mpunct mtight">,</span><span class="mord mtight"><span class="mord mathnormal mtight" style="margin-right: 0.03588em;">g</span><span class="msupsub"><span class="vlist-t vlist-t2"><span class="vlist-r"><span class="vlist" style="height: 0.328086em;"><span class="" style="top: -2.357em; margin-left: -0.03588em; margin-right: 0.0714286em;"><span class="pstrut" style="height: 2.5em;"></span><span class="sizing reset-size3 size1 mtight"><span class="mord mathnormal mtight" style="margin-right: 0.05724em;">j</span></span></span></span><span class="vlist-s">​</span></span><span class="vlist-r"><span class="vlist" style="height: 0.281886em;"><span class=""></span></span></span></span></span></span></span></span></span></span><span class="vlist-s">​</span></span><span class="vlist-r"><span class="vlist" style="height: 0.34732em;"><span class=""></span></span></span></span></span></span><span class="mpunct">,</span></span></span></span></span></span><br>
where parameters <span class="katex--inline"><span class="katex"><span class="katex-mathml"><math xmlns="http://www.w3.org/1998/Math/MathML"><semantics><mrow><msub><mi>θ</mi><mi>i</mi></msub></mrow><annotation encoding="application/x-tex">\theta_i</annotation></semantics></math></span><span class="katex-html" aria-hidden="true"><span class="base"><span class="strut" style="height: 0.84444em; vertical-align: -0.15em;"></span><span class="mord"><span class="mord mathnormal" style="margin-right: 0.02778em;">θ</span><span class="msupsub"><span class="vlist-t vlist-t2"><span class="vlist-r"><span class="vlist" style="height: 0.311664em;"><span class="" style="top: -2.55em; margin-left: -0.02778em; margin-right: 0.05em;"><span class="pstrut" style="height: 2.7em;"></span><span class="sizing reset-size6 size3 mtight"><span class="mord mathnormal mtight">i</span></span></span></span><span class="vlist-s">​</span></span><span class="vlist-r"><span class="vlist" style="height: 0.15em;"><span class=""></span></span></span></span></span></span></span></span></span></span> and <span class="katex--inline"><span class="katex"><span class="katex-mathml"><math xmlns="http://www.w3.org/1998/Math/MathML"><semantics><mrow><msub><mi>θ</mi><mi>j</mi></msub></mrow><annotation encoding="application/x-tex">\theta_j</annotation></semantics></math></span><span class="katex-html" aria-hidden="true"><span class="base"><span class="strut" style="height: 0.980548em; vertical-align: -0.286108em;"></span><span class="mord"><span class="mord mathnormal" style="margin-right: 0.02778em;">θ</span><span class="msupsub"><span class="vlist-t vlist-t2"><span class="vlist-r"><span class="vlist" style="height: 0.311664em;"><span class="" style="top: -2.55em; margin-left: -0.02778em; margin-right: 0.05em;"><span class="pstrut" style="height: 2.7em;"></span><span class="sizing reset-size6 size3 mtight"><span class="mord mathnormal mtight" style="margin-right: 0.05724em;">j</span></span></span></span><span class="vlist-s">​</span></span><span class="vlist-r"><span class="vlist" style="height: 0.286108em;"><span class=""></span></span></span></span></span></span></span></span></span></span> have been introduced so that the model takes into account the position of the node itself rather than simply what community it belongs to.</p>
<p>\subsection{Spectral Clustering and K-means}<br>
A popular example of a community detection algorithm which searches for clusters of nodes using the spectral properties of a matrix related to the Network.<br>
Now using what is called a similarity matrix such as the Network’s adjacency matrix <span class="katex--inline"><span class="katex"><span class="katex-mathml"><math xmlns="http://www.w3.org/1998/Math/MathML"><semantics><mrow><mi>A</mi></mrow><annotation encoding="application/x-tex">A</annotation></semantics></math></span><span class="katex-html" aria-hidden="true"><span class="base"><span class="strut" style="height: 0.68333em; vertical-align: 0em;"></span><span class="mord mathnormal">A</span></span></span></span></span>, we calculate the Laplacian <span class="katex--inline"><span class="katex"><span class="katex-mathml"><math xmlns="http://www.w3.org/1998/Math/MathML"><semantics><mrow><mi>L</mi><mo>=</mo><mi>D</mi><mo>−</mo><mi>A</mi></mrow><annotation encoding="application/x-tex">L = D-A</annotation></semantics></math></span><span class="katex-html" aria-hidden="true"><span class="base"><span class="strut" style="height: 0.68333em; vertical-align: 0em;"></span><span class="mord mathnormal">L</span><span class="mspace" style="margin-right: 0.277778em;"></span><span class="mrel">=</span><span class="mspace" style="margin-right: 0.277778em;"></span></span><span class="base"><span class="strut" style="height: 0.76666em; vertical-align: -0.08333em;"></span><span class="mord mathnormal" style="margin-right: 0.02778em;">D</span><span class="mspace" style="margin-right: 0.222222em;"></span><span class="mbin">−</span><span class="mspace" style="margin-right: 0.222222em;"></span></span><span class="base"><span class="strut" style="height: 0.68333em; vertical-align: 0em;"></span><span class="mord mathnormal">A</span></span></span></span></span> and a selection of the <span class="katex--inline"><span class="katex"><span class="katex-mathml"><math xmlns="http://www.w3.org/1998/Math/MathML"><semantics><mrow><mi>K</mi></mrow><annotation encoding="application/x-tex">K</annotation></semantics></math></span><span class="katex-html" aria-hidden="true"><span class="base"><span class="strut" style="height: 0.68333em; vertical-align: 0em;"></span><span class="mord mathnormal" style="margin-right: 0.07153em;">K</span></span></span></span></span> smallest eigenvalues are then chosen where <span class="katex--inline"><span class="katex"><span class="katex-mathml"><math xmlns="http://www.w3.org/1998/Math/MathML"><semantics><mrow><mi>K</mi></mrow><annotation encoding="application/x-tex">K</annotation></semantics></math></span><span class="katex-html" aria-hidden="true"><span class="base"><span class="strut" style="height: 0.68333em; vertical-align: 0em;"></span><span class="mord mathnormal" style="margin-right: 0.07153em;">K</span></span></span></span></span> must be prespecified. The eigenvectors are placed in columns of a <span class="katex--inline"><span class="katex"><span class="katex-mathml"><math xmlns="http://www.w3.org/1998/Math/MathML"><semantics><mrow><mi>n</mi><mo>×</mo><mi>K</mi></mrow><annotation encoding="application/x-tex">n \times K</annotation></semantics></math></span><span class="katex-html" aria-hidden="true"><span class="base"><span class="strut" style="height: 0.66666em; vertical-align: -0.08333em;"></span><span class="mord mathnormal">n</span><span class="mspace" style="margin-right: 0.222222em;"></span><span class="mbin">×</span><span class="mspace" style="margin-right: 0.222222em;"></span></span><span class="base"><span class="strut" style="height: 0.68333em; vertical-align: 0em;"></span><span class="mord mathnormal" style="margin-right: 0.07153em;">K</span></span></span></span></span> matrix. Each row can then be considered as the ‘coordinates’ of a node in a <span class="katex--inline"><span class="katex"><span class="katex-mathml"><math xmlns="http://www.w3.org/1998/Math/MathML"><semantics><mrow><mi>K</mi></mrow><annotation encoding="application/x-tex">K</annotation></semantics></math></span><span class="katex-html" aria-hidden="true"><span class="base"><span class="strut" style="height: 0.68333em; vertical-align: 0em;"></span><span class="mord mathnormal" style="margin-right: 0.07153em;">K</span></span></span></span></span> dimensional space. If these eigenvectors are well correlated with the community label assignments of each node (as they are well known to be with the Laplacian), then the nodes should be positioned in such a way that they are close in Euclidean distance to other nodes in their true groupings.<br>
\newline<br>
The K-means algorithm now works as follows \cite{MacQueen1967}:<br>
\begin{enumerate}<br>
\item initial centroids are chosen and each node is assigned to a provisional cluster depending on which centroid they are closest too in Euclidean distance.<br>
\item The geometric mean of each provisional cluster is then determined and this mean is titled the new centroid of this cluster.<br>
\item The nodes are reassigned to the new provisional clusters corresponding to the new centroids they are closest too.<br>
\item Steps 2. and 3. are repeated until no node moves from a cluster. These clusterings are then deemed the communities/blocks.<br>
\end{enumerate}</p>
<p>There are many ways to choose the initial centroids. A well accepted approach is to simply choose <span class="katex--inline"><span class="katex"><span class="katex-mathml"><math xmlns="http://www.w3.org/1998/Math/MathML"><semantics><mrow><mi>K</mi></mrow><annotation encoding="application/x-tex">K</annotation></semantics></math></span><span class="katex-html" aria-hidden="true"><span class="base"><span class="strut" style="height: 0.68333em; vertical-align: 0em;"></span><span class="mord mathnormal" style="margin-right: 0.07153em;">K</span></span></span></span></span> random initial centroids. Running the K-means algorithm multiple times with random initial centroids, we could take the the clustering obtain from the initial centroids where the nodes changed cluster the least. This is the Forgy Method \cite{Forgy65} and is the procedure that we use in our examples in sections 2 and 3. (It is the default method used by scikit-learn’s \textit{k-means}.)<br>
\newline<br>
Finally, we remark that the Laplacian of a network is not the only matrix with eigenvectors well correlated with the block partitioning of the nodes! In practice, it is often better to use the normalized Laplacian or better still, a matrix which will be discussed in section <span class="katex--inline"><span class="katex"><span class="katex-mathml"><math xmlns="http://www.w3.org/1998/Math/MathML"><semantics><mrow><mn>2.</mn><mi>B</mi></mrow><annotation encoding="application/x-tex">2.B</annotation></semantics></math></span><span class="katex-html" aria-hidden="true"><span class="base"><span class="strut" style="height: 0.68333em; vertical-align: 0em;"></span><span class="mord">2.</span><span class="mord mathnormal" style="margin-right: 0.05017em;">B</span></span></span></span></span>.</p>
<p>\section{How many communities?}<br>
There is a vast amount of literature on the detection of the assortative and disassortative communities described in the first section. For instance, one can use the maximum likelihood function described in section 1.B to try and find the best fit of a SBM to an empirical network; similarly, we could also attempt to fit a DCSBM as described in 1.C.<br>
Another incredibly popular approach is modularity maximization. To determine the number of assortative communities, we can maximize the Newman modularity \cite{newman_2006} or with a bit more care, we could maximize the adapted modularity described in \cite{Reichardt} which will also detect disassortative communities.<br>
\newline<br>
As another example, we could use a spectral clustering technique as described in section 1.D applied to a similarity matrix (typically the adjacency matrix <span class="katex--inline"><span class="katex"><span class="katex-mathml"><math xmlns="http://www.w3.org/1998/Math/MathML"><semantics><mrow><mi>A</mi></mrow><annotation encoding="application/x-tex">A</annotation></semantics></math></span><span class="katex-html" aria-hidden="true"><span class="base"><span class="strut" style="height: 0.68333em; vertical-align: 0em;"></span><span class="mord mathnormal">A</span></span></span></span></span>) of the network.<br>
\newline<br>
Solving these problems is a lot easier said than done. Consequently creation of algorithms designed to maximize objective functions such as the maximum likelihood or modularity function is a constantly growing field. Many of these algorithms, as ingenious as they are, often require the number of communities which we shall denote by <span class="katex--inline"><span class="katex"><span class="katex-mathml"><math xmlns="http://www.w3.org/1998/Math/MathML"><semantics><mrow><mi>K</mi></mrow><annotation encoding="application/x-tex">K</annotation></semantics></math></span><span class="katex-html" aria-hidden="true"><span class="base"><span class="strut" style="height: 0.68333em; vertical-align: 0em;"></span><span class="mord mathnormal" style="margin-right: 0.07153em;">K</span></span></span></span></span> to be estimated. This is an issue that we will address drawing primarily on the work of \cite{SaadeBethe}.</p>
<p>\subsection{The Non-backtracking matrix}<br>
Let <span class="katex--inline"><span class="katex"><span class="katex-mathml"><math xmlns="http://www.w3.org/1998/Math/MathML"><semantics><mrow><mi>G</mi></mrow><annotation encoding="application/x-tex">G</annotation></semantics></math></span><span class="katex-html" aria-hidden="true"><span class="base"><span class="strut" style="height: 0.68333em; vertical-align: 0em;"></span><span class="mord mathnormal">G</span></span></span></span></span> be an undirected, connected network with <span class="katex--inline"><span class="katex"><span class="katex-mathml"><math xmlns="http://www.w3.org/1998/Math/MathML"><semantics><mrow><mi>n</mi></mrow><annotation encoding="application/x-tex">n</annotation></semantics></math></span><span class="katex-html" aria-hidden="true"><span class="base"><span class="strut" style="height: 0.43056em; vertical-align: 0em;"></span><span class="mord mathnormal">n</span></span></span></span></span> nodes and <span class="katex--inline"><span class="katex"><span class="katex-mathml"><math xmlns="http://www.w3.org/1998/Math/MathML"><semantics><mrow><mi>m</mi></mrow><annotation encoding="application/x-tex">m</annotation></semantics></math></span><span class="katex-html" aria-hidden="true"><span class="base"><span class="strut" style="height: 0.43056em; vertical-align: 0em;"></span><span class="mord mathnormal">m</span></span></span></span></span> edges. Let the adjacency matrix be denoted by A and the diagonal matrix of the degrees of each node be given by D. Then non-backtracking matrix of G is defined <span class="katex--inline"><span class="katex"><span class="katex-mathml"><math xmlns="http://www.w3.org/1998/Math/MathML"><semantics><mrow><mi>B</mi><mo>∈</mo><msup><mi mathvariant="double-struck">R</mi><mrow><mn>2</mn><mi>m</mi><mo>×</mo><mn>2</mn><mi>m</mi></mrow></msup></mrow><annotation encoding="application/x-tex">B \in \mathbb{R}^{2m \times 2m}</annotation></semantics></math></span><span class="katex-html" aria-hidden="true"><span class="base"><span class="strut" style="height: 0.72243em; vertical-align: -0.0391em;"></span><span class="mord mathnormal" style="margin-right: 0.05017em;">B</span><span class="mspace" style="margin-right: 0.277778em;"></span><span class="mrel">∈</span><span class="mspace" style="margin-right: 0.277778em;"></span></span><span class="base"><span class="strut" style="height: 0.814108em; vertical-align: 0em;"></span><span class="mord"><span class="mord mathbb">R</span><span class="msupsub"><span class="vlist-t"><span class="vlist-r"><span class="vlist" style="height: 0.814108em;"><span class="" style="top: -3.063em; margin-right: 0.05em;"><span class="pstrut" style="height: 2.7em;"></span><span class="sizing reset-size6 size3 mtight"><span class="mord mtight"><span class="mord mtight">2</span><span class="mord mathnormal mtight">m</span><span class="mbin mtight">×</span><span class="mord mtight">2</span><span class="mord mathnormal mtight">m</span></span></span></span></span></span></span></span></span></span></span></span></span> where:</p>
<p>$$B((u,v),(x,y)) = \begin{cases} &amp; 1 \text{ if } v = x, u \neq y \<br>
&amp; 0 \text{ otherwise }</p>
<p>\end{cases}<br>
$$</p>
<p>Where <span class="katex--inline"><span class="katex"><span class="katex-mathml"><math xmlns="http://www.w3.org/1998/Math/MathML"><semantics><mrow><mo stretchy="false">(</mo><mi>u</mi><mo separator="true">,</mo><mi>v</mi><mo stretchy="false">)</mo></mrow><annotation encoding="application/x-tex">(u,v)</annotation></semantics></math></span><span class="katex-html" aria-hidden="true"><span class="base"><span class="strut" style="height: 1em; vertical-align: -0.25em;"></span><span class="mopen">(</span><span class="mord mathnormal">u</span><span class="mpunct">,</span><span class="mspace" style="margin-right: 0.166667em;"></span><span class="mord mathnormal" style="margin-right: 0.03588em;">v</span><span class="mclose">)</span></span></span></span></span> and <span class="katex--inline"><span class="katex"><span class="katex-mathml"><math xmlns="http://www.w3.org/1998/Math/MathML"><semantics><mrow><mo stretchy="false">(</mo><mi>x</mi><mo separator="true">,</mo><mi>y</mi><mo stretchy="false">)</mo></mrow><annotation encoding="application/x-tex">(x,y)</annotation></semantics></math></span><span class="katex-html" aria-hidden="true"><span class="base"><span class="strut" style="height: 1em; vertical-align: -0.25em;"></span><span class="mopen">(</span><span class="mord mathnormal">x</span><span class="mpunct">,</span><span class="mspace" style="margin-right: 0.166667em;"></span><span class="mord mathnormal" style="margin-right: 0.03588em;">y</span><span class="mclose">)</span></span></span></span></span> are ‘directed’ edges of G. G does not need to be a directed network, we just assign direction both ways to each edge so that <span class="katex--inline"><span class="katex"><span class="katex-mathml"><math xmlns="http://www.w3.org/1998/Math/MathML"><semantics><mrow><mo stretchy="false">(</mo><mi>u</mi><mo separator="true">,</mo><mi>v</mi><mo stretchy="false">)</mo></mrow><annotation encoding="application/x-tex">(u,v)</annotation></semantics></math></span><span class="katex-html" aria-hidden="true"><span class="base"><span class="strut" style="height: 1em; vertical-align: -0.25em;"></span><span class="mopen">(</span><span class="mord mathnormal">u</span><span class="mpunct">,</span><span class="mspace" style="margin-right: 0.166667em;"></span><span class="mord mathnormal" style="margin-right: 0.03588em;">v</span><span class="mclose">)</span></span></span></span></span> is distinct from <span class="katex--inline"><span class="katex"><span class="katex-mathml"><math xmlns="http://www.w3.org/1998/Math/MathML"><semantics><mrow><mo stretchy="false">(</mo><mi>v</mi><mo separator="true">,</mo><mi>u</mi><mo stretchy="false">)</mo></mrow><annotation encoding="application/x-tex">(v,u)</annotation></semantics></math></span><span class="katex-html" aria-hidden="true"><span class="base"><span class="strut" style="height: 1em; vertical-align: -0.25em;"></span><span class="mopen">(</span><span class="mord mathnormal" style="margin-right: 0.03588em;">v</span><span class="mpunct">,</span><span class="mspace" style="margin-right: 0.166667em;"></span><span class="mord mathnormal">u</span><span class="mclose">)</span></span></span></span></span> for nodes <span class="katex--inline"><span class="katex"><span class="katex-mathml"><math xmlns="http://www.w3.org/1998/Math/MathML"><semantics><mrow><mi>u</mi></mrow><annotation encoding="application/x-tex">u</annotation></semantics></math></span><span class="katex-html" aria-hidden="true"><span class="base"><span class="strut" style="height: 0.43056em; vertical-align: 0em;"></span><span class="mord mathnormal">u</span></span></span></span></span> and <span class="katex--inline"><span class="katex"><span class="katex-mathml"><math xmlns="http://www.w3.org/1998/Math/MathML"><semantics><mrow><mi>v</mi></mrow><annotation encoding="application/x-tex">v</annotation></semantics></math></span><span class="katex-html" aria-hidden="true"><span class="base"><span class="strut" style="height: 0.43056em; vertical-align: 0em;"></span><span class="mord mathnormal" style="margin-right: 0.03588em;">v</span></span></span></span></span>.</p>
<p>It has been shown that studying the spectral properties of <span class="katex--inline"><span class="katex"><span class="katex-mathml"><math xmlns="http://www.w3.org/1998/Math/MathML"><semantics><mrow><mi>B</mi></mrow><annotation encoding="application/x-tex">B</annotation></semantics></math></span><span class="katex-html" aria-hidden="true"><span class="base"><span class="strut" style="height: 0.68333em; vertical-align: 0em;"></span><span class="mord mathnormal" style="margin-right: 0.05017em;">B</span></span></span></span></span> can aid community detection \cite{Krzakala20935}. Indeed, its eigenvalues are far more well-behaved than those of the adjacency matrix with B having <span class="katex--inline"><span class="katex"><span class="katex-mathml"><math xmlns="http://www.w3.org/1998/Math/MathML"><semantics><mrow><mn>2</mn><mo stretchy="false">(</mo><mi>m</mi><mo>−</mo><mi>n</mi><mo stretchy="false">)</mo></mrow><annotation encoding="application/x-tex">2(m-n)</annotation></semantics></math></span><span class="katex-html" aria-hidden="true"><span class="base"><span class="strut" style="height: 1em; vertical-align: -0.25em;"></span><span class="mord">2</span><span class="mopen">(</span><span class="mord mathnormal">m</span><span class="mspace" style="margin-right: 0.222222em;"></span><span class="mbin">−</span><span class="mspace" style="margin-right: 0.222222em;"></span></span><span class="base"><span class="strut" style="height: 1em; vertical-align: -0.25em;"></span><span class="mord mathnormal">n</span><span class="mclose">)</span></span></span></span></span> eigenvalues that are <span class="katex--inline"><span class="katex"><span class="katex-mathml"><math xmlns="http://www.w3.org/1998/Math/MathML"><semantics><mrow><mo>±</mo><mn>1</mn></mrow><annotation encoding="application/x-tex">\pm 1</annotation></semantics></math></span><span class="katex-html" aria-hidden="true"><span class="base"><span class="strut" style="height: 0.72777em; vertical-align: -0.08333em;"></span><span class="mord">±</span><span class="mord">1</span></span></span></span></span> and the rest are given as the eigenvalues of a matrix <span class="katex--inline"><span class="katex"><span class="katex-mathml"><math xmlns="http://www.w3.org/1998/Math/MathML"><semantics><mrow><mi>W</mi></mrow><annotation encoding="application/x-tex">W</annotation></semantics></math></span><span class="katex-html" aria-hidden="true"><span class="base"><span class="strut" style="height: 0.68333em; vertical-align: 0em;"></span><span class="mord mathnormal" style="margin-right: 0.13889em;">W</span></span></span></span></span>:</p>
<p><span class="katex--display"><span class="katex-display"><span class="katex"><span class="katex-mathml"><math xmlns="http://www.w3.org/1998/Math/MathML" display="block"><semantics><mrow><mi>W</mi><mo>=</mo><mrow><mo fence="true">[</mo><mtable rowspacing="0.1600em" columnalign="center center" columnspacing="1em"><mtr><mtd><mstyle scriptlevel="0" displaystyle="false"><mi>A</mi></mstyle></mtd><mtd><mstyle scriptlevel="0" displaystyle="false"><mrow><mi>D</mi><mo>−</mo><mi>I</mi></mrow></mstyle></mtd></mtr><mtr><mtd><mstyle scriptlevel="0" displaystyle="false"><mrow><mo>−</mo><mi>I</mi></mrow></mstyle></mtd><mtd><mstyle scriptlevel="0" displaystyle="false"><mn>0</mn></mstyle></mtd></mtr></mtable><mo fence="true">]</mo></mrow></mrow><annotation encoding="application/x-tex">W = \left[ \begin{array}{c c}
                                A &amp; D - I \\
                                -I &amp; 0            \end{array} \right] </annotation></semantics></math></span><span class="katex-html" aria-hidden="true"><span class="base"><span class="strut" style="height: 0.68333em; vertical-align: 0em;"></span><span class="mord mathnormal" style="margin-right: 0.13889em;">W</span><span class="mspace" style="margin-right: 0.277778em;"></span><span class="mrel">=</span><span class="mspace" style="margin-right: 0.277778em;"></span></span><span class="base"><span class="strut" style="height: 2.40003em; vertical-align: -0.95003em;"></span><span class="minner"><span class="mopen delimcenter" style="top: 0em;"><span class="delimsizing size3">[</span></span><span class="mord"><span class="mtable"><span class="arraycolsep" style="width: 0.5em;"></span><span class="col-align-c"><span class="vlist-t vlist-t2"><span class="vlist-r"><span class="vlist" style="height: 1.45em;"><span class="" style="top: -3.61em;"><span class="pstrut" style="height: 3em;"></span><span class="mord"><span class="mord mathnormal">A</span></span></span><span class="" style="top: -2.41em;"><span class="pstrut" style="height: 3em;"></span><span class="mord"><span class="mord">−</span><span class="mord mathnormal" style="margin-right: 0.07847em;">I</span></span></span></span><span class="vlist-s">​</span></span><span class="vlist-r"><span class="vlist" style="height: 0.95em;"><span class=""></span></span></span></span></span><span class="arraycolsep" style="width: 0.5em;"></span><span class="arraycolsep" style="width: 0.5em;"></span><span class="col-align-c"><span class="vlist-t vlist-t2"><span class="vlist-r"><span class="vlist" style="height: 1.45em;"><span class="" style="top: -3.61em;"><span class="pstrut" style="height: 3em;"></span><span class="mord"><span class="mord mathnormal" style="margin-right: 0.02778em;">D</span><span class="mspace" style="margin-right: 0.222222em;"></span><span class="mbin">−</span><span class="mspace" style="margin-right: 0.222222em;"></span><span class="mord mathnormal" style="margin-right: 0.07847em;">I</span></span></span><span class="" style="top: -2.41em;"><span class="pstrut" style="height: 3em;"></span><span class="mord"><span class="mord">0</span></span></span></span><span class="vlist-s">​</span></span><span class="vlist-r"><span class="vlist" style="height: 0.95em;"><span class=""></span></span></span></span></span><span class="arraycolsep" style="width: 0.5em;"></span></span></span><span class="mclose delimcenter" style="top: 0em;"><span class="delimsizing size3">]</span></span></span></span></span></span></span></span><br>
This is Ihara’s Theorem and we outline the proof provided in \cite{glover2020spectral} in the supplementary material.<br>
\newline<br>
Suppose that an eigenvector of W is <span class="katex--inline"><span class="katex"><span class="katex-mathml"><math xmlns="http://www.w3.org/1998/Math/MathML"><semantics><mrow><msup><mrow><mo fence="true">[</mo><mi>x</mi><mo separator="true">,</mo><mi>y</mi><mo fence="true">]</mo></mrow><mi>T</mi></msup></mrow><annotation encoding="application/x-tex">\left[ x , y \right]^T</annotation></semantics></math></span><span class="katex-html" aria-hidden="true"><span class="base"><span class="strut" style="height: 1.23123em; vertical-align: -0.25em;"></span><span class="minner"><span class="minner"><span class="mopen delimcenter" style="top: 0em;">[</span><span class="mord mathnormal">x</span><span class="mpunct">,</span><span class="mspace" style="margin-right: 0.166667em;"></span><span class="mord mathnormal" style="margin-right: 0.03588em;">y</span><span class="mclose delimcenter" style="top: 0em;">]</span></span><span class="msupsub"><span class="vlist-t"><span class="vlist-r"><span class="vlist" style="height: 0.981231em;"><span class="" style="top: -3.2029em; margin-right: 0.05em;"><span class="pstrut" style="height: 2.7em;"></span><span class="sizing reset-size6 size3 mtight"><span class="mord mathnormal mtight" style="margin-right: 0.13889em;">T</span></span></span></span></span></span></span></span></span></span></span></span>, <span class="katex--inline"><span class="katex"><span class="katex-mathml"><math xmlns="http://www.w3.org/1998/Math/MathML"><semantics><mrow><mi>x</mi><mo separator="true">,</mo><mi>y</mi><mo>∈</mo><msup><mi mathvariant="double-struck">R</mi><mi>n</mi></msup></mrow><annotation encoding="application/x-tex">x,y \in \mathbb{R}^n</annotation></semantics></math></span><span class="katex-html" aria-hidden="true"><span class="base"><span class="strut" style="height: 0.73354em; vertical-align: -0.19444em;"></span><span class="mord mathnormal">x</span><span class="mpunct">,</span><span class="mspace" style="margin-right: 0.166667em;"></span><span class="mord mathnormal" style="margin-right: 0.03588em;">y</span><span class="mspace" style="margin-right: 0.277778em;"></span><span class="mrel">∈</span><span class="mspace" style="margin-right: 0.277778em;"></span></span><span class="base"><span class="strut" style="height: 0.68889em; vertical-align: 0em;"></span><span class="mord"><span class="mord mathbb">R</span><span class="msupsub"><span class="vlist-t"><span class="vlist-r"><span class="vlist" style="height: 0.664392em;"><span class="" style="top: -3.063em; margin-right: 0.05em;"><span class="pstrut" style="height: 2.7em;"></span><span class="sizing reset-size6 size3 mtight"><span class="mord mathnormal mtight">n</span></span></span></span></span></span></span></span></span></span></span></span> with corresponding eigenvalue <span class="katex--inline"><span class="katex"><span class="katex-mathml"><math xmlns="http://www.w3.org/1998/Math/MathML"><semantics><mrow><mi>μ</mi></mrow><annotation encoding="application/x-tex">\mu</annotation></semantics></math></span><span class="katex-html" aria-hidden="true"><span class="base"><span class="strut" style="height: 0.625em; vertical-align: -0.19444em;"></span><span class="mord mathnormal">μ</span></span></span></span></span>. Then,</p>
<p>\begin{equation} \left[ \begin{array}{c c}<br>
A &amp; D - I \<br>
-I &amp; 0      \end{array} \right]<br>
\left[   \begin{array}{c}x \ y    \end{array} \right] = \left[ \begin{array}{c}<br>
\mu x  \ \mu y<br>
\end{array} \right]<br>
\label{eigenval B}<br>
\end{equation}<br>
Which implies that <span class="katex--inline"><span class="katex"><span class="katex-mathml"><math xmlns="http://www.w3.org/1998/Math/MathML"><semantics><mrow><mi>x</mi><mo>=</mo><mo>−</mo><mi>μ</mi><mi>y</mi></mrow><annotation encoding="application/x-tex">x = -\mu y</annotation></semantics></math></span><span class="katex-html" aria-hidden="true"><span class="base"><span class="strut" style="height: 0.43056em; vertical-align: 0em;"></span><span class="mord mathnormal">x</span><span class="mspace" style="margin-right: 0.277778em;"></span><span class="mrel">=</span><span class="mspace" style="margin-right: 0.277778em;"></span></span><span class="base"><span class="strut" style="height: 0.77777em; vertical-align: -0.19444em;"></span><span class="mord">−</span><span class="mord mathnormal">μ</span><span class="mord mathnormal" style="margin-right: 0.03588em;">y</span></span></span></span></span> and so consequently, every eigenpair of W is determined by the equation induced by the top row block:</p>
<pre><code>\begin{equation}
-\mu Ay + (D-I)y = -\mu^2y 
\label{quadratic eval}
\end{equation}.
</code></pre>
<p>\subsubsection{The non-backtracking matrix’s use in detecting block structure}<br>
A good question to ask at this point is ‘why is the non-backtracking matrix important in detecting community structure?’ There are already plenty of spectral clustering methods that rely on different matrices such as the adjacency matrix, the Laplacian, the normalised Laplacian or the modularity matrix (although the last two are almost identical). These methods rely on the matrix’s eigenvalues showing a clear distinction in how to determine the number of communities <span class="katex--inline"><span class="katex"><span class="katex-mathml"><math xmlns="http://www.w3.org/1998/Math/MathML"><semantics><mrow><mi>K</mi></mrow><annotation encoding="application/x-tex">K</annotation></semantics></math></span><span class="katex-html" aria-hidden="true"><span class="base"><span class="strut" style="height: 0.68333em; vertical-align: 0em;"></span><span class="mord mathnormal" style="margin-right: 0.07153em;">K</span></span></span></span></span>. For the (normalized) Laplacian, the number of communities is often intuited by the smallest <span class="katex--inline"><span class="katex"><span class="katex-mathml"><math xmlns="http://www.w3.org/1998/Math/MathML"><semantics><mrow><mi>K</mi></mrow><annotation encoding="application/x-tex">K</annotation></semantics></math></span><span class="katex-html" aria-hidden="true"><span class="base"><span class="strut" style="height: 0.68333em; vertical-align: 0em;"></span><span class="mord mathnormal" style="margin-right: 0.07153em;">K</span></span></span></span></span> eigenvalues before there is a ‘spectral gap’. In the case of a densely populated network with clear block structure this would be fine, but in the case of sparsely populated networks, spectral methods tend to struggle. It was theorised by \cite{decelle2011asymptotic} that there is an asymptotic limit for which communities can be detected from an SBM with equal sized communities and equal probabilities.</p>
<p>Denoting <span class="katex--inline"><span class="katex"><span class="katex-mathml"><math xmlns="http://www.w3.org/1998/Math/MathML"><semantics><mrow><msub><mi>p</mi><mrow><msub><mi>g</mi><mi>u</mi></msub><mo separator="true">,</mo><msub><mi>g</mi><mi>v</mi></msub></mrow></msub><mo>=</mo><mi mathvariant="double-struck">P</mi><mo stretchy="false">(</mo><msub><mi>A</mi><mrow><mi>u</mi><mi>v</mi></mrow></msub><mo>=</mo><mn>1</mn><mo stretchy="false">)</mo></mrow><annotation encoding="application/x-tex">p_{g_u,g_v} = \mathbb{P}(A_{uv} = 1)</annotation></semantics></math></span><span class="katex-html" aria-hidden="true"><span class="base"><span class="strut" style="height: 0.716668em; vertical-align: -0.286108em;"></span><span class="mord"><span class="mord mathnormal">p</span><span class="msupsub"><span class="vlist-t vlist-t2"><span class="vlist-r"><span class="vlist" style="height: 0.151392em;"><span class="" style="top: -2.55em; margin-left: 0em; margin-right: 0.05em;"><span class="pstrut" style="height: 2.7em;"></span><span class="sizing reset-size6 size3 mtight"><span class="mord mtight"><span class="mord mtight"><span class="mord mathnormal mtight" style="margin-right: 0.03588em;">g</span><span class="msupsub"><span class="vlist-t vlist-t2"><span class="vlist-r"><span class="vlist" style="height: 0.164543em;"><span class="" style="top: -2.357em; margin-left: -0.03588em; margin-right: 0.0714286em;"><span class="pstrut" style="height: 2.5em;"></span><span class="sizing reset-size3 size1 mtight"><span class="mord mathnormal mtight">u</span></span></span></span><span class="vlist-s">​</span></span><span class="vlist-r"><span class="vlist" style="height: 0.143em;"><span class=""></span></span></span></span></span></span><span class="mpunct mtight">,</span><span class="mord mtight"><span class="mord mathnormal mtight" style="margin-right: 0.03588em;">g</span><span class="msupsub"><span class="vlist-t vlist-t2"><span class="vlist-r"><span class="vlist" style="height: 0.164543em;"><span class="" style="top: -2.357em; margin-left: -0.03588em; margin-right: 0.0714286em;"><span class="pstrut" style="height: 2.5em;"></span><span class="sizing reset-size3 size1 mtight"><span class="mord mathnormal mtight" style="margin-right: 0.03588em;">v</span></span></span></span><span class="vlist-s">​</span></span><span class="vlist-r"><span class="vlist" style="height: 0.143em;"><span class=""></span></span></span></span></span></span></span></span></span></span><span class="vlist-s">​</span></span><span class="vlist-r"><span class="vlist" style="height: 0.286108em;"><span class=""></span></span></span></span></span></span><span class="mspace" style="margin-right: 0.277778em;"></span><span class="mrel">=</span><span class="mspace" style="margin-right: 0.277778em;"></span></span><span class="base"><span class="strut" style="height: 1em; vertical-align: -0.25em;"></span><span class="mord mathbb">P</span><span class="mopen">(</span><span class="mord"><span class="mord mathnormal">A</span><span class="msupsub"><span class="vlist-t vlist-t2"><span class="vlist-r"><span class="vlist" style="height: 0.151392em;"><span class="" style="top: -2.55em; margin-left: 0em; margin-right: 0.05em;"><span class="pstrut" style="height: 2.7em;"></span><span class="sizing reset-size6 size3 mtight"><span class="mord mtight"><span class="mord mathnormal mtight" style="margin-right: 0.03588em;">uv</span></span></span></span></span><span class="vlist-s">​</span></span><span class="vlist-r"><span class="vlist" style="height: 0.15em;"><span class=""></span></span></span></span></span></span><span class="mspace" style="margin-right: 0.277778em;"></span><span class="mrel">=</span><span class="mspace" style="margin-right: 0.277778em;"></span></span><span class="base"><span class="strut" style="height: 1em; vertical-align: -0.25em;"></span><span class="mord">1</span><span class="mclose">)</span></span></span></span></span>, let <span class="katex--inline"><span class="katex"><span class="katex-mathml"><math xmlns="http://www.w3.org/1998/Math/MathML"><semantics><mrow><msub><mi>c</mi><mrow><mi>a</mi><mi>b</mi></mrow></msub></mrow><annotation encoding="application/x-tex">c_{ab}</annotation></semantics></math></span><span class="katex-html" aria-hidden="true"><span class="base"><span class="strut" style="height: 0.58056em; vertical-align: -0.15em;"></span><span class="mord"><span class="mord mathnormal">c</span><span class="msupsub"><span class="vlist-t vlist-t2"><span class="vlist-r"><span class="vlist" style="height: 0.336108em;"><span class="" style="top: -2.55em; margin-left: 0em; margin-right: 0.05em;"><span class="pstrut" style="height: 2.7em;"></span><span class="sizing reset-size6 size3 mtight"><span class="mord mtight"><span class="mord mathnormal mtight">ab</span></span></span></span></span><span class="vlist-s">​</span></span><span class="vlist-r"><span class="vlist" style="height: 0.15em;"><span class=""></span></span></span></span></span></span></span></span></span></span><br>
be such that <span class="katex--inline"><span class="katex"><span class="katex-mathml"><math xmlns="http://www.w3.org/1998/Math/MathML"><semantics><mrow><msub><mi>c</mi><mrow><mi>a</mi><mi>b</mi></mrow></msub><mo>=</mo><mi>n</mi><msub><mi>p</mi><mrow><mi>a</mi><mi>b</mi></mrow></msub></mrow><annotation encoding="application/x-tex">c_{ab} = np_{ab}</annotation></semantics></math></span><span class="katex-html" aria-hidden="true"><span class="base"><span class="strut" style="height: 0.58056em; vertical-align: -0.15em;"></span><span class="mord"><span class="mord mathnormal">c</span><span class="msupsub"><span class="vlist-t vlist-t2"><span class="vlist-r"><span class="vlist" style="height: 0.336108em;"><span class="" style="top: -2.55em; margin-left: 0em; margin-right: 0.05em;"><span class="pstrut" style="height: 2.7em;"></span><span class="sizing reset-size6 size3 mtight"><span class="mord mtight"><span class="mord mathnormal mtight">ab</span></span></span></span></span><span class="vlist-s">​</span></span><span class="vlist-r"><span class="vlist" style="height: 0.15em;"><span class=""></span></span></span></span></span></span><span class="mspace" style="margin-right: 0.277778em;"></span><span class="mrel">=</span><span class="mspace" style="margin-right: 0.277778em;"></span></span><span class="base"><span class="strut" style="height: 0.625em; vertical-align: -0.19444em;"></span><span class="mord mathnormal">n</span><span class="mord"><span class="mord mathnormal">p</span><span class="msupsub"><span class="vlist-t vlist-t2"><span class="vlist-r"><span class="vlist" style="height: 0.336108em;"><span class="" style="top: -2.55em; margin-left: 0em; margin-right: 0.05em;"><span class="pstrut" style="height: 2.7em;"></span><span class="sizing reset-size6 size3 mtight"><span class="mord mtight"><span class="mord mathnormal mtight">ab</span></span></span></span></span><span class="vlist-s">​</span></span><span class="vlist-r"><span class="vlist" style="height: 0.15em;"><span class=""></span></span></span></span></span></span></span></span></span></span> where <span class="katex--inline"><span class="katex"><span class="katex-mathml"><math xmlns="http://www.w3.org/1998/Math/MathML"><semantics><mrow><mi>n</mi></mrow><annotation encoding="application/x-tex">n</annotation></semantics></math></span><span class="katex-html" aria-hidden="true"><span class="base"><span class="strut" style="height: 0.43056em; vertical-align: 0em;"></span><span class="mord mathnormal">n</span></span></span></span></span> is the number of nodes in the SBM and further suppose <span class="katex--inline"><span class="katex"><span class="katex-mathml"><math xmlns="http://www.w3.org/1998/Math/MathML"><semantics><mrow><msub><mi>p</mi><mrow><mi>a</mi><mi>b</mi></mrow></msub></mrow><annotation encoding="application/x-tex">p_{ab}</annotation></semantics></math></span><span class="katex-html" aria-hidden="true"><span class="base"><span class="strut" style="height: 0.625em; vertical-align: -0.19444em;"></span><span class="mord"><span class="mord mathnormal">p</span><span class="msupsub"><span class="vlist-t vlist-t2"><span class="vlist-r"><span class="vlist" style="height: 0.336108em;"><span class="" style="top: -2.55em; margin-left: 0em; margin-right: 0.05em;"><span class="pstrut" style="height: 2.7em;"></span><span class="sizing reset-size6 size3 mtight"><span class="mord mtight"><span class="mord mathnormal mtight">ab</span></span></span></span></span><span class="vlist-s">​</span></span><span class="vlist-r"><span class="vlist" style="height: 0.15em;"><span class=""></span></span></span></span></span></span></span></span></span></span>, <span class="katex--inline"><span class="katex"><span class="katex-mathml"><math xmlns="http://www.w3.org/1998/Math/MathML"><semantics><mrow><mi>a</mi><mo mathvariant="normal">≠</mo><mi>b</mi></mrow><annotation encoding="application/x-tex">a \neq b</annotation></semantics></math></span><span class="katex-html" aria-hidden="true"><span class="base"><span class="strut" style="height: 0.88888em; vertical-align: -0.19444em;"></span><span class="mord mathnormal">a</span><span class="mspace" style="margin-right: 0.277778em;"></span><span class="mrel"><span class="mrel"><span class="mord vbox"><span class="thinbox"><span class="rlap"><span class="strut" style="height: 0.88888em; vertical-align: -0.19444em;"></span><span class="inner"><span class="mord"><span class="mrel"></span></span></span><span class="fix"></span></span></span></span></span><span class="mrel">=</span></span><span class="mspace" style="margin-right: 0.277778em;"></span></span><span class="base"><span class="strut" style="height: 0.69444em; vertical-align: 0em;"></span><span class="mord mathnormal">b</span></span></span></span></span> and <span class="katex--inline"><span class="katex"><span class="katex-mathml"><math xmlns="http://www.w3.org/1998/Math/MathML"><semantics><mrow><msub><mi>p</mi><mrow><mi>a</mi><mi>a</mi></mrow></msub></mrow><annotation encoding="application/x-tex">p_{aa}</annotation></semantics></math></span><span class="katex-html" aria-hidden="true"><span class="base"><span class="strut" style="height: 0.625em; vertical-align: -0.19444em;"></span><span class="mord"><span class="mord mathnormal">p</span><span class="msupsub"><span class="vlist-t vlist-t2"><span class="vlist-r"><span class="vlist" style="height: 0.151392em;"><span class="" style="top: -2.55em; margin-left: 0em; margin-right: 0.05em;"><span class="pstrut" style="height: 2.7em;"></span><span class="sizing reset-size6 size3 mtight"><span class="mord mtight"><span class="mord mathnormal mtight">aa</span></span></span></span></span><span class="vlist-s">​</span></span><span class="vlist-r"><span class="vlist" style="height: 0.15em;"><span class=""></span></span></span></span></span></span></span></span></span></span> are the same across all communities <span class="katex--inline"><span class="katex"><span class="katex-mathml"><math xmlns="http://www.w3.org/1998/Math/MathML"><semantics><mrow><mi>a</mi></mrow><annotation encoding="application/x-tex">a</annotation></semantics></math></span><span class="katex-html" aria-hidden="true"><span class="base"><span class="strut" style="height: 0.43056em; vertical-align: 0em;"></span><span class="mord mathnormal">a</span></span></span></span></span> and <span class="katex--inline"><span class="katex"><span class="katex-mathml"><math xmlns="http://www.w3.org/1998/Math/MathML"><semantics><mrow><mi>b</mi></mrow><annotation encoding="application/x-tex">b</annotation></semantics></math></span><span class="katex-html" aria-hidden="true"><span class="base"><span class="strut" style="height: 0.69444em; vertical-align: 0em;"></span><span class="mord mathnormal">b</span></span></span></span></span>. Then,</p>
<p>\begin{equation}<br>
\mid c_{ab}- c_{aa} \mid &gt; K^* \sqrt{c}<br>
\label{threshold}<br>
\end{equation}</p>
<p>where <span class="katex--inline"><span class="katex"><span class="katex-mathml"><math xmlns="http://www.w3.org/1998/Math/MathML"><semantics><mrow><mi>c</mi></mrow><annotation encoding="application/x-tex">c</annotation></semantics></math></span><span class="katex-html" aria-hidden="true"><span class="base"><span class="strut" style="height: 0.43056em; vertical-align: 0em;"></span><span class="mord mathnormal">c</span></span></span></span></span> is the mean degree of a node and <span class="katex--inline"><span class="katex"><span class="katex-mathml"><math xmlns="http://www.w3.org/1998/Math/MathML"><semantics><mrow><msup><mi>K</mi><mo>∗</mo></msup></mrow><annotation encoding="application/x-tex">K^*</annotation></semantics></math></span><span class="katex-html" aria-hidden="true"><span class="base"><span class="strut" style="height: 0.688696em; vertical-align: 0em;"></span><span class="mord"><span class="mord mathnormal" style="margin-right: 0.07153em;">K</span><span class="msupsub"><span class="vlist-t"><span class="vlist-r"><span class="vlist" style="height: 0.688696em;"><span class="" style="top: -3.063em; margin-right: 0.05em;"><span class="pstrut" style="height: 2.7em;"></span><span class="sizing reset-size6 size3 mtight"><span class="mbin mtight">∗</span></span></span></span></span></span></span></span></span></span></span></span> denotes the true number of communities.</p>
<p>Spectral methods exploiting the spectrum of the adjacency matrix, the (normalized) Laplacian or the modularity matrix fail to detect the correct number of communities before this limit is reached. This is therefore an issue for detecting block structure in sparse networks since they are more likely to have ‘weaker’, less clear structure with the out-link and in-link probability of communities closer in value.  \cite{Krzakala20935} demonstrated that the non-backtracking matrix has more clearly defined differences in it’s eigenvalues corresponding to communities and it’s uninformative excess eigenvalues. It has been noted that the eigenvalues of the non-backtracking matrix B are mostly contained within the complex circle of radius <span class="katex--inline"><span class="katex"><span class="katex-mathml"><math xmlns="http://www.w3.org/1998/Math/MathML"><semantics><mrow><msqrt><mrow><mo stretchy="false">(</mo><mi>ρ</mi><mo stretchy="false">(</mo><mi>B</mi><mo stretchy="false">)</mo><mo stretchy="false">)</mo></mrow></msqrt></mrow><annotation encoding="application/x-tex">\sqrt{(\rho(B))}</annotation></semantics></math></span><span class="katex-html" aria-hidden="true"><span class="base"><span class="strut" style="height: 1.24em; vertical-align: -0.305em;"></span><span class="mord sqrt"><span class="vlist-t vlist-t2"><span class="vlist-r"><span class="vlist" style="height: 0.935em;"><span class="svg-align" style="top: -3.2em;"><span class="pstrut" style="height: 3.2em;"></span><span class="mord" style="padding-left: 1em;"><span class="mopen">(</span><span class="mord mathnormal">ρ</span><span class="mopen">(</span><span class="mord mathnormal" style="margin-right: 0.05017em;">B</span><span class="mclose">))</span></span></span><span class="" style="top: -2.895em;"><span class="pstrut" style="height: 3.2em;"></span><span class="hide-tail" style="min-width: 1.02em; height: 1.28em;"><svg width="400em" height="1.28em" viewBox="0 0 400000 1296" preserveAspectRatio="xMinYMin slice"><path d="M263,681c0.7,0,18,39.7,52,119
c34,79.3,68.167,158.7,102.5,238c34.3,79.3,51.8,119.3,52.5,120
c340,-704.7,510.7,-1060.3,512,-1067
l0 -0
c4.7,-7.3,11,-11,19,-11
H40000v40H1012.3
s-271.3,567,-271.3,567c-38.7,80.7,-84,175,-136,283c-52,108,-89.167,185.3,-111.5,232
c-22.3,46.7,-33.8,70.3,-34.5,71c-4.7,4.7,-12.3,7,-23,7s-12,-1,-12,-1
s-109,-253,-109,-253c-72.7,-168,-109.3,-252,-110,-252c-10.7,8,-22,16.7,-34,26
c-22,17.3,-33.3,26,-34,26s-26,-26,-26,-26s76,-59,76,-59s76,-60,76,-60z
M1001 80h400000v40h-400000z"></path></svg></span></span></span><span class="vlist-s">​</span></span><span class="vlist-r"><span class="vlist" style="height: 0.305em;"><span class=""></span></span></span></span></span></span></span></span></span> and centre at the origin with <span class="katex--inline"><span class="katex"><span class="katex-mathml"><math xmlns="http://www.w3.org/1998/Math/MathML"><semantics><mrow><mi>ρ</mi><mo stretchy="false">(</mo><mi>B</mi><mo stretchy="false">)</mo></mrow><annotation encoding="application/x-tex">\rho(B)</annotation></semantics></math></span><span class="katex-html" aria-hidden="true"><span class="base"><span class="strut" style="height: 1em; vertical-align: -0.25em;"></span><span class="mord mathnormal">ρ</span><span class="mopen">(</span><span class="mord mathnormal" style="margin-right: 0.05017em;">B</span><span class="mclose">)</span></span></span></span></span> being the spectrum of <span class="katex--inline"><span class="katex"><span class="katex-mathml"><math xmlns="http://www.w3.org/1998/Math/MathML"><semantics><mrow><mi>B</mi></mrow><annotation encoding="application/x-tex">B</annotation></semantics></math></span><span class="katex-html" aria-hidden="true"><span class="base"><span class="strut" style="height: 0.68333em; vertical-align: 0em;"></span><span class="mord mathnormal" style="margin-right: 0.05017em;">B</span></span></span></span></span>. The real eigenvalues that stray far from this circle are the number of communities in the network with negative real eigenvalues outside the circle corresponding to disassortative communities and positive real eigenvalues to assortative communities. We provide some figures illustrating these details in the supplementary material.</p>
<p>\subsection{Bethe Hessian Matrix}<br>
We have that the potentially informative eigenvalues <span class="katex--inline"><span class="katex"><span class="katex-mathml"><math xmlns="http://www.w3.org/1998/Math/MathML"><semantics><mrow><mi>μ</mi></mrow><annotation encoding="application/x-tex">\mu</annotation></semantics></math></span><span class="katex-html" aria-hidden="true"><span class="base"><span class="strut" style="height: 0.625em; vertical-align: -0.19444em;"></span><span class="mord mathnormal">μ</span></span></span></span></span> of the non-backtracking matrix are given by:<br>
<span class="katex--display"><span class="katex-display"><span class="katex"><span class="katex-mathml"><math xmlns="http://www.w3.org/1998/Math/MathML" display="block"><semantics><mrow><mi>det</mi><mo>⁡</mo><mo stretchy="false">(</mo><mo stretchy="false">(</mo><msup><mi>μ</mi><mn>2</mn></msup><mo>−</mo><mn>1</mn><mo stretchy="false">)</mo><mi>I</mi><mo>−</mo><mi>μ</mi><mi>A</mi><mo>+</mo><mi>D</mi><mo stretchy="false">)</mo></mrow><annotation encoding="application/x-tex"> \det ((\mu^2 - 1)I - \mu A + D)</annotation></semantics></math></span><span class="katex-html" aria-hidden="true"><span class="base"><span class="strut" style="height: 1.11411em; vertical-align: -0.25em;"></span><span class="mop">det</span><span class="mopen">((</span><span class="mord"><span class="mord mathnormal">μ</span><span class="msupsub"><span class="vlist-t"><span class="vlist-r"><span class="vlist" style="height: 0.864108em;"><span class="" style="top: -3.113em; margin-right: 0.05em;"><span class="pstrut" style="height: 2.7em;"></span><span class="sizing reset-size6 size3 mtight"><span class="mord mtight">2</span></span></span></span></span></span></span></span><span class="mspace" style="margin-right: 0.222222em;"></span><span class="mbin">−</span><span class="mspace" style="margin-right: 0.222222em;"></span></span><span class="base"><span class="strut" style="height: 1em; vertical-align: -0.25em;"></span><span class="mord">1</span><span class="mclose">)</span><span class="mord mathnormal" style="margin-right: 0.07847em;">I</span><span class="mspace" style="margin-right: 0.222222em;"></span><span class="mbin">−</span><span class="mspace" style="margin-right: 0.222222em;"></span></span><span class="base"><span class="strut" style="height: 0.87777em; vertical-align: -0.19444em;"></span><span class="mord mathnormal">μ</span><span class="mord mathnormal">A</span><span class="mspace" style="margin-right: 0.222222em;"></span><span class="mbin">+</span><span class="mspace" style="margin-right: 0.222222em;"></span></span><span class="base"><span class="strut" style="height: 1em; vertical-align: -0.25em;"></span><span class="mord mathnormal" style="margin-right: 0.02778em;">D</span><span class="mclose">)</span></span></span></span></span></span></p>
<p>The matrix <span class="katex--inline"><span class="katex"><span class="katex-mathml"><math xmlns="http://www.w3.org/1998/Math/MathML"><semantics><mrow><mi>H</mi><mo stretchy="false">(</mo><mi>r</mi><mo stretchy="false">)</mo><mo>:</mo><mo>=</mo><mo stretchy="false">(</mo><mo stretchy="false">(</mo><msup><mi>r</mi><mn>2</mn></msup><mo>−</mo><mn>1</mn><mo stretchy="false">)</mo><mi>I</mi><mo>−</mo><mi>r</mi><mi>A</mi><mo>+</mo><mi>D</mi><mo stretchy="false">)</mo></mrow><annotation encoding="application/x-tex">H(r) := ((r^2 - 1)I - r A + D)</annotation></semantics></math></span><span class="katex-html" aria-hidden="true"><span class="base"><span class="strut" style="height: 1em; vertical-align: -0.25em;"></span><span class="mord mathnormal" style="margin-right: 0.08125em;">H</span><span class="mopen">(</span><span class="mord mathnormal" style="margin-right: 0.02778em;">r</span><span class="mclose">)</span><span class="mspace" style="margin-right: 0.277778em;"></span><span class="mrel">:=</span><span class="mspace" style="margin-right: 0.277778em;"></span></span><span class="base"><span class="strut" style="height: 1.06411em; vertical-align: -0.25em;"></span><span class="mopen">((</span><span class="mord"><span class="mord mathnormal" style="margin-right: 0.02778em;">r</span><span class="msupsub"><span class="vlist-t"><span class="vlist-r"><span class="vlist" style="height: 0.814108em;"><span class="" style="top: -3.063em; margin-right: 0.05em;"><span class="pstrut" style="height: 2.7em;"></span><span class="sizing reset-size6 size3 mtight"><span class="mord mtight">2</span></span></span></span></span></span></span></span><span class="mspace" style="margin-right: 0.222222em;"></span><span class="mbin">−</span><span class="mspace" style="margin-right: 0.222222em;"></span></span><span class="base"><span class="strut" style="height: 1em; vertical-align: -0.25em;"></span><span class="mord">1</span><span class="mclose">)</span><span class="mord mathnormal" style="margin-right: 0.07847em;">I</span><span class="mspace" style="margin-right: 0.222222em;"></span><span class="mbin">−</span><span class="mspace" style="margin-right: 0.222222em;"></span></span><span class="base"><span class="strut" style="height: 0.76666em; vertical-align: -0.08333em;"></span><span class="mord mathnormal" style="margin-right: 0.02778em;">r</span><span class="mord mathnormal">A</span><span class="mspace" style="margin-right: 0.222222em;"></span><span class="mbin">+</span><span class="mspace" style="margin-right: 0.222222em;"></span></span><span class="base"><span class="strut" style="height: 1em; vertical-align: -0.25em;"></span><span class="mord mathnormal" style="margin-right: 0.02778em;">D</span><span class="mclose">)</span></span></span></span></span> is called the Bethe Hessian matrix. As explained in \cite{SaadeBethe}, for large enough r, <span class="katex--inline"><span class="katex"><span class="katex-mathml"><math xmlns="http://www.w3.org/1998/Math/MathML"><semantics><mrow><mi>H</mi><mo stretchy="false">(</mo><mi>r</mi><mo stretchy="false">)</mo></mrow><annotation encoding="application/x-tex">H(r)</annotation></semantics></math></span><span class="katex-html" aria-hidden="true"><span class="base"><span class="strut" style="height: 1em; vertical-align: -0.25em;"></span><span class="mord mathnormal" style="margin-right: 0.08125em;">H</span><span class="mopen">(</span><span class="mord mathnormal" style="margin-right: 0.02778em;">r</span><span class="mclose">)</span></span></span></span></span> for a given network with adjacency matrix <span class="katex--inline"><span class="katex"><span class="katex-mathml"><math xmlns="http://www.w3.org/1998/Math/MathML"><semantics><mrow><mi>A</mi></mrow><annotation encoding="application/x-tex">A</annotation></semantics></math></span><span class="katex-html" aria-hidden="true"><span class="base"><span class="strut" style="height: 0.68333em; vertical-align: 0em;"></span><span class="mord mathnormal">A</span></span></span></span></span> and diagonal degree matrix <span class="katex--inline"><span class="katex"><span class="katex-mathml"><math xmlns="http://www.w3.org/1998/Math/MathML"><semantics><mrow><mi>D</mi></mrow><annotation encoding="application/x-tex">D</annotation></semantics></math></span><span class="katex-html" aria-hidden="true"><span class="base"><span class="strut" style="height: 0.68333em; vertical-align: 0em;"></span><span class="mord mathnormal" style="margin-right: 0.02778em;">D</span></span></span></span></span> will be positive definite. This is because of Gershgorin’s circle theorem which states that the eigenvalues of any complex matrix <span class="katex--inline"><span class="katex"><span class="katex-mathml"><math xmlns="http://www.w3.org/1998/Math/MathML"><semantics><mrow><mi>C</mi></mrow><annotation encoding="application/x-tex">C</annotation></semantics></math></span><span class="katex-html" aria-hidden="true"><span class="base"><span class="strut" style="height: 0.68333em; vertical-align: 0em;"></span><span class="mord mathnormal" style="margin-right: 0.07153em;">C</span></span></span></span></span> lie in the union of complex circles with centres given by the diagonal elements of <span class="katex--inline"><span class="katex"><span class="katex-mathml"><math xmlns="http://www.w3.org/1998/Math/MathML"><semantics><mrow><msub><mi>C</mi><mrow><mi>i</mi><mi>i</mi></mrow></msub></mrow><annotation encoding="application/x-tex">C_{ii}</annotation></semantics></math></span><span class="katex-html" aria-hidden="true"><span class="base"><span class="strut" style="height: 0.83333em; vertical-align: -0.15em;"></span><span class="mord"><span class="mord mathnormal" style="margin-right: 0.07153em;">C</span><span class="msupsub"><span class="vlist-t vlist-t2"><span class="vlist-r"><span class="vlist" style="height: 0.311664em;"><span class="" style="top: -2.55em; margin-left: -0.07153em; margin-right: 0.05em;"><span class="pstrut" style="height: 2.7em;"></span><span class="sizing reset-size6 size3 mtight"><span class="mord mtight"><span class="mord mathnormal mtight">ii</span></span></span></span></span><span class="vlist-s">​</span></span><span class="vlist-r"><span class="vlist" style="height: 0.15em;"><span class=""></span></span></span></span></span></span></span></span></span></span> and corresponding radii <span class="katex--inline"><span class="katex"><span class="katex-mathml"><math xmlns="http://www.w3.org/1998/Math/MathML"><semantics><mrow><msub><mo>∑</mo><mrow><mi>j</mi><mo mathvariant="normal">≠</mo><mi>i</mi></mrow></msub><msub><mi>C</mi><mrow><mi>i</mi><mi>j</mi></mrow></msub></mrow><annotation encoding="application/x-tex">\sum_{j \neq i} C_{ij}</annotation></semantics></math></span><span class="katex-html" aria-hidden="true"><span class="base"><span class="strut" style="height: 1.18582em; vertical-align: -0.435818em;"></span><span class="mop"><span class="mop op-symbol small-op" style="position: relative; top: -5e-06em;">∑</span><span class="msupsub"><span class="vlist-t vlist-t2"><span class="vlist-r"><span class="vlist" style="height: 0.186398em;"><span class="" style="top: -2.40029em; margin-left: 0em; margin-right: 0.05em;"><span class="pstrut" style="height: 2.7em;"></span><span class="sizing reset-size6 size3 mtight"><span class="mord mtight"><span class="mord mathnormal mtight" style="margin-right: 0.05724em;">j</span><span class="mrel mtight"><span class="mrel mtight"><span class="mord vbox mtight"><span class="thinbox mtight"><span class="rlap mtight"><span class="strut" style="height: 0.88888em; vertical-align: -0.19444em;"></span><span class="inner"><span class="mord mtight"><span class="mrel mtight"></span></span></span><span class="fix"></span></span></span></span></span><span class="mrel mtight">=</span></span><span class="mord mathnormal mtight">i</span></span></span></span></span><span class="vlist-s">​</span></span><span class="vlist-r"><span class="vlist" style="height: 0.435818em;"><span class=""></span></span></span></span></span></span><span class="mspace" style="margin-right: 0.166667em;"></span><span class="mord"><span class="mord mathnormal" style="margin-right: 0.07153em;">C</span><span class="msupsub"><span class="vlist-t vlist-t2"><span class="vlist-r"><span class="vlist" style="height: 0.311664em;"><span class="" style="top: -2.55em; margin-left: -0.07153em; margin-right: 0.05em;"><span class="pstrut" style="height: 2.7em;"></span><span class="sizing reset-size6 size3 mtight"><span class="mord mtight"><span class="mord mathnormal mtight" style="margin-right: 0.05724em;">ij</span></span></span></span></span><span class="vlist-s">​</span></span><span class="vlist-r"><span class="vlist" style="height: 0.286108em;"><span class=""></span></span></span></span></span></span></span></span></span></span>. Thus in the case of <span class="katex--inline"><span class="katex"><span class="katex-mathml"><math xmlns="http://www.w3.org/1998/Math/MathML"><semantics><mrow><mi>H</mi><mo stretchy="false">(</mo><mi>r</mi><mo stretchy="false">)</mo></mrow><annotation encoding="application/x-tex">H(r)</annotation></semantics></math></span><span class="katex-html" aria-hidden="true"><span class="base"><span class="strut" style="height: 1em; vertical-align: -0.25em;"></span><span class="mord mathnormal" style="margin-right: 0.08125em;">H</span><span class="mopen">(</span><span class="mord mathnormal" style="margin-right: 0.02778em;">r</span><span class="mclose">)</span></span></span></span></span>, as <span class="katex--inline"><span class="katex"><span class="katex-mathml"><math xmlns="http://www.w3.org/1998/Math/MathML"><semantics><mrow><mi>r</mi></mrow><annotation encoding="application/x-tex">r</annotation></semantics></math></span><span class="katex-html" aria-hidden="true"><span class="base"><span class="strut" style="height: 0.43056em; vertical-align: 0em;"></span><span class="mord mathnormal" style="margin-right: 0.02778em;">r</span></span></span></span></span> is increased, the diagonal elements grow with order <span class="katex--inline"><span class="katex"><span class="katex-mathml"><math xmlns="http://www.w3.org/1998/Math/MathML"><semantics><mrow><msup><mi>r</mi><mn>2</mn></msup></mrow><annotation encoding="application/x-tex">r^2</annotation></semantics></math></span><span class="katex-html" aria-hidden="true"><span class="base"><span class="strut" style="height: 0.814108em; vertical-align: 0em;"></span><span class="mord"><span class="mord mathnormal" style="margin-right: 0.02778em;">r</span><span class="msupsub"><span class="vlist-t"><span class="vlist-r"><span class="vlist" style="height: 0.814108em;"><span class="" style="top: -3.063em; margin-right: 0.05em;"><span class="pstrut" style="height: 2.7em;"></span><span class="sizing reset-size6 size3 mtight"><span class="mord mtight">2</span></span></span></span></span></span></span></span></span></span></span></span> and the off-diagonal elements with order <span class="katex--inline"><span class="katex"><span class="katex-mathml"><math xmlns="http://www.w3.org/1998/Math/MathML"><semantics><mrow><mi>r</mi></mrow><annotation encoding="application/x-tex">r</annotation></semantics></math></span><span class="katex-html" aria-hidden="true"><span class="base"><span class="strut" style="height: 0.43056em; vertical-align: 0em;"></span><span class="mord mathnormal" style="margin-right: 0.02778em;">r</span></span></span></span></span>. <span class="katex--inline"><span class="katex"><span class="katex-mathml"><math xmlns="http://www.w3.org/1998/Math/MathML"><semantics><mrow><mi>H</mi><mo stretchy="false">(</mo><mi>r</mi><mo stretchy="false">)</mo></mrow><annotation encoding="application/x-tex">H(r)</annotation></semantics></math></span><span class="katex-html" aria-hidden="true"><span class="base"><span class="strut" style="height: 1em; vertical-align: -0.25em;"></span><span class="mord mathnormal" style="margin-right: 0.08125em;">H</span><span class="mopen">(</span><span class="mord mathnormal" style="margin-right: 0.02778em;">r</span><span class="mclose">)</span></span></span></span></span> is also symmetric so all its eigenvalues are real. Thus it follows that for large enough r, we obtain eigenvalues in a union of intervals that all lie contained in the positive real numbers.<br>
\cite{SaadeBethe} go on to explain that if we start with large enough positive <span class="katex--inline"><span class="katex"><span class="katex-mathml"><math xmlns="http://www.w3.org/1998/Math/MathML"><semantics><mrow><mi>r</mi></mrow><annotation encoding="application/x-tex">r</annotation></semantics></math></span><span class="katex-html" aria-hidden="true"><span class="base"><span class="strut" style="height: 0.43056em; vertical-align: 0em;"></span><span class="mord mathnormal" style="margin-right: 0.02778em;">r</span></span></span></span></span> where <span class="katex--inline"><span class="katex"><span class="katex-mathml"><math xmlns="http://www.w3.org/1998/Math/MathML"><semantics><mrow><mi>H</mi><mo stretchy="false">(</mo><mi>r</mi><mo stretchy="false">)</mo></mrow><annotation encoding="application/x-tex">H(r)</annotation></semantics></math></span><span class="katex-html" aria-hidden="true"><span class="base"><span class="strut" style="height: 1em; vertical-align: -0.25em;"></span><span class="mord mathnormal" style="margin-right: 0.08125em;">H</span><span class="mopen">(</span><span class="mord mathnormal" style="margin-right: 0.02778em;">r</span><span class="mclose">)</span></span></span></span></span> is positive definite then as <span class="katex--inline"><span class="katex"><span class="katex-mathml"><math xmlns="http://www.w3.org/1998/Math/MathML"><semantics><mrow><mi>r</mi></mrow><annotation encoding="application/x-tex">r</annotation></semantics></math></span><span class="katex-html" aria-hidden="true"><span class="base"><span class="strut" style="height: 0.43056em; vertical-align: 0em;"></span><span class="mord mathnormal" style="margin-right: 0.02778em;">r</span></span></span></span></span> decreases, <span class="katex--inline"><span class="katex"><span class="katex-mathml"><math xmlns="http://www.w3.org/1998/Math/MathML"><semantics><mrow><mi>H</mi><mo stretchy="false">(</mo><mi>r</mi><mo stretchy="false">)</mo></mrow><annotation encoding="application/x-tex">H(r)</annotation></semantics></math></span><span class="katex-html" aria-hidden="true"><span class="base"><span class="strut" style="height: 1em; vertical-align: -0.25em;"></span><span class="mord mathnormal" style="margin-right: 0.08125em;">H</span><span class="mopen">(</span><span class="mord mathnormal" style="margin-right: 0.02778em;">r</span><span class="mclose">)</span></span></span></span></span> gains a negative eigenvalue exactly when <span class="katex--inline"><span class="katex"><span class="katex-mathml"><math xmlns="http://www.w3.org/1998/Math/MathML"><semantics><mrow><mi>d</mi><mi>e</mi><mi>t</mi><mo stretchy="false">(</mo><mi>H</mi><mo stretchy="false">(</mo><mi>r</mi><mo stretchy="false">)</mo><mo stretchy="false">)</mo></mrow><annotation encoding="application/x-tex">det(H(r))</annotation></semantics></math></span><span class="katex-html" aria-hidden="true"><span class="base"><span class="strut" style="height: 1em; vertical-align: -0.25em;"></span><span class="mord mathnormal">d</span><span class="mord mathnormal">e</span><span class="mord mathnormal">t</span><span class="mopen">(</span><span class="mord mathnormal" style="margin-right: 0.08125em;">H</span><span class="mopen">(</span><span class="mord mathnormal" style="margin-right: 0.02778em;">r</span><span class="mclose">))</span></span></span></span></span> crosses the r-axis. But this is of course when <span class="katex--inline"><span class="katex"><span class="katex-mathml"><math xmlns="http://www.w3.org/1998/Math/MathML"><semantics><mrow><mi>r</mi></mrow><annotation encoding="application/x-tex">r</annotation></semantics></math></span><span class="katex-html" aria-hidden="true"><span class="base"><span class="strut" style="height: 0.43056em; vertical-align: 0em;"></span><span class="mord mathnormal" style="margin-right: 0.02778em;">r</span></span></span></span></span> has dropped past an eigenvalue of the non-backtracking matrix <span class="katex--inline"><span class="katex"><span class="katex-mathml"><math xmlns="http://www.w3.org/1998/Math/MathML"><semantics><mrow><mi>B</mi></mrow><annotation encoding="application/x-tex">B</annotation></semantics></math></span><span class="katex-html" aria-hidden="true"><span class="base"><span class="strut" style="height: 0.68333em; vertical-align: 0em;"></span><span class="mord mathnormal" style="margin-right: 0.05017em;">B</span></span></span></span></span>. Therefore, the negative eigenvalues of <span class="katex--inline"><span class="katex"><span class="katex-mathml"><math xmlns="http://www.w3.org/1998/Math/MathML"><semantics><mrow><mi>H</mi><mo stretchy="false">(</mo><mi>r</mi><mo stretchy="false">)</mo><mo separator="true">,</mo><mi>r</mi><mo>&gt;</mo><mn>0</mn></mrow><annotation encoding="application/x-tex">H(r), r &gt; 0</annotation></semantics></math></span><span class="katex-html" aria-hidden="true"><span class="base"><span class="strut" style="height: 1em; vertical-align: -0.25em;"></span><span class="mord mathnormal" style="margin-right: 0.08125em;">H</span><span class="mopen">(</span><span class="mord mathnormal" style="margin-right: 0.02778em;">r</span><span class="mclose">)</span><span class="mpunct">,</span><span class="mspace" style="margin-right: 0.166667em;"></span><span class="mord mathnormal" style="margin-right: 0.02778em;">r</span><span class="mspace" style="margin-right: 0.277778em;"></span><span class="mrel">&gt;</span><span class="mspace" style="margin-right: 0.277778em;"></span></span><span class="base"><span class="strut" style="height: 0.64444em; vertical-align: 0em;"></span><span class="mord">0</span></span></span></span></span> correspond to the real positive eigenvalues of the non-backtracking matrix <span class="katex--inline"><span class="katex"><span class="katex-mathml"><math xmlns="http://www.w3.org/1998/Math/MathML"><semantics><mrow><mi>B</mi></mrow><annotation encoding="application/x-tex">B</annotation></semantics></math></span><span class="katex-html" aria-hidden="true"><span class="base"><span class="strut" style="height: 0.68333em; vertical-align: 0em;"></span><span class="mord mathnormal" style="margin-right: 0.05017em;">B</span></span></span></span></span> that are greater than <span class="katex--inline"><span class="katex"><span class="katex-mathml"><math xmlns="http://www.w3.org/1998/Math/MathML"><semantics><mrow><mi>r</mi></mrow><annotation encoding="application/x-tex">r</annotation></semantics></math></span><span class="katex-html" aria-hidden="true"><span class="base"><span class="strut" style="height: 0.43056em; vertical-align: 0em;"></span><span class="mord mathnormal" style="margin-right: 0.02778em;">r</span></span></span></span></span>. Noting that the diagonal term of <span class="katex--inline"><span class="katex"><span class="katex-mathml"><math xmlns="http://www.w3.org/1998/Math/MathML"><semantics><mrow><mi>H</mi><mo stretchy="false">(</mo><mi>r</mi><mo stretchy="false">)</mo></mrow><annotation encoding="application/x-tex">H(r)</annotation></semantics></math></span><span class="katex-html" aria-hidden="true"><span class="base"><span class="strut" style="height: 1em; vertical-align: -0.25em;"></span><span class="mord mathnormal" style="margin-right: 0.08125em;">H</span><span class="mopen">(</span><span class="mord mathnormal" style="margin-right: 0.02778em;">r</span><span class="mclose">)</span></span></span></span></span> is invariant to the sign of <span class="katex--inline"><span class="katex"><span class="katex-mathml"><math xmlns="http://www.w3.org/1998/Math/MathML"><semantics><mrow><mi>r</mi></mrow><annotation encoding="application/x-tex">r</annotation></semantics></math></span><span class="katex-html" aria-hidden="true"><span class="base"><span class="strut" style="height: 0.43056em; vertical-align: 0em;"></span><span class="mord mathnormal" style="margin-right: 0.02778em;">r</span></span></span></span></span>, the same idea applies where we take a large negative value of <span class="katex--inline"><span class="katex"><span class="katex-mathml"><math xmlns="http://www.w3.org/1998/Math/MathML"><semantics><mrow><mi>r</mi></mrow><annotation encoding="application/x-tex">r</annotation></semantics></math></span><span class="katex-html" aria-hidden="true"><span class="base"><span class="strut" style="height: 0.43056em; vertical-align: 0em;"></span><span class="mord mathnormal" style="margin-right: 0.02778em;">r</span></span></span></span></span> where guaranteeing <span class="katex--inline"><span class="katex"><span class="katex-mathml"><math xmlns="http://www.w3.org/1998/Math/MathML"><semantics><mrow><mi>H</mi><mo stretchy="false">(</mo><mi>r</mi><mo stretchy="false">)</mo></mrow><annotation encoding="application/x-tex">H(r)</annotation></semantics></math></span><span class="katex-html" aria-hidden="true"><span class="base"><span class="strut" style="height: 1em; vertical-align: -0.25em;"></span><span class="mord mathnormal" style="margin-right: 0.08125em;">H</span><span class="mopen">(</span><span class="mord mathnormal" style="margin-right: 0.02778em;">r</span><span class="mclose">)</span></span></span></span></span> is positive definite by Gershgorin’s circle Theorem and then consider increasing it until <span class="katex--inline"><span class="katex"><span class="katex-mathml"><math xmlns="http://www.w3.org/1998/Math/MathML"><semantics><mrow><mi>H</mi><mo stretchy="false">(</mo><mi>r</mi><mo stretchy="false">)</mo></mrow><annotation encoding="application/x-tex">H(r)</annotation></semantics></math></span><span class="katex-html" aria-hidden="true"><span class="base"><span class="strut" style="height: 1em; vertical-align: -0.25em;"></span><span class="mord mathnormal" style="margin-right: 0.08125em;">H</span><span class="mopen">(</span><span class="mord mathnormal" style="margin-right: 0.02778em;">r</span><span class="mclose">)</span></span></span></span></span> has a negative eigenvalue. It follows that negative eigenvalues of <span class="katex--inline"><span class="katex"><span class="katex-mathml"><math xmlns="http://www.w3.org/1998/Math/MathML"><semantics><mrow><mi>H</mi><mo stretchy="false">(</mo><mi>r</mi><mo stretchy="false">)</mo></mrow><annotation encoding="application/x-tex">H(r)</annotation></semantics></math></span><span class="katex-html" aria-hidden="true"><span class="base"><span class="strut" style="height: 1em; vertical-align: -0.25em;"></span><span class="mord mathnormal" style="margin-right: 0.08125em;">H</span><span class="mopen">(</span><span class="mord mathnormal" style="margin-right: 0.02778em;">r</span><span class="mclose">)</span></span></span></span></span> for negative <span class="katex--inline"><span class="katex"><span class="katex-mathml"><math xmlns="http://www.w3.org/1998/Math/MathML"><semantics><mrow><mi>r</mi></mrow><annotation encoding="application/x-tex">r</annotation></semantics></math></span><span class="katex-html" aria-hidden="true"><span class="base"><span class="strut" style="height: 0.43056em; vertical-align: 0em;"></span><span class="mord mathnormal" style="margin-right: 0.02778em;">r</span></span></span></span></span> correspond to negative real eigenvalues of <span class="katex--inline"><span class="katex"><span class="katex-mathml"><math xmlns="http://www.w3.org/1998/Math/MathML"><semantics><mrow><mi>B</mi></mrow><annotation encoding="application/x-tex">B</annotation></semantics></math></span><span class="katex-html" aria-hidden="true"><span class="base"><span class="strut" style="height: 0.68333em; vertical-align: 0em;"></span><span class="mord mathnormal" style="margin-right: 0.05017em;">B</span></span></span></span></span> less than <span class="katex--inline"><span class="katex"><span class="katex-mathml"><math xmlns="http://www.w3.org/1998/Math/MathML"><semantics><mrow><mi>r</mi></mrow><annotation encoding="application/x-tex">r</annotation></semantics></math></span><span class="katex-html" aria-hidden="true"><span class="base"><span class="strut" style="height: 0.43056em; vertical-align: 0em;"></span><span class="mord mathnormal" style="margin-right: 0.02778em;">r</span></span></span></span></span>.<br>
\newline<br>
Now as discussed previously, the number of real eigenvalues of <span class="katex--inline"><span class="katex"><span class="katex-mathml"><math xmlns="http://www.w3.org/1998/Math/MathML"><semantics><mrow><mi>B</mi></mrow><annotation encoding="application/x-tex">B</annotation></semantics></math></span><span class="katex-html" aria-hidden="true"><span class="base"><span class="strut" style="height: 0.68333em; vertical-align: 0em;"></span><span class="mord mathnormal" style="margin-right: 0.05017em;">B</span></span></span></span></span> outside the complex circle of radius <span class="katex--inline"><span class="katex"><span class="katex-mathml"><math xmlns="http://www.w3.org/1998/Math/MathML"><semantics><mrow><msqrt><mrow><mi>ρ</mi><mo stretchy="false">(</mo><mi>B</mi><mo stretchy="false">)</mo></mrow></msqrt></mrow><annotation encoding="application/x-tex">\sqrt{\rho(B)}</annotation></semantics></math></span><span class="katex-html" aria-hidden="true"><span class="base"><span class="strut" style="height: 1.24em; vertical-align: -0.305em;"></span><span class="mord sqrt"><span class="vlist-t vlist-t2"><span class="vlist-r"><span class="vlist" style="height: 0.935em;"><span class="svg-align" style="top: -3.2em;"><span class="pstrut" style="height: 3.2em;"></span><span class="mord" style="padding-left: 1em;"><span class="mord mathnormal">ρ</span><span class="mopen">(</span><span class="mord mathnormal" style="margin-right: 0.05017em;">B</span><span class="mclose">)</span></span></span><span class="" style="top: -2.895em;"><span class="pstrut" style="height: 3.2em;"></span><span class="hide-tail" style="min-width: 1.02em; height: 1.28em;"><svg width="400em" height="1.28em" viewBox="0 0 400000 1296" preserveAspectRatio="xMinYMin slice"><path d="M263,681c0.7,0,18,39.7,52,119
c34,79.3,68.167,158.7,102.5,238c34.3,79.3,51.8,119.3,52.5,120
c340,-704.7,510.7,-1060.3,512,-1067
l0 -0
c4.7,-7.3,11,-11,19,-11
H40000v40H1012.3
s-271.3,567,-271.3,567c-38.7,80.7,-84,175,-136,283c-52,108,-89.167,185.3,-111.5,232
c-22.3,46.7,-33.8,70.3,-34.5,71c-4.7,4.7,-12.3,7,-23,7s-12,-1,-12,-1
s-109,-253,-109,-253c-72.7,-168,-109.3,-252,-110,-252c-10.7,8,-22,16.7,-34,26
c-22,17.3,-33.3,26,-34,26s-26,-26,-26,-26s76,-59,76,-59s76,-60,76,-60z
M1001 80h400000v40h-400000z"></path></svg></span></span></span><span class="vlist-s">​</span></span><span class="vlist-r"><span class="vlist" style="height: 0.305em;"><span class=""></span></span></span></span></span></span></span></span></span> is the number of communities, so <span class="katex--inline"><span class="katex"><span class="katex-mathml"><math xmlns="http://www.w3.org/1998/Math/MathML"><semantics><mrow><mi>r</mi><mo>=</mo><mo>±</mo><msqrt><mrow><mi>ρ</mi><mrow><mo fence="true">(</mo><mi>B</mi><mo fence="true">)</mo></mrow></mrow></msqrt></mrow><annotation encoding="application/x-tex">r = \pm \sqrt{\rho\left(B\right)}</annotation></semantics></math></span><span class="katex-html" aria-hidden="true"><span class="base"><span class="strut" style="height: 0.43056em; vertical-align: 0em;"></span><span class="mord mathnormal" style="margin-right: 0.02778em;">r</span><span class="mspace" style="margin-right: 0.277778em;"></span><span class="mrel">=</span><span class="mspace" style="margin-right: 0.277778em;"></span></span><span class="base"><span class="strut" style="height: 1.24em; vertical-align: -0.305em;"></span><span class="mord">±</span><span class="mord sqrt"><span class="vlist-t vlist-t2"><span class="vlist-r"><span class="vlist" style="height: 0.935em;"><span class="svg-align" style="top: -3.2em;"><span class="pstrut" style="height: 3.2em;"></span><span class="mord" style="padding-left: 1em;"><span class="mord mathnormal">ρ</span><span class="mspace" style="margin-right: 0.166667em;"></span><span class="minner"><span class="mopen delimcenter" style="top: 0em;">(</span><span class="mord mathnormal" style="margin-right: 0.05017em;">B</span><span class="mclose delimcenter" style="top: 0em;">)</span></span></span></span><span class="" style="top: -2.895em;"><span class="pstrut" style="height: 3.2em;"></span><span class="hide-tail" style="min-width: 1.02em; height: 1.28em;"><svg width="400em" height="1.28em" viewBox="0 0 400000 1296" preserveAspectRatio="xMinYMin slice"><path d="M263,681c0.7,0,18,39.7,52,119
c34,79.3,68.167,158.7,102.5,238c34.3,79.3,51.8,119.3,52.5,120
c340,-704.7,510.7,-1060.3,512,-1067
l0 -0
c4.7,-7.3,11,-11,19,-11
H40000v40H1012.3
s-271.3,567,-271.3,567c-38.7,80.7,-84,175,-136,283c-52,108,-89.167,185.3,-111.5,232
c-22.3,46.7,-33.8,70.3,-34.5,71c-4.7,4.7,-12.3,7,-23,7s-12,-1,-12,-1
s-109,-253,-109,-253c-72.7,-168,-109.3,-252,-110,-252c-10.7,8,-22,16.7,-34,26
c-22,17.3,-33.3,26,-34,26s-26,-26,-26,-26s76,-59,76,-59s76,-60,76,-60z
M1001 80h400000v40h-400000z"></path></svg></span></span></span><span class="vlist-s">​</span></span><span class="vlist-r"><span class="vlist" style="height: 0.305em;"><span class=""></span></span></span></span></span></span></span></span></span> is a good choice of substitution into <span class="katex--inline"><span class="katex"><span class="katex-mathml"><math xmlns="http://www.w3.org/1998/Math/MathML"><semantics><mrow><mi>H</mi><mo stretchy="false">(</mo><mi>r</mi><mo stretchy="false">)</mo></mrow><annotation encoding="application/x-tex">H(r)</annotation></semantics></math></span><span class="katex-html" aria-hidden="true"><span class="base"><span class="strut" style="height: 1em; vertical-align: -0.25em;"></span><span class="mord mathnormal" style="margin-right: 0.08125em;">H</span><span class="mopen">(</span><span class="mord mathnormal" style="margin-right: 0.02778em;">r</span><span class="mclose">)</span></span></span></span></span> . The number of negative eigenvalues of <span class="katex--inline"><span class="katex"><span class="katex-mathml"><math xmlns="http://www.w3.org/1998/Math/MathML"><semantics><mrow><mi>H</mi><mo stretchy="false">(</mo><mo>−</mo><msqrt><mrow><mi>ρ</mi><mo stretchy="false">(</mo><mi>B</mi><mo stretchy="false">)</mo></mrow></msqrt><mo stretchy="false">)</mo></mrow><annotation encoding="application/x-tex">H(-\sqrt{\rho(B)})</annotation></semantics></math></span><span class="katex-html" aria-hidden="true"><span class="base"><span class="strut" style="height: 1.24em; vertical-align: -0.305em;"></span><span class="mord mathnormal" style="margin-right: 0.08125em;">H</span><span class="mopen">(</span><span class="mord">−</span><span class="mord sqrt"><span class="vlist-t vlist-t2"><span class="vlist-r"><span class="vlist" style="height: 0.935em;"><span class="svg-align" style="top: -3.2em;"><span class="pstrut" style="height: 3.2em;"></span><span class="mord" style="padding-left: 1em;"><span class="mord mathnormal">ρ</span><span class="mopen">(</span><span class="mord mathnormal" style="margin-right: 0.05017em;">B</span><span class="mclose">)</span></span></span><span class="" style="top: -2.895em;"><span class="pstrut" style="height: 3.2em;"></span><span class="hide-tail" style="min-width: 1.02em; height: 1.28em;"><svg width="400em" height="1.28em" viewBox="0 0 400000 1296" preserveAspectRatio="xMinYMin slice"><path d="M263,681c0.7,0,18,39.7,52,119
c34,79.3,68.167,158.7,102.5,238c34.3,79.3,51.8,119.3,52.5,120
c340,-704.7,510.7,-1060.3,512,-1067
l0 -0
c4.7,-7.3,11,-11,19,-11
H40000v40H1012.3
s-271.3,567,-271.3,567c-38.7,80.7,-84,175,-136,283c-52,108,-89.167,185.3,-111.5,232
c-22.3,46.7,-33.8,70.3,-34.5,71c-4.7,4.7,-12.3,7,-23,7s-12,-1,-12,-1
s-109,-253,-109,-253c-72.7,-168,-109.3,-252,-110,-252c-10.7,8,-22,16.7,-34,26
c-22,17.3,-33.3,26,-34,26s-26,-26,-26,-26s76,-59,76,-59s76,-60,76,-60z
M1001 80h400000v40h-400000z"></path></svg></span></span></span><span class="vlist-s">​</span></span><span class="vlist-r"><span class="vlist" style="height: 0.305em;"><span class=""></span></span></span></span></span><span class="mclose">)</span></span></span></span></span> are the number of disassortative communities and in a similar fashion, <span class="katex--inline"><span class="katex"><span class="katex-mathml"><math xmlns="http://www.w3.org/1998/Math/MathML"><semantics><mrow><mi>H</mi><mo stretchy="false">(</mo><msqrt><mrow><mi>ρ</mi><mo stretchy="false">(</mo><mi>B</mi><mo stretchy="false">)</mo></mrow></msqrt><mo stretchy="false">)</mo></mrow><annotation encoding="application/x-tex">H(\sqrt{\rho(B)})</annotation></semantics></math></span><span class="katex-html" aria-hidden="true"><span class="base"><span class="strut" style="height: 1.24em; vertical-align: -0.305em;"></span><span class="mord mathnormal" style="margin-right: 0.08125em;">H</span><span class="mopen">(</span><span class="mord sqrt"><span class="vlist-t vlist-t2"><span class="vlist-r"><span class="vlist" style="height: 0.935em;"><span class="svg-align" style="top: -3.2em;"><span class="pstrut" style="height: 3.2em;"></span><span class="mord" style="padding-left: 1em;"><span class="mord mathnormal">ρ</span><span class="mopen">(</span><span class="mord mathnormal" style="margin-right: 0.05017em;">B</span><span class="mclose">)</span></span></span><span class="" style="top: -2.895em;"><span class="pstrut" style="height: 3.2em;"></span><span class="hide-tail" style="min-width: 1.02em; height: 1.28em;"><svg width="400em" height="1.28em" viewBox="0 0 400000 1296" preserveAspectRatio="xMinYMin slice"><path d="M263,681c0.7,0,18,39.7,52,119
c34,79.3,68.167,158.7,102.5,238c34.3,79.3,51.8,119.3,52.5,120
c340,-704.7,510.7,-1060.3,512,-1067
l0 -0
c4.7,-7.3,11,-11,19,-11
H40000v40H1012.3
s-271.3,567,-271.3,567c-38.7,80.7,-84,175,-136,283c-52,108,-89.167,185.3,-111.5,232
c-22.3,46.7,-33.8,70.3,-34.5,71c-4.7,4.7,-12.3,7,-23,7s-12,-1,-12,-1
s-109,-253,-109,-253c-72.7,-168,-109.3,-252,-110,-252c-10.7,8,-22,16.7,-34,26
c-22,17.3,-33.3,26,-34,26s-26,-26,-26,-26s76,-59,76,-59s76,-60,76,-60z
M1001 80h400000v40h-400000z"></path></svg></span></span></span><span class="vlist-s">​</span></span><span class="vlist-r"><span class="vlist" style="height: 0.305em;"><span class=""></span></span></span></span></span><span class="mclose">)</span></span></span></span></span> gives the quantity of assortative communities. So the Bethe Hessian inherits <span class="katex--inline"><span class="katex"><span class="katex-mathml"><math xmlns="http://www.w3.org/1998/Math/MathML"><semantics><mrow><mi>B</mi></mrow><annotation encoding="application/x-tex">B</annotation></semantics></math></span><span class="katex-html" aria-hidden="true"><span class="base"><span class="strut" style="height: 0.68333em; vertical-align: 0em;"></span><span class="mord mathnormal" style="margin-right: 0.05017em;">B</span></span></span></span></span>'s ability to detect communities right down to the threshold. \cite{SaadeBethe} also argue that the eigenvectors of <span class="katex--inline"><span class="katex"><span class="katex-mathml"><math xmlns="http://www.w3.org/1998/Math/MathML"><semantics><mrow><mi>H</mi><mo stretchy="false">(</mo><mo>±</mo><msqrt><mrow><mi>ρ</mi><mo stretchy="false">(</mo><mi>B</mi><mo stretchy="false">)</mo></mrow></msqrt><mo stretchy="false">)</mo></mrow><annotation encoding="application/x-tex">H(\pm \sqrt{\rho(B)})</annotation></semantics></math></span><span class="katex-html" aria-hidden="true"><span class="base"><span class="strut" style="height: 1.24em; vertical-align: -0.305em;"></span><span class="mord mathnormal" style="margin-right: 0.08125em;">H</span><span class="mopen">(</span><span class="mord">±</span><span class="mord sqrt"><span class="vlist-t vlist-t2"><span class="vlist-r"><span class="vlist" style="height: 0.935em;"><span class="svg-align" style="top: -3.2em;"><span class="pstrut" style="height: 3.2em;"></span><span class="mord" style="padding-left: 1em;"><span class="mord mathnormal">ρ</span><span class="mopen">(</span><span class="mord mathnormal" style="margin-right: 0.05017em;">B</span><span class="mclose">)</span></span></span><span class="" style="top: -2.895em;"><span class="pstrut" style="height: 3.2em;"></span><span class="hide-tail" style="min-width: 1.02em; height: 1.28em;"><svg width="400em" height="1.28em" viewBox="0 0 400000 1296" preserveAspectRatio="xMinYMin slice"><path d="M263,681c0.7,0,18,39.7,52,119
c34,79.3,68.167,158.7,102.5,238c34.3,79.3,51.8,119.3,52.5,120
c340,-704.7,510.7,-1060.3,512,-1067
l0 -0
c4.7,-7.3,11,-11,19,-11
H40000v40H1012.3
s-271.3,567,-271.3,567c-38.7,80.7,-84,175,-136,283c-52,108,-89.167,185.3,-111.5,232
c-22.3,46.7,-33.8,70.3,-34.5,71c-4.7,4.7,-12.3,7,-23,7s-12,-1,-12,-1
s-109,-253,-109,-253c-72.7,-168,-109.3,-252,-110,-252c-10.7,8,-22,16.7,-34,26
c-22,17.3,-33.3,26,-34,26s-26,-26,-26,-26s76,-59,76,-59s76,-60,76,-60z
M1001 80h400000v40h-400000z"></path></svg></span></span></span><span class="vlist-s">​</span></span><span class="vlist-r"><span class="vlist" style="height: 0.305em;"><span class=""></span></span></span></span></span><span class="mclose">)</span></span></span></span></span> provide ‘the direction of the clusters’. In other words, they are highly correlated with the block assignments of the nodes.<br>
\newline<br>
We illustrate the power of using the Bethe Hessian with the K-means algorithm numerically in Table \ref{numericaltable}<br>
\begin{table}[h]%[tbhp]<br>
\centering<br>
\caption{A comparative table of the accuracy of the Bethe-Hessian and Laplacian clustering method for detection of block structure in SBMs}<br>
\begin{tabular}{lll}<br>
<span class="katex--inline"><span class="katex"><span class="katex-mathml"><math xmlns="http://www.w3.org/1998/Math/MathML"><semantics><mrow><mfrac><mrow><msub><mi>c</mi><mrow><mi>a</mi><mi>b</mi></mrow></msub><mo>−</mo><msub><mi>c</mi><mrow><mi>a</mi><mi>a</mi></mrow></msub></mrow><mrow><mn>3</mn><msqrt><mi>c</mi></msqrt></mrow></mfrac></mrow><annotation encoding="application/x-tex">\frac{c_{ab} - c_{aa}}{3\sqrt{c}}</annotation></semantics></math></span><span class="katex-html" aria-hidden="true"><span class="base"><span class="strut" style="height: 1.36219em; vertical-align: -0.538em;"></span><span class="mord"><span class="mopen nulldelimiter"></span><span class="mfrac"><span class="vlist-t vlist-t2"><span class="vlist-r"><span class="vlist" style="height: 0.824191em;"><span class="" style="top: -2.62587em;"><span class="pstrut" style="height: 3em;"></span><span class="sizing reset-size6 size3 mtight"><span class="mord mtight"><span class="mord mtight">3</span><span class="mord sqrt mtight"><span class="vlist-t vlist-t2"><span class="vlist-r"><span class="vlist" style="height: 0.805905em;"><span class="svg-align" style="top: -3em;"><span class="pstrut" style="height: 3em;"></span><span class="mord mtight" style="padding-left: 0.833em;"><span class="mord mathnormal mtight">c</span></span></span><span class="" style="top: -2.76591em;"><span class="pstrut" style="height: 3em;"></span><span class="hide-tail mtight" style="min-width: 0.853em; height: 1.08em;"><svg width="400em" height="1.08em" viewBox="0 0 400000 1080" preserveAspectRatio="xMinYMin slice"><path d="M95,702
c-2.7,0,-7.17,-2.7,-13.5,-8c-5.8,-5.3,-9.5,-10,-9.5,-14
c0,-2,0.3,-3.3,1,-4c1.3,-2.7,23.83,-20.7,67.5,-54
c44.2,-33.3,65.8,-50.3,66.5,-51c1.3,-1.3,3,-2,5,-2c4.7,0,8.7,3.3,12,10
s173,378,173,378c0.7,0,35.3,-71,104,-213c68.7,-142,137.5,-285,206.5,-429
c69,-144,104.5,-217.7,106.5,-221
l0 -0
c5.3,-9.3,12,-14,20,-14
H400000v40H845.2724
s-225.272,467,-225.272,467s-235,486,-235,486c-2.7,4.7,-9,7,-19,7
c-6,0,-10,-1,-12,-3s-194,-422,-194,-422s-65,47,-65,47z
M834 80h400000v40h-400000z"></path></svg></span></span></span><span class="vlist-s">​</span></span><span class="vlist-r"><span class="vlist" style="height: 0.234095em;"><span class=""></span></span></span></span></span></span></span></span><span class="" style="top: -3.23em;"><span class="pstrut" style="height: 3em;"></span><span class="frac-line" style="border-bottom-width: 0.04em;"></span></span><span class="" style="top: -3.41586em;"><span class="pstrut" style="height: 3em;"></span><span class="sizing reset-size6 size3 mtight"><span class="mord mtight"><span class="mord mtight"><span class="mord mathnormal mtight">c</span><span class="msupsub"><span class="vlist-t vlist-t2"><span class="vlist-r"><span class="vlist" style="height: 0.3448em;"><span class="" style="top: -2.34877em; margin-left: 0em; margin-right: 0.0714286em;"><span class="pstrut" style="height: 2.5em;"></span><span class="sizing reset-size3 size1 mtight"><span class="mord mtight"><span class="mord mathnormal mtight">ab</span></span></span></span></span><span class="vlist-s">​</span></span><span class="vlist-r"><span class="vlist" style="height: 0.151229em;"><span class=""></span></span></span></span></span></span><span class="mbin mtight">−</span><span class="mord mtight"><span class="mord mathnormal mtight">c</span><span class="msupsub"><span class="vlist-t vlist-t2"><span class="vlist-r"><span class="vlist" style="height: 0.164543em;"><span class="" style="top: -2.357em; margin-left: 0em; margin-right: 0.0714286em;"><span class="pstrut" style="height: 2.5em;"></span><span class="sizing reset-size3 size1 mtight"><span class="mord mtight"><span class="mord mathnormal mtight">aa</span></span></span></span></span><span class="vlist-s">​</span></span><span class="vlist-r"><span class="vlist" style="height: 0.143em;"><span class=""></span></span></span></span></span></span></span></span></span></span><span class="vlist-s">​</span></span><span class="vlist-r"><span class="vlist" style="height: 0.538em;"><span class=""></span></span></span></span></span><span class="mclose nulldelimiter"></span></span></span></span></span></span> &amp; Bethe Hessian &amp; Laplacian \<br>
\midrule<br>
2.936                                                                 &amp; 1             &amp; 1\<br>
2.004                                                                 &amp; 1             &amp; 1\<br>
1.524                                                                 &amp; 1             &amp; 0.983\<br>
1.224                                                                 &amp; 0.963         &amp; 0.01\<br>
1.026                                                                 &amp; 0.907         &amp; <span class="katex--inline"><span class="katex"><span class="katex-mathml"><math xmlns="http://www.w3.org/1998/Math/MathML"><semantics><mrow><mo>∼</mo></mrow><annotation encoding="application/x-tex">\sim</annotation></semantics></math></span><span class="katex-html" aria-hidden="true"><span class="base"><span class="strut" style="height: 0.36687em; vertical-align: 0em;"></span><span class="mrel">∼</span></span></span></span></span>\<br>
\bottomrule</p>
<p>\end{tabular}</p>
<p>\addtabletext{The SBM used to produce the above results has 300 nodes split into 3 assortative blocks as in Figure 1. The right column is a measure of how close the in/out probabilities in the SBM are to the threshold. The probabilities were kept the same for each block. The values in the second two columns are proportions of correctly assigned nodes by the K-means algorithm (K being 3 here) using eigenvectors taken from the Bethe-Hessian and Laplacian matrix respectively to give the node coordinate positions. }</p>
<p>\label{numericaltable}<br>
\end{table}</p>
<p>As one can see, the Bethe Hessian approach still has <span class="katex--inline"><span class="katex"><span class="katex-mathml"><math xmlns="http://www.w3.org/1998/Math/MathML"><semantics><mrow><mn>90.7</mn><mi mathvariant="normal">%</mi></mrow><annotation encoding="application/x-tex">90.7\%</annotation></semantics></math></span><span class="katex-html" aria-hidden="true"><span class="base"><span class="strut" style="height: 0.80556em; vertical-align: -0.05556em;"></span><span class="mord">90.7%</span></span></span></span></span> accuracy when close to the detectable threshold whereas the Laplacian approach loses accuracy well before then.</p>
<p>\section{An Application to stock market data}<br>
We now explore an application of the Bethe Hessian matrix using an original data set.<br>
\newline<br>
An example of potentially sparse graphs with block structure are those generated by the correlation matrix of stocks and shares. More specifically, if we take the correlation of two stocks <span class="katex--inline"><span class="katex"><span class="katex-mathml"><math xmlns="http://www.w3.org/1998/Math/MathML"><semantics><mrow><mi>i</mi></mrow><annotation encoding="application/x-tex">i</annotation></semantics></math></span><span class="katex-html" aria-hidden="true"><span class="base"><span class="strut" style="height: 0.65952em; vertical-align: 0em;"></span><span class="mord mathnormal">i</span></span></span></span></span> and <span class="katex--inline"><span class="katex"><span class="katex-mathml"><math xmlns="http://www.w3.org/1998/Math/MathML"><semantics><mrow><mi>j</mi></mrow><annotation encoding="application/x-tex">j</annotation></semantics></math></span><span class="katex-html" aria-hidden="true"><span class="base"><span class="strut" style="height: 0.85396em; vertical-align: -0.19444em;"></span><span class="mord mathnormal" style="margin-right: 0.05724em;">j</span></span></span></span></span> over a certain time period e.g. 12 months, then this can be inputted into the <span class="katex--inline"><span class="katex"><span class="katex-mathml"><math xmlns="http://www.w3.org/1998/Math/MathML"><semantics><mrow><mo stretchy="false">(</mo><mi>i</mi><mo separator="true">,</mo><mi>j</mi><mo stretchy="false">)</mo></mrow><annotation encoding="application/x-tex">(i,j)</annotation></semantics></math></span><span class="katex-html" aria-hidden="true"><span class="base"><span class="strut" style="height: 1em; vertical-align: -0.25em;"></span><span class="mopen">(</span><span class="mord mathnormal">i</span><span class="mpunct">,</span><span class="mspace" style="margin-right: 0.166667em;"></span><span class="mord mathnormal" style="margin-right: 0.05724em;">j</span><span class="mclose">)</span></span></span></span></span> entry of matrix. This matrix can then be the adjacency matrix of a weighted, signed network. In our examples, we will only consider matrices of positive correlations and negative correlations and we will transform the entries to binary terms by setting correlations to 1 over a certain threshold and the rest to 0. This last step is acceptable since stock data is incredibly noisy.<br>
\newline<br>
We believe that it is an appropriate application since highly correlated stocks may form assortative and disassortative communities but with a high enough correlation threshold, the network will be relatively sparse.<br>
\newline<br>
We think that this is an interesting application too, since this concept could be applied to an investment portfolio. Often investment portfolios are classified by which types of stocks (and other asset classes) they focus on. For instance, an investment portfolio may be said to be more “growth-focused” if the majority of its investment are in high-growth technology companies. The risk levels of portfolios are often assessed by how ‘diversified’ they are and which types of stocks they include. Thus it may be important to see which stocks are considered a community since this may differ from their classification. A portfolio that is deemed ‘diversified’ may actually be less so.</p>
<p>\subsection{Technology vs. Oil}<br>
In the following experiment we have formed a correlation matrix of 20 large technology and 20 large oil and gas related companies.The correlations were taken over the daily closing price of each company over the past 12 months from 23rd March 2020 to 23rd March 2021. The data was drawn from Yahoo Finance and is available in the supplementary material. We then formed links between companies that had a positive correlation over 0.8. See figure \ref{techvsoil1}.</p>
<p>\begin{figure}[h]%[tbhp]<br>
\centering<br>
\includegraphics[width = 1.0\linewidth]{techvsoil.png}<br>
\caption{A network of 20 technology companies and 20 oil related companies with links drawn between companies for positive correlations in closing price over 0.8. This is not a particularly ‘sparse’ network. However there are quite a lot of connections between technology and oil related companies (the technology companies are the upper half of the semi-circle) and so the communities may be hard to detect for this reason by standard spectral methods.}<br>
\label{techvsoil1}<br>
\end{figure}</p>
<p>We have removed a single isolated node, the technology company Intel Corporation and we have removed a disconnected component of oil companies DSSI, FRO and DHT.<br>
\newline<br>
Now, one would assume that this network consists of two communities: the oil-related companies and the technology companies. Under this assumption, we see how the Bethe Hessian eigenvectors partition the network. We do this by forcing <span class="katex--inline"><span class="katex"><span class="katex-mathml"><math xmlns="http://www.w3.org/1998/Math/MathML"><semantics><mrow><mi>K</mi><mo>=</mo><mn>2</mn></mrow><annotation encoding="application/x-tex">K=2</annotation></semantics></math></span><span class="katex-html" aria-hidden="true"><span class="base"><span class="strut" style="height: 0.68333em; vertical-align: 0em;"></span><span class="mord mathnormal" style="margin-right: 0.07153em;">K</span><span class="mspace" style="margin-right: 0.277778em;"></span><span class="mrel">=</span><span class="mspace" style="margin-right: 0.277778em;"></span></span><span class="base"><span class="strut" style="height: 0.64444em; vertical-align: 0em;"></span><span class="mord">2</span></span></span></span></span> in the application of the K-means algorithm. Figure \ref{techvsoilcomm1} illustrate this partition.</p>
<p>\begin{figure}[h]%[tbhp]<br>
\centering<br>
\includegraphics[width=1.0\linewidth]{techvsoilcomm1.png}<br>
\caption{The communities detected by the Bethe-Hessian spectral clustering approach for a network of technology companies and oil and gas related companies with links created by positive correlations greater than 0.8.}<br>
\label{techvsoilcomm1}<br>
\end{figure}<br>
Now, this result is unsurprising. The method has split the companies up by oil-related and technology with the exception of erroneously including Dropbox (DBX) as an oil company (for which we do not know why but it is highly correlated with two oil-related companies.) It also has included Baker Hughes Company (BKR) as a technology company which we had originally classified as oil-related. Indeed, it is owned by GE Oil and Gas, but it is in fact a company that makes oilfield mining technology.<br>
\newline<br>
Applying the normalized Laplacian with K-means (still with <span class="katex--inline"><span class="katex"><span class="katex-mathml"><math xmlns="http://www.w3.org/1998/Math/MathML"><semantics><mrow><mi>K</mi><mo>=</mo><mn>2</mn></mrow><annotation encoding="application/x-tex">K=2</annotation></semantics></math></span><span class="katex-html" aria-hidden="true"><span class="base"><span class="strut" style="height: 0.68333em; vertical-align: 0em;"></span><span class="mord mathnormal" style="margin-right: 0.07153em;">K</span><span class="mspace" style="margin-right: 0.277778em;"></span><span class="mrel">=</span><span class="mspace" style="margin-right: 0.277778em;"></span></span><span class="base"><span class="strut" style="height: 0.64444em; vertical-align: 0em;"></span><span class="mord">2</span></span></span></span></span>), it produces the exact same partitioning which we do not include in the figures. By inferring from the graph of the Laplacian eigenvalues, it is hard to determine how many communities there should be from reading off the sizes of the eigenvalues. The normalized Laplacian does a better job of indicating this - see Figure \ref{eigval plots}.</p>
<p>\begin{figure}[h]<br>
\centering<br>
\includegraphics[width = 0.48\linewidth]{eigvalsLplot.png}<br>
\hfill<br>
\includegraphics[width = 0.48\linewidth]{eigvalsnLplot.png}<br>
\caption{The eigenvalues of the unnormalized and normalized Laplacian on the left and right respectively. These are the Laplacians formed from the network of technology and oil-related companies connected by positive correlations. }<br>
\label{eigval plots}<br>
\end{figure}<br>
The eigenvalues of the normalized Laplacian suggest there are 3 communities within this network. Confirming this result with the eigenvalues of the Bethe Hessian, we find that the Bethe Hessian has 3 negative eigenvalues for <span class="katex--inline"><span class="katex"><span class="katex-mathml"><math xmlns="http://www.w3.org/1998/Math/MathML"><semantics><mrow><mi>H</mi><mo stretchy="false">(</mo><msqrt><mrow><mi>ρ</mi><mo stretchy="false">(</mo><mi>B</mi><mo stretchy="false">)</mo></mrow></msqrt><mo stretchy="false">)</mo></mrow><annotation encoding="application/x-tex">H(\sqrt{\rho(B)})</annotation></semantics></math></span><span class="katex-html" aria-hidden="true"><span class="base"><span class="strut" style="height: 1.24em; vertical-align: -0.305em;"></span><span class="mord mathnormal" style="margin-right: 0.08125em;">H</span><span class="mopen">(</span><span class="mord sqrt"><span class="vlist-t vlist-t2"><span class="vlist-r"><span class="vlist" style="height: 0.935em;"><span class="svg-align" style="top: -3.2em;"><span class="pstrut" style="height: 3.2em;"></span><span class="mord" style="padding-left: 1em;"><span class="mord mathnormal">ρ</span><span class="mopen">(</span><span class="mord mathnormal" style="margin-right: 0.05017em;">B</span><span class="mclose">)</span></span></span><span class="" style="top: -2.895em;"><span class="pstrut" style="height: 3.2em;"></span><span class="hide-tail" style="min-width: 1.02em; height: 1.28em;"><svg width="400em" height="1.28em" viewBox="0 0 400000 1296" preserveAspectRatio="xMinYMin slice"><path d="M263,681c0.7,0,18,39.7,52,119
c34,79.3,68.167,158.7,102.5,238c34.3,79.3,51.8,119.3,52.5,120
c340,-704.7,510.7,-1060.3,512,-1067
l0 -0
c4.7,-7.3,11,-11,19,-11
H40000v40H1012.3
s-271.3,567,-271.3,567c-38.7,80.7,-84,175,-136,283c-52,108,-89.167,185.3,-111.5,232
c-22.3,46.7,-33.8,70.3,-34.5,71c-4.7,4.7,-12.3,7,-23,7s-12,-1,-12,-1
s-109,-253,-109,-253c-72.7,-168,-109.3,-252,-110,-252c-10.7,8,-22,16.7,-34,26
c-22,17.3,-33.3,26,-34,26s-26,-26,-26,-26s76,-59,76,-59s76,-60,76,-60z
M1001 80h400000v40h-400000z"></path></svg></span></span></span><span class="vlist-s">​</span></span><span class="vlist-r"><span class="vlist" style="height: 0.305em;"><span class=""></span></span></span></span></span><span class="mclose">)</span></span></span></span></span>, those being: -17.82268079, -10.64823615 and  -0.29985242. There were no negative eigenvalues for <span class="katex--inline"><span class="katex"><span class="katex-mathml"><math xmlns="http://www.w3.org/1998/Math/MathML"><semantics><mrow><mi>H</mi><mo stretchy="false">(</mo><mo>−</mo><msqrt><mrow><mi>ρ</mi><mo stretchy="false">(</mo><mi>B</mi><mo stretchy="false">)</mo></mrow></msqrt><mo stretchy="false">)</mo></mrow><annotation encoding="application/x-tex">H(-\sqrt{\rho(B)})</annotation></semantics></math></span><span class="katex-html" aria-hidden="true"><span class="base"><span class="strut" style="height: 1.24em; vertical-align: -0.305em;"></span><span class="mord mathnormal" style="margin-right: 0.08125em;">H</span><span class="mopen">(</span><span class="mord">−</span><span class="mord sqrt"><span class="vlist-t vlist-t2"><span class="vlist-r"><span class="vlist" style="height: 0.935em;"><span class="svg-align" style="top: -3.2em;"><span class="pstrut" style="height: 3.2em;"></span><span class="mord" style="padding-left: 1em;"><span class="mord mathnormal">ρ</span><span class="mopen">(</span><span class="mord mathnormal" style="margin-right: 0.05017em;">B</span><span class="mclose">)</span></span></span><span class="" style="top: -2.895em;"><span class="pstrut" style="height: 3.2em;"></span><span class="hide-tail" style="min-width: 1.02em; height: 1.28em;"><svg width="400em" height="1.28em" viewBox="0 0 400000 1296" preserveAspectRatio="xMinYMin slice"><path d="M263,681c0.7,0,18,39.7,52,119
c34,79.3,68.167,158.7,102.5,238c34.3,79.3,51.8,119.3,52.5,120
c340,-704.7,510.7,-1060.3,512,-1067
l0 -0
c4.7,-7.3,11,-11,19,-11
H40000v40H1012.3
s-271.3,567,-271.3,567c-38.7,80.7,-84,175,-136,283c-52,108,-89.167,185.3,-111.5,232
c-22.3,46.7,-33.8,70.3,-34.5,71c-4.7,4.7,-12.3,7,-23,7s-12,-1,-12,-1
s-109,-253,-109,-253c-72.7,-168,-109.3,-252,-110,-252c-10.7,8,-22,16.7,-34,26
c-22,17.3,-33.3,26,-34,26s-26,-26,-26,-26s76,-59,76,-59s76,-60,76,-60z
M1001 80h400000v40h-400000z"></path></svg></span></span></span><span class="vlist-s">​</span></span><span class="vlist-r"><span class="vlist" style="height: 0.305em;"><span class=""></span></span></span></span></span><span class="mclose">)</span></span></span></span></span> so we deduce that there are 3 assortative blocks and no disassortative ones. Figure \ref{techvsoil3comms} displays the partitioning given by Bethe Hessian spectral clustering approach.</p>
<p>\begin{figure}%[tbhp]<br>
\centering<br>
\includegraphics[width = 1.0\linewidth]{techvsoilcomm2.png}<br>
\caption{The three assortative communities detected by the Bethe-Hessian spectral clustering approach (using K-means) for our network of technology and oil-related companies. }<br>
\label{techvsoil3comms}<br>
\end{figure}</p>
<p>This result is very interesting. We see that there is a reasonably clear divide between technology and oil-related companies but we also are shown something more. We have now see a third community (in blue) residing mostly within technology (especially if we include BKR now as a technology company) consisting of TSLA, GOOG, HPQ, SNE, PCRFY, TWTR, XRX, MPC, BKR and PXD. A lot of these companies are noticeably more specialised in hardware than a lot of the companies included in the white, technology-dominant community. Companies such as Panasonic (PCRFY), Sony Corporation (SNE) and HP Inc. (HP) have large hardware focus and so have links with the oil companies, likely because of shipping materials. It should be noted that Dell Technologies Inc. (DELL) is the only node that changes and becomes a member of the blue community when we study the network with links created for positive correlations greater than 0.85. Some of this divide in the technology focused companies may also be stimulated by competition between these companies such as Twitter (TWTR) and Facebook (FB) or Apple (AAPL) and Google (GOOG).<br>
\newline<br>
An issue that must be addressed is that the third negative eigenvalues of the Bethe Hessian was -0.300 (3 d.p) suggesting intuitively that if there is a third community, it is nowhere near as heavily in-grained in the network as the other two. Another major limitation is the choice of threshold for the correlations. We simply chose the point before the network lost a majority of its links but this is completely arbitrary. A different choice of correlation may change the network and therefore the outcome drastically.</p>
<p>\newline<br>
We now consider the same nodes but with links generated via negative correlations. The negative correlations between these stocks are weaker so we chose the threshold to be where a link is generated if the correlation is less than -0.3. Many nodes were isolated and so have been removed to give the network in Figure \ref{techvsoilcomm1neg}. The tickers of technology companies that were removed are: DBX and XOM. The tickers of oil-related companies that were removed are RDS-B, TOT, NOG, COP, OGZPY, MPC, SNP, REPYY, VLO, OXY, PXD, OG, CVX leaving only BKR, BP and PBR.</p>
<p>\begin{figure}[h]%[tbhp]<br>
\centering<br>
\includegraphics[width=0.8\linewidth]{negativecorr tech vs oil.png}<br>
\caption{A network of technology and oil-related companies with links generated by negative correlations less than -0.3. }<br>
\label{techvsoilcomm1neg}<br>
\end{figure}<br>
Now the Bethe Hessian approach detects (quite rightly) a disassortative community of the three tanker companies. It also included BP (BP) and perhaps erroneously the technology company Intel corporation (INTC). In contrast, the normalized Laplacian failed to detect the community of oil tanker companies and sought an assortative community from a handful of sufficiently negatively correlated companies. These results can be see in Figure \ref{bethe v lap}.</p>
<p>\begin{figure}[h]%[tbhp]<br>
\centering<br>
\includegraphics[width = 0.48\linewidth]{Bethe Hess neg corr comms spring.png}<br>
\hfill<br>
\includegraphics[width = 0.48\linewidth]{norm lap neg corr comms spring.png}</p>
<pre><code>\caption{Left, Bethe Hessian spectral clustering approach for detection of assortative and disassortative communities in a network of technology and oil-related companies where links have been generated for negative correlations in share prices less than -0.3. Right, spectral clustering using the Laplacian.}
\label{bethe v lap}
</code></pre>
<p>\end{figure}</p>
<p>\section{Discussion and Potential Improvements}<br>
\subsection{A summary of implementation}<br>
As we have seen the Bethe Hessian spectral clustering method outcompetes other spectral methods in the detection of block structure in networks in certain circumstances. It is particularly useful in sparse networks and also in those with less distinct block structure where the number of links leaving a community is close in value to the number of links within it. Another merit of it is that it is not too computationally expensive. The non-backtracking matrix which typically has far larger dimension than the number of nodes, does not need to be calculated. We can obtain the necessary <span class="katex--inline"><span class="katex"><span class="katex-mathml"><math xmlns="http://www.w3.org/1998/Math/MathML"><semantics><mrow><mi>ρ</mi><mo stretchy="false">(</mo><mi>B</mi><mo stretchy="false">)</mo></mrow><annotation encoding="application/x-tex">\rho(B)</annotation></semantics></math></span><span class="katex-html" aria-hidden="true"><span class="base"><span class="strut" style="height: 1em; vertical-align: -0.25em;"></span><span class="mord mathnormal">ρ</span><span class="mopen">(</span><span class="mord mathnormal" style="margin-right: 0.05017em;">B</span><span class="mclose">)</span></span></span></span></span> by solving the quadratic eigenvalue problem in equation \ref{quadratic eval}. As discussed in section 2.A, this problem comes from the eigenvalue problem in equation \ref{eigenval B} for which there are plenty of solvers.<br>
We solved equation \ref{eigenval B} to obtain <span class="katex--inline"><span class="katex"><span class="katex-mathml"><math xmlns="http://www.w3.org/1998/Math/MathML"><semantics><mrow><mi>ρ</mi><mo stretchy="false">(</mo><mi>B</mi><mo stretchy="false">)</mo></mrow><annotation encoding="application/x-tex">\rho(B)</annotation></semantics></math></span><span class="katex-html" aria-hidden="true"><span class="base"><span class="strut" style="height: 1em; vertical-align: -0.25em;"></span><span class="mord mathnormal">ρ</span><span class="mopen">(</span><span class="mord mathnormal" style="margin-right: 0.05017em;">B</span><span class="mclose">)</span></span></span></span></span> using NumPy’s function \textit{eig} (see \textit{quadeig} in the supplementary material). MATLAB also has a function named \textit{polyeig} which would work.<br>
\newline<br>
<span class="katex--inline"><span class="katex"><span class="katex-mathml"><math xmlns="http://www.w3.org/1998/Math/MathML"><semantics><mrow><mi>H</mi><mo stretchy="false">(</mo><mo>±</mo><msqrt><mrow><mi>ρ</mi><mo stretchy="false">(</mo><mi>B</mi><mo stretchy="false">)</mo></mrow></msqrt><mo stretchy="false">)</mo></mrow><annotation encoding="application/x-tex">H(\pm \sqrt{\rho(B)})</annotation></semantics></math></span><span class="katex-html" aria-hidden="true"><span class="base"><span class="strut" style="height: 1.24em; vertical-align: -0.305em;"></span><span class="mord mathnormal" style="margin-right: 0.08125em;">H</span><span class="mopen">(</span><span class="mord">±</span><span class="mord sqrt"><span class="vlist-t vlist-t2"><span class="vlist-r"><span class="vlist" style="height: 0.935em;"><span class="svg-align" style="top: -3.2em;"><span class="pstrut" style="height: 3.2em;"></span><span class="mord" style="padding-left: 1em;"><span class="mord mathnormal">ρ</span><span class="mopen">(</span><span class="mord mathnormal" style="margin-right: 0.05017em;">B</span><span class="mclose">)</span></span></span><span class="" style="top: -2.895em;"><span class="pstrut" style="height: 3.2em;"></span><span class="hide-tail" style="min-width: 1.02em; height: 1.28em;"><svg width="400em" height="1.28em" viewBox="0 0 400000 1296" preserveAspectRatio="xMinYMin slice"><path d="M263,681c0.7,0,18,39.7,52,119
c34,79.3,68.167,158.7,102.5,238c34.3,79.3,51.8,119.3,52.5,120
c340,-704.7,510.7,-1060.3,512,-1067
l0 -0
c4.7,-7.3,11,-11,19,-11
H40000v40H1012.3
s-271.3,567,-271.3,567c-38.7,80.7,-84,175,-136,283c-52,108,-89.167,185.3,-111.5,232
c-22.3,46.7,-33.8,70.3,-34.5,71c-4.7,4.7,-12.3,7,-23,7s-12,-1,-12,-1
s-109,-253,-109,-253c-72.7,-168,-109.3,-252,-110,-252c-10.7,8,-22,16.7,-34,26
c-22,17.3,-33.3,26,-34,26s-26,-26,-26,-26s76,-59,76,-59s76,-60,76,-60z
M1001 80h400000v40h-400000z"></path></svg></span></span></span><span class="vlist-s">​</span></span><span class="vlist-r"><span class="vlist" style="height: 0.305em;"><span class=""></span></span></span></span></span><span class="mclose">)</span></span></span></span></span> can be generated using networkx’s  function \textit{bethe_hessian_matrix} and its negative eigenvalues can then be determined very easily since this matrix is symmetric so we can use a variety of fast solvers such as taking a <span class="katex--inline"><span class="katex"><span class="katex-mathml"><math xmlns="http://www.w3.org/1998/Math/MathML"><semantics><mrow><mi>Q</mi><mi>R</mi></mrow><annotation encoding="application/x-tex">QR</annotation></semantics></math></span><span class="katex-html" aria-hidden="true"><span class="base"><span class="strut" style="height: 0.87777em; vertical-align: -0.19444em;"></span><span class="mord mathnormal" style="margin-right: 0.00773em;">QR</span></span></span></span></span> decomposition using the Lanczos Algorithm \cite{trefethen1997numerical}.<br>
\newline<br>
The eigenvectors corresponding to the negative eigenvalues of the matrices <span class="katex--inline"><span class="katex"><span class="katex-mathml"><math xmlns="http://www.w3.org/1998/Math/MathML"><semantics><mrow><mi>H</mi><mo stretchy="false">(</mo><mo>±</mo><msqrt><mrow><mi>ρ</mi><mo stretchy="false">(</mo><mi>B</mi><mo stretchy="false">)</mo></mrow></msqrt><mo stretchy="false">)</mo></mrow><annotation encoding="application/x-tex">H(\pm \sqrt{\rho(B)})</annotation></semantics></math></span><span class="katex-html" aria-hidden="true"><span class="base"><span class="strut" style="height: 1.24em; vertical-align: -0.305em;"></span><span class="mord mathnormal" style="margin-right: 0.08125em;">H</span><span class="mopen">(</span><span class="mord">±</span><span class="mord sqrt"><span class="vlist-t vlist-t2"><span class="vlist-r"><span class="vlist" style="height: 0.935em;"><span class="svg-align" style="top: -3.2em;"><span class="pstrut" style="height: 3.2em;"></span><span class="mord" style="padding-left: 1em;"><span class="mord mathnormal">ρ</span><span class="mopen">(</span><span class="mord mathnormal" style="margin-right: 0.05017em;">B</span><span class="mclose">)</span></span></span><span class="" style="top: -2.895em;"><span class="pstrut" style="height: 3.2em;"></span><span class="hide-tail" style="min-width: 1.02em; height: 1.28em;"><svg width="400em" height="1.28em" viewBox="0 0 400000 1296" preserveAspectRatio="xMinYMin slice"><path d="M263,681c0.7,0,18,39.7,52,119
c34,79.3,68.167,158.7,102.5,238c34.3,79.3,51.8,119.3,52.5,120
c340,-704.7,510.7,-1060.3,512,-1067
l0 -0
c4.7,-7.3,11,-11,19,-11
H40000v40H1012.3
s-271.3,567,-271.3,567c-38.7,80.7,-84,175,-136,283c-52,108,-89.167,185.3,-111.5,232
c-22.3,46.7,-33.8,70.3,-34.5,71c-4.7,4.7,-12.3,7,-23,7s-12,-1,-12,-1
s-109,-253,-109,-253c-72.7,-168,-109.3,-252,-110,-252c-10.7,8,-22,16.7,-34,26
c-22,17.3,-33.3,26,-34,26s-26,-26,-26,-26s76,-59,76,-59s76,-60,76,-60z
M1001 80h400000v40h-400000z"></path></svg></span></span></span><span class="vlist-s">​</span></span><span class="vlist-r"><span class="vlist" style="height: 0.305em;"><span class=""></span></span></span></span></span><span class="mclose">)</span></span></span></span></span> are then used in the K-means algorithm to determine the communities. We used Scikit-Learn’s implementation of \textit{kmeans} to perform this.</p>
<p>\subsection{Alternatives to K-means}<br>
Throughout our synthetic and empirical examples of the Bethe Hessian spectral clustering approach, we solely used the K-means algorithm to generate the clusters. We chose K-means since it is fast and well known. Yet, there are many alternative clustering algorithms they may have been a better choice. A famous alternative worth mentioning is K-medians which, as it sounds takes the centroids on each iteration as the median of the data points in each provisional cluster instead of the mean \cite{MacQueen1967}. Another potential alternative is K-harmonic means proposed by \cite{zhang1999k} which uses the harmonic mean instead of the geometric mean for determining the centroids.</p>
<p>\subsection{Alternative choices of r in the Bethe Hessian}<br>
In our application of Bethe Hessian spectral clustering to networks of companies, our choice of <span class="katex--inline"><span class="katex"><span class="katex-mathml"><math xmlns="http://www.w3.org/1998/Math/MathML"><semantics><mrow><mi>r</mi><mo>=</mo><mo>±</mo><msqrt><mrow><mi>ρ</mi><mo stretchy="false">(</mo><mi>B</mi><mo stretchy="false">)</mo></mrow></msqrt></mrow><annotation encoding="application/x-tex">r = \pm\sqrt{\rho(B)}</annotation></semantics></math></span><span class="katex-html" aria-hidden="true"><span class="base"><span class="strut" style="height: 0.43056em; vertical-align: 0em;"></span><span class="mord mathnormal" style="margin-right: 0.02778em;">r</span><span class="mspace" style="margin-right: 0.277778em;"></span><span class="mrel">=</span><span class="mspace" style="margin-right: 0.277778em;"></span></span><span class="base"><span class="strut" style="height: 1.24em; vertical-align: -0.305em;"></span><span class="mord">±</span><span class="mord sqrt"><span class="vlist-t vlist-t2"><span class="vlist-r"><span class="vlist" style="height: 0.935em;"><span class="svg-align" style="top: -3.2em;"><span class="pstrut" style="height: 3.2em;"></span><span class="mord" style="padding-left: 1em;"><span class="mord mathnormal">ρ</span><span class="mopen">(</span><span class="mord mathnormal" style="margin-right: 0.05017em;">B</span><span class="mclose">)</span></span></span><span class="" style="top: -2.895em;"><span class="pstrut" style="height: 3.2em;"></span><span class="hide-tail" style="min-width: 1.02em; height: 1.28em;"><svg width="400em" height="1.28em" viewBox="0 0 400000 1296" preserveAspectRatio="xMinYMin slice"><path d="M263,681c0.7,0,18,39.7,52,119
c34,79.3,68.167,158.7,102.5,238c34.3,79.3,51.8,119.3,52.5,120
c340,-704.7,510.7,-1060.3,512,-1067
l0 -0
c4.7,-7.3,11,-11,19,-11
H40000v40H1012.3
s-271.3,567,-271.3,567c-38.7,80.7,-84,175,-136,283c-52,108,-89.167,185.3,-111.5,232
c-22.3,46.7,-33.8,70.3,-34.5,71c-4.7,4.7,-12.3,7,-23,7s-12,-1,-12,-1
s-109,-253,-109,-253c-72.7,-168,-109.3,-252,-110,-252c-10.7,8,-22,16.7,-34,26
c-22,17.3,-33.3,26,-34,26s-26,-26,-26,-26s76,-59,76,-59s76,-60,76,-60z
M1001 80h400000v40h-400000z"></path></svg></span></span></span><span class="vlist-s">​</span></span><span class="vlist-r"><span class="vlist" style="height: 0.305em;"><span class=""></span></span></span></span></span></span></span></span></span><br>
was fairly justified. \cite{le2015estimating} gives a full analysis of the different choices of r. According to their analysis, our choice of <span class="katex--inline"><span class="katex"><span class="katex-mathml"><math xmlns="http://www.w3.org/1998/Math/MathML"><semantics><mrow><mi>r</mi><mo>=</mo><mo>±</mo><msqrt><mrow><mi>ρ</mi><mo stretchy="false">(</mo><mi>B</mi><mo stretchy="false">)</mo></mrow></msqrt></mrow><annotation encoding="application/x-tex">r=\pm\sqrt{\rho(B)}</annotation></semantics></math></span><span class="katex-html" aria-hidden="true"><span class="base"><span class="strut" style="height: 0.43056em; vertical-align: 0em;"></span><span class="mord mathnormal" style="margin-right: 0.02778em;">r</span><span class="mspace" style="margin-right: 0.277778em;"></span><span class="mrel">=</span><span class="mspace" style="margin-right: 0.277778em;"></span></span><span class="base"><span class="strut" style="height: 1.24em; vertical-align: -0.305em;"></span><span class="mord">±</span><span class="mord sqrt"><span class="vlist-t vlist-t2"><span class="vlist-r"><span class="vlist" style="height: 0.935em;"><span class="svg-align" style="top: -3.2em;"><span class="pstrut" style="height: 3.2em;"></span><span class="mord" style="padding-left: 1em;"><span class="mord mathnormal">ρ</span><span class="mopen">(</span><span class="mord mathnormal" style="margin-right: 0.05017em;">B</span><span class="mclose">)</span></span></span><span class="" style="top: -2.895em;"><span class="pstrut" style="height: 3.2em;"></span><span class="hide-tail" style="min-width: 1.02em; height: 1.28em;"><svg width="400em" height="1.28em" viewBox="0 0 400000 1296" preserveAspectRatio="xMinYMin slice"><path d="M263,681c0.7,0,18,39.7,52,119
c34,79.3,68.167,158.7,102.5,238c34.3,79.3,51.8,119.3,52.5,120
c340,-704.7,510.7,-1060.3,512,-1067
l0 -0
c4.7,-7.3,11,-11,19,-11
H40000v40H1012.3
s-271.3,567,-271.3,567c-38.7,80.7,-84,175,-136,283c-52,108,-89.167,185.3,-111.5,232
c-22.3,46.7,-33.8,70.3,-34.5,71c-4.7,4.7,-12.3,7,-23,7s-12,-1,-12,-1
s-109,-253,-109,-253c-72.7,-168,-109.3,-252,-110,-252c-10.7,8,-22,16.7,-34,26
c-22,17.3,-33.3,26,-34,26s-26,-26,-26,-26s76,-59,76,-59s76,-60,76,-60z
M1001 80h400000v40h-400000z"></path></svg></span></span></span><span class="vlist-s">​</span></span><span class="vlist-r"><span class="vlist" style="height: 0.305em;"><span class=""></span></span></span></span></span></span></span></span></span> is valid since the networks were quite general and we could not be sure that they were very closely aligned with an SBM-like structure. However, they do mention a good approximation:<br>
<span class="katex--display"><span class="katex-display"><span class="katex"><span class="katex-mathml"><math xmlns="http://www.w3.org/1998/Math/MathML" display="block"><semantics><mrow><mi>ρ</mi><mo stretchy="false">(</mo><mi>B</mi><mo stretchy="false">)</mo><mo>=</mo><mfrac><mrow><munderover><mo>∑</mo><mrow><mi>i</mi><mo>=</mo><mn>1</mn></mrow><mi>n</mi></munderover><msubsup><mi>d</mi><mi>i</mi><mn>2</mn></msubsup></mrow><mrow><munderover><mo>∑</mo><mrow><mi>i</mi><mo>=</mo><mn>1</mn></mrow><mi>n</mi></munderover><msub><mi>d</mi><mi>i</mi></msub></mrow></mfrac><mo>−</mo><mn>1</mn></mrow><annotation encoding="application/x-tex">\rho(B) = \frac{\sum_{i=1}^{n} d_i^2}{\sum_{i=1}^{n} d_i} -1</annotation></semantics></math></span><span class="katex-html" aria-hidden="true"><span class="base"><span class="strut" style="height: 1em; vertical-align: -0.25em;"></span><span class="mord mathnormal">ρ</span><span class="mopen">(</span><span class="mord mathnormal" style="margin-right: 0.05017em;">B</span><span class="mclose">)</span><span class="mspace" style="margin-right: 0.277778em;"></span><span class="mrel">=</span><span class="mspace" style="margin-right: 0.277778em;"></span></span><span class="base"><span class="strut" style="height: 2.49782em; vertical-align: -0.994002em;"></span><span class="mord"><span class="mopen nulldelimiter"></span><span class="mfrac"><span class="vlist-t vlist-t2"><span class="vlist-r"><span class="vlist" style="height: 1.50382em;"><span class="" style="top: -2.30571em;"><span class="pstrut" style="height: 3em;"></span><span class="mord"><span class="mop"><span class="mop op-symbol small-op" style="position: relative; top: -5e-06em;">∑</span><span class="msupsub"><span class="vlist-t vlist-t2"><span class="vlist-r"><span class="vlist" style="height: 0.804292em;"><span class="" style="top: -2.40029em; margin-left: 0em; margin-right: 0.05em;"><span class="pstrut" style="height: 2.7em;"></span><span class="sizing reset-size6 size3 mtight"><span class="mord mtight"><span class="mord mathnormal mtight">i</span><span class="mrel mtight">=</span><span class="mord mtight">1</span></span></span></span><span class="" style="top: -3.2029em; margin-right: 0.05em;"><span class="pstrut" style="height: 2.7em;"></span><span class="sizing reset-size6 size3 mtight"><span class="mord mtight"><span class="mord mathnormal mtight">n</span></span></span></span></span><span class="vlist-s">​</span></span><span class="vlist-r"><span class="vlist" style="height: 0.29971em;"><span class=""></span></span></span></span></span></span><span class="mspace" style="margin-right: 0.166667em;"></span><span class="mord"><span class="mord mathnormal">d</span><span class="msupsub"><span class="vlist-t vlist-t2"><span class="vlist-r"><span class="vlist" style="height: 0.311664em;"><span class="" style="top: -2.55em; margin-left: 0em; margin-right: 0.05em;"><span class="pstrut" style="height: 2.7em;"></span><span class="sizing reset-size6 size3 mtight"><span class="mord mathnormal mtight">i</span></span></span></span><span class="vlist-s">​</span></span><span class="vlist-r"><span class="vlist" style="height: 0.15em;"><span class=""></span></span></span></span></span></span></span></span><span class="" style="top: -3.23em;"><span class="pstrut" style="height: 3em;"></span><span class="frac-line" style="border-bottom-width: 0.04em;"></span></span><span class="" style="top: -3.68971em;"><span class="pstrut" style="height: 3em;"></span><span class="mord"><span class="mop"><span class="mop op-symbol small-op" style="position: relative; top: -5e-06em;">∑</span><span class="msupsub"><span class="vlist-t vlist-t2"><span class="vlist-r"><span class="vlist" style="height: 0.804292em;"><span class="" style="top: -2.40029em; margin-left: 0em; margin-right: 0.05em;"><span class="pstrut" style="height: 2.7em;"></span><span class="sizing reset-size6 size3 mtight"><span class="mord mtight"><span class="mord mathnormal mtight">i</span><span class="mrel mtight">=</span><span class="mord mtight">1</span></span></span></span><span class="" style="top: -3.2029em; margin-right: 0.05em;"><span class="pstrut" style="height: 2.7em;"></span><span class="sizing reset-size6 size3 mtight"><span class="mord mtight"><span class="mord mathnormal mtight">n</span></span></span></span></span><span class="vlist-s">​</span></span><span class="vlist-r"><span class="vlist" style="height: 0.29971em;"><span class=""></span></span></span></span></span></span><span class="mspace" style="margin-right: 0.166667em;"></span><span class="mord"><span class="mord mathnormal">d</span><span class="msupsub"><span class="vlist-t vlist-t2"><span class="vlist-r"><span class="vlist" style="height: 0.814108em;"><span class="" style="top: -2.44134em; margin-left: 0em; margin-right: 0.05em;"><span class="pstrut" style="height: 2.7em;"></span><span class="sizing reset-size6 size3 mtight"><span class="mord mathnormal mtight">i</span></span></span><span class="" style="top: -3.063em; margin-right: 0.05em;"><span class="pstrut" style="height: 2.7em;"></span><span class="sizing reset-size6 size3 mtight"><span class="mord mtight">2</span></span></span></span><span class="vlist-s">​</span></span><span class="vlist-r"><span class="vlist" style="height: 0.258664em;"><span class=""></span></span></span></span></span></span></span></span></span><span class="vlist-s">​</span></span><span class="vlist-r"><span class="vlist" style="height: 0.994002em;"><span class=""></span></span></span></span></span><span class="mclose nulldelimiter"></span></span><span class="mspace" style="margin-right: 0.222222em;"></span><span class="mbin">−</span><span class="mspace" style="margin-right: 0.222222em;"></span></span><span class="base"><span class="strut" style="height: 0.64444em; vertical-align: 0em;"></span><span class="mord">1</span></span></span></span></span></span>,</p>
<p>where <span class="katex--inline"><span class="katex"><span class="katex-mathml"><math xmlns="http://www.w3.org/1998/Math/MathML"><semantics><mrow><msub><mi>d</mi><mi>i</mi></msub></mrow><annotation encoding="application/x-tex">d_i</annotation></semantics></math></span><span class="katex-html" aria-hidden="true"><span class="base"><span class="strut" style="height: 0.84444em; vertical-align: -0.15em;"></span><span class="mord"><span class="mord mathnormal">d</span><span class="msupsub"><span class="vlist-t vlist-t2"><span class="vlist-r"><span class="vlist" style="height: 0.311664em;"><span class="" style="top: -2.55em; margin-left: 0em; margin-right: 0.05em;"><span class="pstrut" style="height: 2.7em;"></span><span class="sizing reset-size6 size3 mtight"><span class="mord mathnormal mtight">i</span></span></span></span><span class="vlist-s">​</span></span><span class="vlist-r"><span class="vlist" style="height: 0.15em;"><span class=""></span></span></span></span></span></span></span></span></span></span> denotes the degree of the <span class="katex--inline"><span class="katex"><span class="katex-mathml"><math xmlns="http://www.w3.org/1998/Math/MathML"><semantics><mrow><mi>i</mi></mrow><annotation encoding="application/x-tex">i</annotation></semantics></math></span><span class="katex-html" aria-hidden="true"><span class="base"><span class="strut" style="height: 0.65952em; vertical-align: 0em;"></span><span class="mord mathnormal">i</span></span></span></span></span>th node.<br>
\newline<br>
This is useful since it reduces the computational cost, particularly for large networks. Since our novel empirical networks were not too large, we did not utilize this but it is worth making note of. Furthermore, for SBMs, it was argued by \cite{SaadeBethe} that the best choice was <span class="katex--inline"><span class="katex"><span class="katex-mathml"><math xmlns="http://www.w3.org/1998/Math/MathML"><semantics><mrow><mi>r</mi><mo>=</mo><msqrt><mrow><mfrac><mn>1</mn><mi>n</mi></mfrac><msubsup><mo>∑</mo><mrow><mi>i</mi><mo>=</mo><mn>1</mn></mrow><mi>n</mi></msubsup><msub><mi>d</mi><mi>i</mi></msub></mrow></msqrt></mrow><annotation encoding="application/x-tex">r = \sqrt{\frac{1}{n}\sum_{i=1}^n d_i}</annotation></semantics></math></span><span class="katex-html" aria-hidden="true"><span class="base"><span class="strut" style="height: 0.43056em; vertical-align: 0em;"></span><span class="mord mathnormal" style="margin-right: 0.02778em;">r</span><span class="mspace" style="margin-right: 0.277778em;"></span><span class="mrel">=</span><span class="mspace" style="margin-right: 0.277778em;"></span></span><span class="base"><span class="strut" style="height: 1.84em; vertical-align: -0.604946em;"></span><span class="mord sqrt"><span class="vlist-t vlist-t2"><span class="vlist-r"><span class="vlist" style="height: 1.23505em;"><span class="svg-align" style="top: -3.8em;"><span class="pstrut" style="height: 3.8em;"></span><span class="mord" style="padding-left: 1em;"><span class="mord"><span class="mopen nulldelimiter"></span><span class="mfrac"><span class="vlist-t vlist-t2"><span class="vlist-r"><span class="vlist" style="height: 0.845108em;"><span class="" style="top: -2.655em;"><span class="pstrut" style="height: 3em;"></span><span class="sizing reset-size6 size3 mtight"><span class="mord mtight"><span class="mord mathnormal mtight">n</span></span></span></span><span class="" style="top: -3.23em;"><span class="pstrut" style="height: 3em;"></span><span class="frac-line" style="border-bottom-width: 0.04em;"></span></span><span class="" style="top: -3.394em;"><span class="pstrut" style="height: 3em;"></span><span class="sizing reset-size6 size3 mtight"><span class="mord mtight"><span class="mord mtight">1</span></span></span></span></span><span class="vlist-s">​</span></span><span class="vlist-r"><span class="vlist" style="height: 0.345em;"><span class=""></span></span></span></span></span><span class="mclose nulldelimiter"></span></span><span class="mspace" style="margin-right: 0.166667em;"></span><span class="mop"><span class="mop op-symbol small-op" style="position: relative; top: -5e-06em;">∑</span><span class="msupsub"><span class="vlist-t vlist-t2"><span class="vlist-r"><span class="vlist" style="height: 0.804292em;"><span class="" style="top: -2.40029em; margin-left: 0em; margin-right: 0.05em;"><span class="pstrut" style="height: 2.7em;"></span><span class="sizing reset-size6 size3 mtight"><span class="mord mtight"><span class="mord mathnormal mtight">i</span><span class="mrel mtight">=</span><span class="mord mtight">1</span></span></span></span><span class="" style="top: -3.2029em; margin-right: 0.05em;"><span class="pstrut" style="height: 2.7em;"></span><span class="sizing reset-size6 size3 mtight"><span class="mord mathnormal mtight">n</span></span></span></span><span class="vlist-s">​</span></span><span class="vlist-r"><span class="vlist" style="height: 0.29971em;"><span class=""></span></span></span></span></span></span><span class="mspace" style="margin-right: 0.166667em;"></span><span class="mord"><span class="mord mathnormal">d</span><span class="msupsub"><span class="vlist-t vlist-t2"><span class="vlist-r"><span class="vlist" style="height: 0.311664em;"><span class="" style="top: -2.55em; margin-left: 0em; margin-right: 0.05em;"><span class="pstrut" style="height: 2.7em;"></span><span class="sizing reset-size6 size3 mtight"><span class="mord mathnormal mtight">i</span></span></span></span><span class="vlist-s">​</span></span><span class="vlist-r"><span class="vlist" style="height: 0.15em;"><span class=""></span></span></span></span></span></span></span></span><span class="" style="top: -3.19505em;"><span class="pstrut" style="height: 3.8em;"></span><span class="hide-tail" style="min-width: 1.02em; height: 1.88em;"><svg width="400em" height="1.8800000000000001em" viewBox="0 0 400000 1944" preserveAspectRatio="xMinYMin slice"><path d="M983 90
l0 -0
c4,-6.7,10,-10,18,-10 H400000v40
H1013.1s-83.4,268,-264.1,840c-180.7,572,-277,876.3,-289,913c-4.7,4.7,-12.7,7,-24,7
s-12,0,-12,0c-1.3,-3.3,-3.7,-11.7,-7,-25c-35.3,-125.3,-106.7,-373.3,-214,-744
c-10,12,-21,25,-33,39s-32,39,-32,39c-6,-5.3,-15,-14,-27,-26s25,-30,25,-30
c26.7,-32.7,52,-63,76,-91s52,-60,52,-60s208,722,208,722
c56,-175.3,126.3,-397.3,211,-666c84.7,-268.7,153.8,-488.2,207.5,-658.5
c53.7,-170.3,84.5,-266.8,92.5,-289.5z
M1001 80h400000v40h-400000z"></path></svg></span></span></span><span class="vlist-s">​</span></span><span class="vlist-r"><span class="vlist" style="height: 0.604946em;"><span class=""></span></span></span></span></span></span></span></span></span>.</p>
<p>\subsection{Weighted Bethe Hessian}<br>
It is mentioned in \cite{SaadeBethe} that the Bethe-Hessian can be generalised to a weighted version of itself which can then be used for detecting block structure in weighted networks. In our application to company share prices, we generated a network of weighted links where each link was then set to unit weight or destroyed depending, respectively, on whether they had surpassed a given threshold or not. We could have attempted instead to carry out our experiments keeping the weighted links and using the weighted Bethe Hessian. This was decided against since stock market data is incredibly noisy and so the correlations were unlikely to be incredibly accurate anyway. This extension is definitely worth considering for other applications where the weights of links between nodes are reliable.</p>
<p>\section{Conclusion}<br>
We have reviewed spectral methods for community detection in networks, focusing on a new method that finds assortative and disassortative communities or ‘blocks’. We have demonstrated this method to have more success than older spectral clustering approaches. The Bethe Hessian spectral clustering approach provides an answer to the question of how many communities to check for and whether they are assortative or disassortative in structure; a question that other methods often fail to provide a certain answer to.<br>
\newline<br>
We then demonstrated a real-world possible application of this method to financial markets. We formed networks based on strong positive or negative correlations between share prices of companies and used Bethe Hessian spectral clustering to identify block structure in these networks. As these correlations were drawn from noisy data, we placed reasonably high thresholds on the correlations to justify a link between two nodes. This in turn produced moderately sparse networks which other spectral methods struggled to cluster. This further illustrated the Bethe Hessian’s merits.<br>
\newline<br>
Finally, we discussed a few other potential directions of exploration furthering what we have discussed. We briefly mention  other choices of clustering algorithm to K-means and also the possibility of generalising to weighted networks by using the weighted Bethe Hessian matrix. Overall, using the Bethe Hessian for determining block structure in networks is a  powerful method, eclipsing the (normalized) Laplacian in most regards and with a fairly low computational cost if implemented nicely.<br>
\showacknow{} % Display the acknowledgments section</p>

